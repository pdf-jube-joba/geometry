\documentclass[dvipdfmx]{jsarticle}
% パッケージ
\usepackage{amsthm}
\usepackage{amsmath,amssymb}
\usepackage{color}
\usepackage{tikz}

% 定理環境
%% 本体
\theoremstyle{definition}
\newtheorem*{tDefinition}{定義}
\newtheorem*{tTheorem}{定理}
\newtheorem*{tProof}{証明}
\newtheorem*{tNotation}{記法}
\newtheorem*{tRemark}{注意}
\newtheorem*{tWhen}{設定}

\newenvironment{Mini}{
  \begin{minipage}[t]{0.9\hsize}
  \setlength{\parindent}{12pt}
  \begin{itemize}
  \setlength{\labelsep}{10pt}
}{
  \end{itemize}
  \vspace{5pt}
  \end{minipage}
}

\newenvironment{Definition}{
  \begin{tDefinition}
  \begin{Mini}
}{
  \end{Mini}
  \end{tDefinition}
}

\newenvironment{Theorem}{
  \begin{tTheorem}
  \begin{Mini}
}{
  \end{Mini}
  \end{tTheorem}
}

\newenvironment{Proof}{
  \begin{tProof}
  \begin{Mini}
      }{
  \end{Mini}
  \end{tProof}
}

\newenvironment{When}{
  \begin{tWhen}
  \begin{Mini}
}{
  \end{Mini}
  \end{tWhen}
}

%% マーク
\newcommand{\itemwhen}{\item[\(\bigcirc\)]}
\newcommand{\itemnote}{\item[!]}

\newcommand{\itemdefi}{\item[\(\square\)]}
\newcommand{\itemprop}{\item[\(\vartriangleright\)]}
\newcommand{\itemand}{\item[\(-\)]}

\newcommand{\itemprof}{\item[\(\because\)]}
\newcommand{\itemthen}{\item[\(\rightsquigarrow\)]}

\newcommand{\itemenum}{\item[\(\bullet\)]}
\newcommand{\itemwith}{\item[\(-\)]}

% 記述関係
\newenvironment{indentblock}{
  \\
  \hspace*{5mm}
  \begin{minipage}{0.8\textwidth}
}{
  \end{minipage}
  \\
}

%コマンド関連

\newcommand{\WIP}{\textcolor{red}{工事中}}
\newcommand{\SORRY}{\textcolor{red}{わかりませんでした}}
\newcommand{\ADMIT}{\textcolor{blue}{認めます}}
\newcommand{\AIMAI}[1]{\textit{#1}}

\begin{document}

\section*{The real category}

\begin{Definition}[real space]
\itemdefi
  \Define a space with involution (real space structure) :=
  \begin{itemize}
    \itembase \(X\) : topological space
    \itemenum involution \(\tau\) : \(X \to X\) s.t. \(\tau^2=1\)
  \end{itemize}
\end{Definition}

\begin{Definition}[real map (map between real spaces)]
\itemwhen
  \For \(X,Y\) : real space, \(f\) : \(X \to Y\)
\itemdefi
  \Define \(f\) is real := \AIMAI{\(f\) commutes with the involution} i.e.: \(f(\tau(x)) = \tau(f(x))\)
\end{Definition}

\begin{Definition}[real vector bundle]
\itemwhen
  \For \(X\) : real space
\itemdefi
  \Define a real vector bundle over \(X\) :=
  \begin{itemize}
    \itembase \(E\) : \(\mathbb{C}\)-vector bundle over \(X\)
    \itemenum real space structure on \(E\)
    \itemwith projection is real
    \itemwith \(v \in E_x \to \tau(v) \in E_{\tau(x)}\) is \AIMAI{anti-linear} i.e.: \(\tau(cv) = \bar{c} \tau(v)\) for \(c:\mathbb{C}, v:E\)
  \end{itemize}
\itemnote
  (real vector bundle over real space) and (\(\mathbb{C}\)-vector bundle in the category of \(\mathbb{Z}_2\)) are different. In the definition of the latter, the map \(E_x \to E_{\tau (x)}\) is assumed to be complex-linear.
\end{Definition}

\begin{Definition}[two different structure on complexification of \(\mathbb{R}\)-vector bundle in \(\mathbb{Z}_2\)-category]
\itemwhen
  \For \(E \to X\) : \(\mathbb{R}\)-vector bundle in the category of \(\mathbb{Z}_2\)-spaces.
\itemdefi
  \Define real vector bundle structure on \(E \otimes_{\mathbb{R}} \mathbb{C}\) over \(X\) :=
  \begin{itemize}
    \itembase \(\mathbb{C}\)-vector bundle structure on \(E \otimes \mathbb{C}\) over \(X\) := 
    \itemenum real space structure := linear extension of \(E_{x} \to E_{\tau(x)}\)
  \end{itemize}
  \Define \(\mathbb{C}\)-vector bundle structure in the \(\mathbb{Z}_2\) category on \(E\) over \(X\) :=
  \begin{itemize}
    \itembase \(\mathbb{C}\)-vector bundle structure on \(E \otimes \mathbb{C}\) over \(X\)
    \itemenum real space structure := anti-linear extension of \(E_{x} \to E_{\tau(x)}\)
  \end{itemize}
\end{Definition}

\begin{Definition}[real point]
\itemwhen
  \For \(X\) : real space , \(x\) : \(X\)
\itemdefi
  \Define \(x\) is real point of \(X\) := \(x\) is fixed point of \(\tau\) \\
  \Define \(X_{\text{real}}\) : subset of \(X\) := set of fixed point
\end{Definition}

\begin{Definition}
\itemwhen
  \For \(X\) : real space, \(Y\) : subset of \(X\)
\itemdefi
  \Define \(Y\) is real subspace := \(\tau(Y) = Y\)
\end{Definition}

\begin{Theorem}
\itemwhen
  \For \(E \to X\) : rl.vec.bnd over rl.sp, \(x\) : real point of \(X\)
\itemprop
  \Let \(V_{x}\) : subspace of \(E_{x}\) := \(+1\)-eigenspace of anti-linear map \(\tau_{x}\) \\
  \Then \(E_{x} \cong V_{x} \otimes_{\mathbb{R}} \mathbb{C}\)
\end{Theorem}

\begin{Theorem}
\itemwhen
  \For \(X\) : real space with trivial involution 
\itemdefi
  \Define \(\mathcal{E}(X)\) : category := category of real vector bundles over \(X\) as space \\
  \Define \(\mathcal{F}(X)\) : category := category of real vector bundels over \(X\) as real space \\
  \Let \(\mathcal{E}(X) \to \mathcal{F}(X)\) := \(E \mapsto E \otimes_{\mathbb{R}} \mathbb{C}\) \\
  \Let \(\mathcal{F}(X) \to \mathcal{E}(X)\) := \(F \mapsto F_{\text{real}}\)
\itemprop
  \Then these makes \(\mathcal{E}(X), \mathcal{F}(X)\) equivalent
\end{Theorem}

\begin{Theorem}
\itemwhen
  \For \(E \to X\) : rl.vec.bnd over rl.sp, \(s\) : \(\Gamma(E)\)
\itemdefi
  \Define \(\tau(s)\) : \(\Gamma(E)\) := \(x \mapsto \tau(s(\tau(x)))\) \\
  \Define real structure on \(\Gamma(E)\) := \(s \mapsto \tau(s)\) \\
  \Then \(\Gamma(E)_{\text{real}} \otimes_{\mathbb{R}} \mathbb{C} \cong \Gamma(E)\)
\end{Theorem}

\begin{Definition}
\itemwhen
  \For \(E,F\) : rl.vec.bnd over \(X\)
\itemdefi
  \Define morphism \(E \to F\) := \(\phi\) : homomorphism of complex vector bundle s.t. \(\phi(\tau(v)) = \tau(\phi(v))\)
\end{Definition}

\begin{Definition}
\itemwhen
  \For \(E,F\) : rl.vec.bnd over \(X\)
\itemdefi
  \Define \(E \otimes F\) : rl.vec.bnd over \(X\) := \AIMAI{natural real vector bundle structure on \(E \otimes_{\mathbb{C}} F\)} \\
  \Define \(\text{Hom}(E,F)\) : rl.vec.bnd over \(X\) :=
  \begin{itemize}
    \itembase \(\text{Hom}(E,F)\)
    \itemenum \(\phi \mapsto e \mapsto \tau(\phi(\tau(e)))\)
  \end{itemize}
\end{Definition}

\begin{Theorem}
\itemwhen
  \For \(E,F\) : rl.vec.bnd over \(X\)
\itemprop
  \Then \((E \to F) \cong \Gamma(\text{Hom}(E,F))\)
\end{Theorem}


\begin{Theorem}
\itemwhen
  \For \(E \to X\) : rl.vec.bnd over rl.sp, \(Y\) : rl.sub.sp of \(X\), \(s\) : \(\Gamma(\restr{E}{Y})\)
\itemprop
  \Then \(\exists\) \(\tilde{s}\) : \(\Gamma(E)\) s.t. extension of \(s\)
\end{Theorem}

\begin{Definition}
\itemdefi
  \Define real algebraic vector bundle :=
  \begin{itemize}
    \itembase \((\pi : E \to X)\) : complex algebraic vector bundle
    \itemenum scalar multiplication defined over \(\mathbb{R}\) : \(\mathbb{C} \times E \to E\)
  \end{itemize}
  \Define topological real algebraic vector bundle := take underlying topological spaces and involution
\itemdefi
  \Let \(\mathbb{CP}^{n}\) : real space := \AIMAI{usual definition} \\
  \Let \(E\) : real bundle over \(\mathbb{CP}^{n}\) := \(\{(u,v) \in \mathbb{CP}^{n} \times \mathbb{C}^{n} \mid \sum_{i} u_{i} v_{i} = 0\}\) \\
  \Define standard line bundle over \(\mathbb{CP}^{n}\) : real bundle over \(\mathbb{CP}^{n}\) := \\
  \(H\) of exact vector bundle seqence \(0 \to E \to \mathbb{CP}^{n} \times \mathbb{C}^{n} \to H \to 0\)
\itemdefi
  \SORRY
  another example のところはよくわからなかった( affine ring の上の projective module が"どの" module を指した言葉なのかわからん)それとも各 projective module に対して real vector bundle が定まると言ってる?
\end{Definition}

\begin{Definition}
\itemwhen
  \For \(X\) : real space
\itemprop
  \Define \(KR(X)\) : Abelian group := \\
  Grothendieck group of the category of real vector bundles over \(X\) \\
  \Define ring structure on \(KR(X)\) := by tensor product
\end{Definition}

\begin{Theorem}
\itemprop
  \Then \(KR(X_{\text{real}}) \cong KO(X_{\text{real}})\)
\itemprop
  \Let \(KR(X) \to KO(X_{\text{real}})\) := \(KR(X) \to KR(X_{\text{real}}) \to KO(X_{\text{real}})\) \\
  \IfHold \(X = X_{\text{real}}\) \\
  \Then \(KR(X) \cong KO(X)\)
\itemprop
  \Then \({\mathbb{CP}^{n}}_{\text{real}} \cong \mathbb{RP}^{n}\) \\
  \Then image of standard line bundle under \(KR(\mathbb{CP}^{n}) \to KO(\mathbb{RP}^{n})\) is real Hopf bundle
\end{Theorem}

\begin{Definition}
\itemdefi
  \Define \(c\) : \(KR(X) \to K(X)\) := ignore involution \\
  \Define \(r\) : \(K(X) \to KR(X)\) := \(E \mapsto E \oplus \tau^{*} \bar{E}\)
\itemprop
  \IfHold \(X = X_{\text{real}}\) \\
  \Then this coincidents complexification and realification under \(KR(X) \cong KO(X)\)
\end{Definition}

\section*{The periodicity theorem}

\begin{Definition}
\itemdefi
  \For \(E \to X\) : rl.pair \\
  \Define \(P(E)\) : real space := \AIMAI{projective bundle of \(E\) with natural involution} \\
  \Define standard line-bundle over \(P(E)\) : real line-bundle := \AIMAI{by standard line-bundle of \(P(E)\) with natural involution}
\end{Definition}

\begin{Definition}
\itemdefi
  \For \(E \to X\) : rl.pair \\
  \Define \(KR(X)\)-algebra structure on \(KR(P(E))\) := \AIMAI{よくある形}
\end{Definition}

\begin{Theorem}
\itemprop
  \For \(X\) : real compact space, \(L\) : real line-bundle over \(X\) \\
  \Let \(H\) : real line-bundle over \(P(L \oplus 1)\) := standard line bundle over \(P(L \oplus 1)\) \\
  \Then \(KR(P(L \oplus 1))\) is generated by \(H\) subject to the single relation \(([H] - [1])([L][H] - 1) = 0\)
\end{Theorem}

\begin{Proof}
\itemprof
  関係式が成り立つことを示す。
  \(L\) 上の metric で involution に対して invariant なものをとる。
  \(S\) := unit circle bundle of \(L\) with respect to metric は real space になる。
  \(z\) : section of \(\pi*L\) := defined by the inclusion \(S \to L\) とすると、 \(z, z^k\) もまた real section である。
  したがって、複素束としての同型 \(H^k \cong (1, z^{-k}, L^{-k})\) は real bundle としての同型でもある。
\itemthen
  \(f\) : real section of \(\text{Hom}(\pi^*{E^0}, \pi^*{E^\infty})\) に対してその Fourier coefficients \(a_{k}\) を計算すると \(\text{Hom}(L^{k} \otimes E^0, E^\infty)\) の元であることがわかる。
  \begin{align*}
    \tau(a_k)(x) = \tau(a_k (\tau(x)))
    &= - \frac{1}{2 \pi i} \tau(\int_{S_{\tau(x)}} f_{\tau(x)} z_{\tau(x)}^{-k - 1} d {z_{\tau(x)}}) \\
    &= \frac{1}{2 \pi i} \int_{S_{x}} \tau(f)_{\tau(x)}(\tau(z_{\tau(x)}))^{-k-1} d {\tau(z_{\tau(x)})} \\
    &= \frac{1}{2 \pi i} \int_{S_{x}} f_{x} {z_{x}}^{-k-1} d z_{x} = a_{k}(x)
  \end{align*}
  特に、 \(X\) の real point \(x\) では \(f_{x}\) が real であることは \(f_{x}(e^{-i\theta}) = \tau(f_{x}(e^{i \theta}))\) を満たすということで Fourier coefficients が係数になる。
  こういった形で、複素の場合に示した同様の証明における構成が全て real になることがわかるためよい。
\end{Proof}

\begin{Definition}
\itemdefi
  \Define \(\mathbb{R}\) : real space := base \(\mathbb{R}\) with trivial involution \\
  \Define \(i\mathbb{R}\) : real space := base \(\mathbb{R}\) with involution \(x \mapsto -x\)
\itemdefi
  \Define \(\mathbb{R}^{p,q}\) : real space := \(\mathbb{R}^{p} \oplus i\mathbb{R}^{q}\) \\
  \Define \(B^{p,q}\) : real subspace of \(\mathbb{R}^{p,q}\) := unit ball in \(\mathbb{R}^{p,q}\) \\
  \Define \(S^{p,q}\) : real subspace of \(B^{p,q}\) := unit sphere in \(\mathbb{R}^{p,q}\) \\
\itemprop
  \Then \(\mathbb{R}^{p,p} \cong \mathbb{C}\)
\end{Definition}

\begin{Definition}
\itemdefi
  \For \(X\) : pointed real space \\
  \Define \(\widetilde{KR}(X)\) : Abelian group := kernel of restriction to base point \\
\itemdefi
  \For \((X,Y)\) : real space pair \\
  \Define \(KR(X,Y)\) : Abelian group := \(\widetilde{KR}(X/Y)\)
\itemdefi
  \Define \(KR^{p,q}(X,Y)\) : Abelian group := \(KR((X,Y) \times (B^{p,q}, S^{p,q}))\) \\
  \Define \(KR^{-q}(X,Y)\) : Abelian group := \(KR^{0,q}(X,Y)\) \\
  \Define \(KR^{p}(X,Y)\) : Abelian group := \(KR^{p,0}(X,Y)\)
\itemdefi
  \Define exterior product : \(KR^{p_1,q_1}(X_1,Y_1) \otimes KR^{p_2,q_2}(X_2,Y_2) \to KR^{p_1+p_2, q_1+q_2}((X_1,Y_1) \times (X_2,y_2))\) := \AIMAI{by natural way} \\
  \Define internal product := \AIMAI{by natural way}
\end{Definition}

\begin{Theorem}
\itemprop
  \Then \(\cdots \to KR^{p,1}(X) \to KR^{p,1}(Y) \to KR^{p,0}(X,Y) \to KR^{p,0}(X) \to KR^{p,0}(Y)\) is exact
\itemprop
  \Let \(b\) : \(KR^{1,1}(pt)\) := \([H] - 1\) \\
  \Define \(\beta\) : \(KR^{p,q}(X,Y) \to KR^{p+1,q+1}(X,Y)\) \\
  \Then \(\beta\) is isomorphism
\itemprop
  \Then \(KR^{p,q} \cong KR^{p-q}\)
\end{Theorem}

\begin{Theorem}
\itemprop
  \For \(E \to X\) : rl.pair \\
  \Let \(\lambda_E\) := exterior algebra of \(E\) \\
  \Then \(KR(X) \to \widetilde{KR}(\text{Thom}(E))\) := \(x \mapsto \lambda_E x\)
\end{Theorem}

\begin{Theorem}
\itemprop
  \Let \(i\) : \(\mathbb{R}^{0,1} \to \mathbb{R}^{1,1}\) := by inclusion of \(\mathbb{R} \subset \mathbb{C}\) \\
  \Then \(i^*([H] - 1)\) is the reduced Hopf bundle
\itemdefi
  \Define \(\eta\) := \(i^*([H] - 1)\)
\end{Theorem}

\section*{Coefficient theories}

\begin{Theorem}
\itemprop
  \For \(Y\) : real space \\
  \Define \(KR\)-theory with coefficient theorem in \(Y\) : cohomology theory on the category of real spaces := \(X \mapsto KR(X \times Y)\)
\itemprop
  \Then \(KR\)-theory with coefficient in \(S^{p,0}\) has period \((2,4,8)\) if \(p = 1,2,4\)
\end{Theorem}

\begin{Proof}
\itemprof
  \Let \(\mathbb{R}^{p} \cong \mathbb{R}, \mathbb{C}, \mathbb{H}\) \\
  \Let \(\mu_{p}\) : \(X \times S^{p,0} \times \mathbb{R}^{0,p} \to X \times S^{p,0} \times \mathbb{R}^{p,0}\) := \((x,s,u) \mapsto (x,s,su)\) \\
  \Then \(\mu_{p}^{*}\) is isomorphism \\
  \Let isomorphism \(KR^{p,q}(X \times S^{p,0}) \to KR^{0,p+q}(X \times S^{p,0})\) := using suspension and above isomorphism \\
  \Let isomorphism \(KR(X \times S^{p,0}) \to KR^{-2p}(X \times S^{p,0})\) := using \(\beta\) and above isomorphism
\end{Proof}

\begin{Theorem}
\itemprop
  \Let \(\pi\) : \(S^{p,0} \to \text{pt}\) := projection \\
  \Let \(\chi\) : \(KR^{p-q}(X) \to KR^{-q}(X)\) := product with \((-\eta)^{p}\) where \(\eta\) is the reduced real Hopf bundle \\
  \Then \(\cdots \to KR^{p-q}(X) \overset{\chi}{\to} KR^{-q}(X) \overset{\pi^*}{\to} KR^{-q}(X \times S^{p,0}) \to \cdots \) is exact
\end{Theorem}

\begin{Theorem}
\itemnote
  \(p=1\) の場合の考察
\itemprop
  \(KR(X \times S^{1,0}) \cong K(X)\)
\itemprop
  \(K(X) \cong K^{-2}(X)\)
\end{Theorem}

\begin{Theorem}
\itemnote
  \(p=2\) の場合の考察
\itemprop
  (isomorphism classes of real bundles over \(X \times S^{2,0}\)) and (homotopy classes of self-conjugate bundles over \(X\)) are correspond bijectively compatible with tensor products
\itemprop
  \(KSC(X) \cong KR(X \times S^{2,0})\)
\end{Theorem}

\begin{Theorem}
\itemnote
  \(p=4\) の場合の考察
\itemprop
  \Then \(\eta ^3 = 0\)
\itemprop
  \For \(p \geq 3\) \\
  \Then \(0 \to KR^{-q}(X) \overset{\pi^*}{\to} KR^{-q}(X \times S^{p,0}) \to \overset{\delta}{\to} KR^{p+1-q}(X) \to 0\) is exact
\itemprop
  \Let \(\lambda\) : \(A_8\) := generator \\
  \Let \(\alpha\) : \(A_8 \to KR^{-8}(\text{pt})\) := Atiyah-Bott-Shapiro construction \\
  \Let \(1\) : \(KR^{-8}(S^{4,0})\) := identity \\
  \Then \(\mu_4^*(b^4 \cdot 1) = \alpha(\lambda) \cdot 1\)
\itemprop
  \Then multiplication by \(\alpha(\lambda)\) : \(KR(X) \to KR^{-8}(X)\) is isomorphism
\end{Theorem}

\section*{Relation with Clifford algebras}

\begin{Definition}
\itemdefi
  \Define quadratic form on \(\mathbb{R}^{p,q}\) := by \(- (\sum y_i^2 + \sum x_i^2)\) \\
  \Define involutory automorphism of \(\text{Cliff}(\mathbb{R}^{p,q})\) := induced by \((y,x) \mapsto (-y, x)\) \\
\itemdefi
  \Define real \(\mathbb{Z}_2\)-graded \(\text{Cliff}(\mathbb{R}^{p,q})\)-module :=
  \begin{itemize}
    \itembase \(M\) : complex \(\mathbb{Z}_2\)-graded \(\text{Cliff}(\mathbb{R}^{p,q})\)-module
    \itemenum real structure on \(M\)
    \itemwith \(\tau(M^i) \subset M^i\)
    \itemwith \(\tau(am) = \tau(a) \tau(m)\)
  \end{itemize}
\itemdefi
  \Define \(M(p,q)\) : Abelian group := Grothendieck group of real \(\mathbb{Z}_2\)-graded modules \\
  \Define \(A(p,q)\) : Abelian group := cokernel of restriction map \(M(p,q+1) \to M(p,q)\)
\end{Definition}

\begin{Definition}
\itemdefi
  \For \(M\) : real \(\mathbb{Z}_2\)-graded \(\text{Cliff}(\mathbb{R}^{p,q})\)-module \\
  \Let \(\sigma\) : \(S^{p,q} \times M^0 \to S^{p,q} \times M^1\) := \((s,m) \mapsto (s, sm)\) \\
  \Define \(h\) : \(M(p,q) \to KR^{p,q}(\text{point})\) := \(M \mapsto (M^0, M^1, \sigma)\) \\
  \Define \(\alpha\) : \(A(p,q) \to KR^{p,q}(\text{point})\) := induced by \(h\)
\end{Definition}

\begin{Definition}
\itemdefi
  \Let \(1\) : \(\Lambda^{0} \mathbb{C}\) , \(e\) : \(\Lambda^{1} \mathbb{C}\) := standard generator \\
  \Define \(\text{Cliff}(\mathbb{R}^{1,1})\)-module structure on \(\Lambda^* \mathbb{C}^1\) := induced by \(z \cdot 1 = ze , z \cdot e = - \bar{z} 1\) \\
  \Define \(\lambda_1\) : \(A(1,1)\) := defined by this module
\end{Definition}

\begin{Theorem}
\itemprop
  \Then \(\alpha(\lambda_1) = b\) \\
\end{Theorem}

\begin{Definition}
\itemdefi
  \Let \(M\) : \(\text{Cliff}(\mathbb{R}^{4,4})\) := \(\lambda_1 ^4\) \\
  \Let \(\omega\) : \(\text{Cliff}(\mathbb{R}^{4,4})\) := \(e_1 \cdot e_2 \cdot e_3 \cdot e_4\)
\itemprop
  \Then \(\omega^2 = 1\), \(\tau(\omega) = \omega\), \(\omega z = \tau(z) \omega\) for \(z\) : \(\mathbb{C}^4 = \mathbb{R}^{4,4}\)
\itemdefi
  \Define \(\tau\) : involution on \(M\) := \(m \mapsto -\omega \bar{m}\) \\
  \Define \(N\) : (new) real graded \(\text{Cliff}(\mathbb{R}^{0,8})\)-module structure := \(M\) with this involution
\end{Definition}

\begin{Theorem}
\itemprop
  \Then \(N\) is a irreducible \(C_8\)-module
\itemprop
  \Then \(N = \lambda\) \(A_8\)
\end{Theorem}

\begin{Theorem}
\itemprop
  \Then \(\mu_4^*(M) = N\)
\end{Theorem}

\begin{Definition}
\itemdefi
  \Define \(\xi\) : \(\mathbb{H} \to \text{Cliff}^{0}(\mathbb{R}^{4,0})\) := induced by \(1 , i , j , k \mapsto \frac{1 + \omega}{2}, \frac{1 + \omega}{2} e_1 e_2, \frac{1 + \omega}{2} e_1 e_3, \frac{1 + \omega}{2} e_1 e_4\) \\
  \Define \(\eta\) : \(S(\mathbb{H}) \to \text{Spin}(4)\) := \(s \mapsto \xi(s) + \frac{1}{2}(1 - \omega)\)
\itemdefi
  \Define \(\mu\) : \(S^{4,0} \times \mathbb{R}^8 \times N \to S^{4,0} \times \mathbb{C}^4 \times M\) := \((s,(x,y),n) \mapsto (s, x+isy, \eta(s)n)\)
\end{Definition}

\begin{Theorem}
\itemprop
  \Then \(\mu\) is real map \\
  \Then \(\mu\) is compatible with module structures
\end{Theorem}

\begin{Theorem}
\itemprop
  \For \(M\) : real \(C(\mathbb{R}^{p,q})\)-module \\
  \Define \([\place , \place] \place\) : action of \(\mathbb{R}^{p,q}\) on \(M\) := \([x,y]m = xm + iym\)
  \Let involution on \(M_{\text{real}}\) := by this action \\
  \Then \(M_{\text{real}}\) is real module in the usual sense for the \(\text{Cliff}(\mathbb{R}^{p+q}, \sum_{j=1}^p {y_j}^2 - \sum_{i=1}^{q} {x_i}^2)\)
\itemprop
  \Then \(M(p,q)\) and Grothendieck group of real graded \(C_{p,q}\)-modules.
\end{Theorem}

\section*{Relation with the index}

\begin{Definition}
\itemdefi
  \For \(\phi\) : function \\
  \Define \(\hat{\phi}\) : function := Fourier transform of \(\phi\)
\end{Definition}

\begin{Theorem}
\itemprop
  \Then \(\hat(\tau{\phi})(x) = \tau(\hat{\phi}(-x))\) \\
  \Then \(\sigma(\tau (P))(x, \xi) = \tau(\sigma (P)(x, -\xi))\) 
\end{Theorem}

\begin{Theorem}
\itemwhen
  \For \(X\) : smooth manifold with smooth involution, \(E,F\) : real smooth vector bundles over \(X\)
  \Define real structure on \(\Gamma(E)\) := \AIMAI{by natural definition} \\
  \For \(P\) : linear operator \(\Gamma(E) \to \Gamma(F)\) \\
  \Define \(\bar{P}\) : \(\Gamma(E) \to \Gamma(F)\) := \(\phi \mapsto \tau (P(\tau(\phi)))\)
\itemprop
  \Then \(\sigma (\bar{P})(x, \xi) = \tau (\sigma (P) (\bar{x}, - \tau^*(\xi)))\)
\itemdefi
  \Define \(P\) is real operator := \(P = \bar{P}\)
\end{Theorem}

\begin{Theorem}
\itemprop
  \IfHold \(X\) is trivial \\
  \Then \(P\) is a differential operator with real coefficients with respect to local bases of \(E,F\)
\itemprop
  \Let \(\pi\) : \(S(T^*X) \to X\) := projection \\
  \Let involution on \(S(T^*X)\) := \((x, \xi) \mapsto (\bar{x}, - \tau^*(\xi))\) \\
  \IfHold \(X\) is trivial \\
  \Then involution of \(S(T^*X)\) is the anti-podal map on each fiber
\itemprop
  \Then \((\pi^* E, \pi^* F, \sigma (P))\) represents an element \(KR(B(X), S(X))\)
\itemprop
  \Then index of \(P\) is in \(KR(\text{point})\)
\end{Theorem}

\end{document}