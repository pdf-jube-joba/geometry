\begin{Definition}
\itemnote
  クリフォード代数束を考察する
\itemdefi
  \Define \(cl(\rho_n)\) : \(\text{SO}(n) \to \text{Aut}(\text{Cl}(\mathbb{R}^n))\) := standard representation \\
  \Define \(cl(\rho_n)_*\) : \(\mathfrak{so}(n) \to \text{Der}(\text{Cl}(\mathbb{R}^n))\) := derivative of \(cl(\rho_n)\)
\end{Definition}

\begin{Theorem}
\itemprop
  forall (\(A\) : \(\mathfrak{so}(n)\)) (\(\phi_1 , \phi_2\) : \(\text{Cl}(\mathbb{R}^n)\)) \\
  \({cl(\rho_n)_* A}(\phi_1 \cdot \phi_2) = {cl(\rho_n)_* A}(\phi_1) \cdot \phi_2 + \phi_1 \cdot {cl(\rho_n)_* A}(\phi_2)\)
\end{Theorem}

\begin{Definition}
\itemnote
  クリフォード代数束の上に共変微分を定義する
\itemdefi
  \For (\((\pi : E \to B)\) : riemannian vector bundle)  (\(\tau\) : connection on \(P_{O}(E)\)) \\
  \Define \(\nabla\) : covariant derivative on \(\text{Cl}(E)\) :=
  \begin{indentblock}
    \Let \(\text{Cl}(E) \cong P_{O}(E) \times_{\rho} \text{Cl}(\mathbb{R}^n)\) := \AIMAI{natural definition} \\
    \Then naturaly induced
  \end{indentblock}
\end{Definition}

\begin{Theorem}
\itemprop
  \Let \(\nabla\) : covariant derivative on \(\text{Cl}(E)\) := defined above \\
  \Then forall (\(\phi_1 , \phi_2\) : \(\Gamma(\text{Cl}(E))\)) \(\nabla(\phi_1 \cdot \phi_2) = \nabla(\phi_1) \cdot \phi_2 + \phi_1 \cdot \nabla(\phi_2)\)
\itemprop
  \(\nabla\) preserves \(\text{Cl}^0(E)\) , \(\text{Cl}^1(E)\)
\itemprop
  \For (volume form (\(\omega\)) : \(\Gamma(\text{Cl}(E))\)) \\
  \(\nabla \omega = 0\)
\itemprop
  if \(n \equiv 3 , 4(mod 4)\) \\
  \(\nabla\) preserves \(\text{Cl}^{\pm}(E)\)
\end{Theorem}

\begin{Definition}
\itemwhen
  \Fix \((\pi : E \to M)\) : vector bundle with spin structure
\itemdefi
  \For (\(\tau\) : connection on \(P_{SO}(E)\)) (\(M\) : left \(\text{Cl}(\mathbb{R}^n)\) module) \\
  \Let \(\tilde{\tau}\) : connection on \(P_{Spin}(E)\) := indeced by covering \\
  \Let \(\nabla\) : covariant derivative on \(S(M)\) := induced by \(\tilde{\tau}\)\\
  \Then forall (\(\phi_1\) : \(\Gamma(\text{Cl}(E))\)) (\(\phi_2\) : \(\Gamma(S(M))\)) \(\nabla(\phi_1 \cdot \phi_2) = \nabla(\phi_1) \cdot \phi_2 + \phi_1 \cdot \nabla(\phi_2)\)
\end{Definition}

\begin{Theorem}
\itemnote
  上のようにして構成された共変微分の曲率について
\itemprop
  \For \(R\) : curvature of \(\text{Cl}(M)\) \\
  \Then \(R_{V , W} (\phi_1 \cdot \phi_2) = R_{V , W} (\phi_1) \cdot \phi_2 + \phi_1 \cdot R_{V , W}(\phi_2)\)
\itemprop
  \For \(R\) : curvature of \(\text{S}(M)\) \\
  \Then \(R_{V , W} (\phi_1 \cdot \phi_2) = R_{V , W} (\phi_1) \cdot \phi_2 + \phi_1 \cdot R_{V , W}(\phi_2)\)
\itemprop
  \For \(R\) : curvature of \(\text{S}(M)\) \\
  \Then \(R_{V , W}\) preserves \(S^{\pm}(M)\) if they are defined
\end{Theorem}

\begin{Theorem}
\itemnote
  spin 構造の接続について
\itemwhen
  \Fix \((\pi : E \to M)\) : vector bundle with spin structure
\itemprop
  \For (\(\omega\) : connection on \(P_{SO}(E)\)) \\
  \Then \(\xi^*\omega\) is connection on \(P_{Spin}(E)\)
\itemprop
  \For (\(\omega\) : connection on \(P_{SO}(E)\)) (\(\mathcal{E} = (e_1 , \ldots , e_n)\) : orthonormal frame on \(U\)) \\
  \For (\(\tilde{\mathcal{E}}\) : lift of \(\mathcal{E}\) on \(P_{Spin}(E \mid _{U})\)) \\
  \Let \(\tilde{\omega}\) : connection on \(P_{Spin}(E)\) := defined above \\
  \For \(\omega^s\) : connection on \(P_{SO}(S(M))\) \\
  \Let \(\tilde{\omega}^s\) : \(\Omega^1(U ; \mathfrak{spin}(n))\) := \(\tilde{\mathcal{E}}^*(\omega^s)\) \\
  \Then \(\tilde{\omega}^s = -\frac{1}{2}\sum_{i \lneq j} \tilde{\omega}_{ij} e_i e_j\)
\end{Theorem}

\begin{Theorem}
\itemnote
  local section の lift による符号倍の差を記述
\itemprop
  \WIP
\end{Theorem}

\begin{Theorem}
\itemprop
  \For (\(\omega\) : connection on \(P_{SO}(E)\)) (\(S\) : spinor bundle associated to \(E\) with \(M\)) \\
  \For (\(\mathcal{E} = (e_1 , \ldots , e_n)\) : local section of \(P_{SO}(E)\)) \\
  \Let \(\sigma_1 , \ldots , \sigma_N\) : local section of \(P_{SO}(S)\) := induced by \(\mathcal{E}\) \\
  \Then \(\nabla^s (\sigma_{\alpha} = \frac{1}{2}\sum_{i \lneq j} \tilde{\omega}_{ij} \otimes e_i e_j \cdot \sigma_{\alpha})\)
\itemprop
  \For (\(\sigma\) : \(\Gamma(S)\)) \\
  \Then \(R^s \sigma = \frac{1}{2}\sum_{i \lneq j} \tilde{\Omega}_{ij} \otimes e_i e_j \cdot \sigma\)
\itemprop
  \For (\(V , W\) : tangent vectors at \(x\) : \(M\)) \\
  \Then \(R_{V , W}^s(\sigma) = \frac{1}{2}\sum_{i \lneq j}\langle R_{V , W}(e_i) , e_i \rangle \)
\end{Theorem}

\begin{Definition}
\itemdefi
  \Define \(\mathfrak{R}_{V,W}^s\) : \(\Gamma(\text{Cl}(E)_x)\) :=
  \(\frac{1}{2}\sum_{i \lneq j} \langle R_{V,W}(e_i) , e_j \rangle e_i e_j\)
\itemprop
  \(\mathfrak{R}\) is \(2\) form on \(\text{Cl}(E)\)
\itemdefi
  \Define \(\mathfrak{R}_{V,W}^{cl}\) :=
  \(\frac{1}{2} \sum_{i \lneq j} \langle R_{V,W} e_i , e_j \rangle \text{ad}_{e_i e_j}\)
\itemprop
  \(R_{V,W}^{cl}(\phi) = \mathfrak{R}_{V,W}^{cl}(\phi)\)
\end{Definition}

\begin{Definition}
\itemnote
  torsion tensor を定義する
\itemdefi
  \For (\(\nabla\) ; connection) (\(V , W\) : \(\mathcal{X}(M)\)) \\
  \(T_{V,W}\) := \(\nabla_V W- \nabla_W V - [V,W]\)
\itemprop
  \((V , W) \mapsto T_{V,W}\) is tensor
\itemdefi
  \Define torsion tensor := \(T_{V,W}\)
\end{Definition}

\begin{Theorem}
\itemprop
  there exists unique connection on \(P_{SO}(TM)\) such that tensor vanish identically
\itemprop
  \Let \(R\) := cuvature tensor of canonical connection on \(P_{SO}(TM)\) defined above \\
  \Then \(R_{U,V}W + R_{V,W}U + R_{W,U}V = 0\) and \(\langle R_{U,V} W , Y \rangle = \langle R_{W,Y} U , V \rangle\)
\end{Theorem}