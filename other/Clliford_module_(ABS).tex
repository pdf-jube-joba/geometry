\documentclass[dvipdfmx]{jsarticle}
% パッケージ
\usepackage{amsthm}
\usepackage{amsmath,amssymb}
\usepackage{color}
\usepackage{tikz}

% 定理環境
%% 本体
\theoremstyle{definition}
\newtheorem*{tDefinition}{定義}
\newtheorem*{tTheorem}{定理}
\newtheorem*{tProof}{証明}
\newtheorem*{tNotation}{記法}
\newtheorem*{tRemark}{注意}
\newtheorem*{tWhen}{設定}

\newenvironment{Mini}{
  \begin{minipage}[t]{0.9\hsize}
  \setlength{\parindent}{12pt}
  \begin{itemize}
  \setlength{\labelsep}{10pt}
}{
  \end{itemize}
  \vspace{5pt}
  \end{minipage}
}

\newenvironment{Definition}{
  \begin{tDefinition}
  \begin{Mini}
}{
  \end{Mini}
  \end{tDefinition}
}

\newenvironment{Theorem}{
  \begin{tTheorem}
  \begin{Mini}
}{
  \end{Mini}
  \end{tTheorem}
}

\newenvironment{Proof}{
  \begin{tProof}
  \begin{Mini}
      }{
  \end{Mini}
  \end{tProof}
}

\newenvironment{When}{
  \begin{tWhen}
  \begin{Mini}
}{
  \end{Mini}
  \end{tWhen}
}

%% マーク
\newcommand{\itemwhen}{\item[\(\bigcirc\)]}
\newcommand{\itemnote}{\item[!]}

\newcommand{\itemdefi}{\item[\(\square\)]}
\newcommand{\itemprop}{\item[\(\vartriangleright\)]}
\newcommand{\itemand}{\item[\(-\)]}

\newcommand{\itemprof}{\item[\(\because\)]}
\newcommand{\itemthen}{\item[\(\rightsquigarrow\)]}

\newcommand{\itemenum}{\item[\(\bullet\)]}
\newcommand{\itemwith}{\item[\(-\)]}

% 記述関係
\newenvironment{indentblock}{
  \\
  \hspace*{5mm}
  \begin{minipage}{0.8\textwidth}
}{
  \end{minipage}
  \\
}

%コマンド関連

\newcommand{\WIP}{\textcolor{red}{工事中}}
\newcommand{\SORRY}{\textcolor{red}{わかりませんでした}}
\newcommand{\ADMIT}{\textcolor{blue}{認めます}}
\newcommand{\AIMAI}[1]{\textit{#1}}

\begin{document}

\section*{このノートについて}
Atiyah, Bott, Shapiro の Clifford modules のまとめ

\section*{Introduction}
この論文では Clifford algebras と spinors が \(KO\)-theory で果たす役割を調査する。

\begin{Definition}
\itemwhen
  \Fix \(k\) : commutative field \\
\itemdefi
  \For \(E\) : \(k\)-module , \(Q\) : quadratic form on \(E\) \\
  \Define tensor algebra \(\ldots\) \((T(E))\) : algebra := \(\sum_{i=0}^\infty \otimes_{k} E\) \\
  \Define \(I(Q)\) : two-sided ideal in \(E\) := generated by \(\{x \otimes x - Q(x) \mid x \in E\}\) \\
  \Define Clliford algebra of \(Q\) \(C(Q)\) : algebra := \(T(E) / I(Q)\) \\
  \Define canonical map \(i_Q\) : \(E \to C(Q)\) := by \(E \to T(E) \to C(Q)\) 
\end{Definition}

\begin{Theorem}
\itemprop
  \Then \(i_Q\) is injection
\itemprop
  \For \(A\) : \(k\)-algebra with unit , \(\phi\) : \(E \to A\) \\
  \IfHold \(\forall\) \(x:E\) , \(\phi(x)^2 = Q(x)\) \\
  \Then there exists unique \(\tilde{\phi}\) : \(C(Q) \to A\) s.t. \(\tilde{\phi} \circ i_Q = \phi\)
\itemprop
  \Then \(C(Q)\) is universal algebra with above property
\itemprop
  \Let \(F^{\substedplace{p}} T(E)\) : filter structure := for \(q\) : \(\mathbb{N}\) , \(\sum_{0 \leq i \leq q} \otimes_i E\) \\
  \Let filter structure on \(C(Q)\) := induced by \(F^{q} T(E)\) \\
  \Then associated graded algebra is isomorphic to exterior algebra
\itemprop
  \Then \(\text{dim}_k C(Q) = 2^{\text{dim} E}\)
\itemprop
  \For \(\{e_i\}_{i=1 \ldots n}\) : base for \(i_Q(E)\) \\
  \Then \(\{e_{i_1} \cdots e_{i_k} \mid i_1 \lneq i_2 \cdot\}\) is base for \(C(Q)\)
\itemdefi
  \Let \(C^0(Q)\) := image of \(\sum_{i} \otimes_{2i} E\) in \(C(Q)\) \\
  \Let \(C^1(Q)\) := image of \(\sum_{i} \otimes_{2i+1} E\) in \(C(Q)\) \\
  \Then \(\mathbb{Z}_2\) graded algebra structure on \(C(Q)\) := defined by \(C^0(Q) , C^1(Q)\)
\end{Theorem}

\begin{Theorem}
\itemprop
  \For \(E = E_1 \oplus E_2\) := orthogonal decomposition of \(E\) relative to \(Q\) \\
  \Let \(Q_i\) := restriction of \(Q\) to \(E_i\) \\
  \Then \(C(Q) \cong C(Q_1) \hat{\otimes} C(Q_2)\)
\end{Theorem}



\end{document}