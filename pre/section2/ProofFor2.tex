\documentclass[dvipdfmx]{jsarticle}
% パッケージ
\usepackage{amsthm}
\usepackage{amsmath,amssymb}
\usepackage{color}
\usepackage{tikz}

% 定理環境
%% 本体
\theoremstyle{definition}
\newtheorem*{tDefinition}{定義}
\newtheorem*{tTheorem}{定理}
\newtheorem*{tProof}{証明}
\newtheorem*{tNotation}{記法}
\newtheorem*{tRemark}{注意}
\newtheorem*{tWhen}{設定}

\newenvironment{Mini}{
  \begin{minipage}[t]{0.9\hsize}
  \setlength{\parindent}{12pt}
  \begin{itemize}
  \setlength{\labelsep}{10pt}
}{
  \end{itemize}
  \vspace{5pt}
  \end{minipage}
}

\newenvironment{Definition}{
  \begin{tDefinition}
  \begin{Mini}
}{
  \end{Mini}
  \end{tDefinition}
}

\newenvironment{Theorem}{
  \begin{tTheorem}
  \begin{Mini}
}{
  \end{Mini}
  \end{tTheorem}
}

\newenvironment{Proof}{
  \begin{tProof}
  \begin{Mini}
      }{
  \end{Mini}
  \end{tProof}
}

\newenvironment{When}{
  \begin{tWhen}
  \begin{Mini}
}{
  \end{Mini}
  \end{tWhen}
}

%% マーク
\newcommand{\itemwhen}{\item[\(\bigcirc\)]}
\newcommand{\itemnote}{\item[!]}

\newcommand{\itemdefi}{\item[\(\square\)]}
\newcommand{\itemprop}{\item[\(\vartriangleright\)]}
\newcommand{\itemand}{\item[\(-\)]}

\newcommand{\itemprof}{\item[\(\because\)]}
\newcommand{\itemthen}{\item[\(\rightsquigarrow\)]}

\newcommand{\itemenum}{\item[\(\bullet\)]}
\newcommand{\itemwith}{\item[\(-\)]}

% 記述関係
\newenvironment{indentblock}{
  \\
  \hspace*{5mm}
  \begin{minipage}{0.8\textwidth}
}{
  \end{minipage}
  \\
}

%コマンド関連

\newcommand{\WIP}{\textcolor{red}{工事中}}
\newcommand{\SORRY}{\textcolor{red}{わかりませんでした}}
\newcommand{\ADMIT}{\textcolor{blue}{認めます}}
\newcommand{\AIMAI}[1]{\textit{#1}}

\begin{document}

\begin{Proof}[support の性質]
\itemprof
  \(x\) : \(X\) において複体が exact だったとすると、 \(\text{dim} \, \text{Ker} \, \alpha_x = \text{dim} \, \text{Im} \, \alpha_x\) である。
  ここで元々の複体が \(\alpha^2 = 0\) を満たしていたことから、 \(\text{dim} \, \text{Ker} \, \alpha_y \geq \text{dim} \, \text{Im} \, \alpha_y\) が任意の \(y\) で成り立つ。
  \(\text{dim} \, \text{Im}\) は upper semi-continuous かつ \(\text{dim} \, \text{Ker}\) は lower semi-continuous なため、 \(x\) のある近傍 \(U\) が存在し \(y\) : \(U\) に対して \(\text{dim} \, \text{Ker} \, \alpha_x \geq \text{dim} \, \text{Ker} \, \alpha_y , \text{dim} \, \text{Im} \, \alpha_y \geq \text{dim} \, \text{Im} \, \alpha_x\) を満たす。
  従って \(\text{dim} \, \text{Ker} \, \alpha_y = \text{dim} \, \text{Im} \, \alpha_y\) より \(U\) 上で exact となるからよい。
\itemprof
  support が compact subset の部分閉集合となるから compact である。
\itemprof
  二つの複体の support は少なくとも support の和には含まれることから。
\end{Proof}

\begin{Proof}[\(\sim\) の性質]
\itemprof
  反射律と対称律は明らかに成り立つ。
  推移律はベクトル束の張り合わせ構成( \(X = X_1 \cup X_2\) に対して \(X_i\) 上のベクトル束 \(E_i\) が \(X_1 \cap X_2\) 上同型となるとき、 \(X\) 上のベクトル束で \(X_i\) への制限が \(E_i\) と同型となるものの構成)を homotopy に行えば確かに成り立つ。
\itemprof
  \(E_1 , E_2\) が homotopic のとき \(E_1 \oplus F , E_2 \oplus F\) が homotopic であることと \(F_1 , F_2\) が homotopic のとき \(E \oplus F_1 , E \oplus F_2\) が homotopic であることを示せばよい。
  前者の場合を示す(後者も同様にできる)。
  \(G\) が \(E_1,E_2\) の homotopy を与える complex のとき、 \(G \oplus (E \times [0,1])\) が \(E_1 \oplus F , E_2 \oplus F\) の homotopy を与えている。
\itemprof
  反射律と対称律は明らかに成り立つ。
  推移律は \(E \oplus E_0 \simeq F \oplus F_0 , F \oplus F_1 \simeq G \oplus G_1\) のとき \(E \oplus E_0 \oplus F_1 \simeq F \oplus F_0 \oplus F_1 \simeq G \oplus G_1 \oplus F_0\) より。
\itemprof
  \(E_0 \oplus C_E \simeq F_0 \oplus C_F , E_1 \oplus D_E \simeq F_1 \oplus D_F\) のとき \(E_0 \oplus E_1 \oplus C_E \oplus D_E \simeq F_0 \oplus F_1 \oplus C_F \oplus D_F\) より。
\itemprof
  上の命題からわかる。
\end{Proof}

\begin{Proof}[pushout の性質]
\itemprof
  示すべきことは \(E\) : cpt.supp. cpx over \((X,X-U)\) に対して \(i_*[\restr{E}{U}] = [E]\) である。
  ここでもし \(E \sim_{D}^{(X,X-U)} E^\prime\) であって \(E^\prime\) に対して成り立っていれば \(E\) に対しても成り立つことがわかるから、 \(E\) として \(0 \to \epsilon^n \overset{\alpha}{\to} E \to 0\) によって代表される元に対して成り立つことを示せばよい。
  この場合の \(i_* [\restr{\hat{E}}{U}]\) の構成は次のような順で行われる
  \begin{align*}
    0 \to \epsilon^n \overset{\alpha}{\to} E \to 0 & &/ X \\
    & \downarrow \text{restriction to} \, U \\
    0 \to \epsilon^n \to \restr{E}{U} \to 0 & &/ U \\
    & \downarrow \text{choose extension to} \, U^+ \\
    0 \to \epsilon^n \to \widetilde{\restr{E}{U}} \to 0 & &/ U^+ \\
    & \downarrow \text{pinch} \\
    0 \to \epsilon^n \to r^* \widetilde{\restr{E}{U}} \to 0 & &/ X^+ \\
    & \downarrow \text{restriction to} \, X \\
    0 \to \epsilon^n \overset{\beta}{\to} \restr{r^* \widetilde{\restr{E}{U}}}{X} =: E^\prime \to 0 & &/ X
  \end{align*}
  ただし、 \(\widetilde{\restr{E}{U}}\) は \(\restr{E}{U}\) の適当な拡張で \(r^* := X^+ \to U^+\) と書いた。
  最初と最後の複体が同じ類に属することを示せばよい。
  \(\xi\) の support が \(U\) に含まれることからこの操作の中で support は変わらず、また \(E^\prime\) は \(U\) 上 \(E\) と同型のままである。
  特に、\(\phi\) : \(\restr{E}{U} \to \restr{E^\prime}{U}\) で次を可換にするものがある。
  \[
  \begin{matrix}
    0 & \to & \epsilon^n & \overset{\restr{\alpha}{U}}{\to} & \restr{E}{U} & \to & 0 \\
     & & \downarrow \text{id} & & \downarrow \phi & & \\
     0 & \to & \epsilon^n & \overset{\restr{\beta}{U}}{\to} & \restr{E^\prime}{U} & \to & 0 \\
  \end{matrix}
  \]
  一方で \(\restr{\alpha}{X-U} , \restr{\beta}{X-U}\) は同型であるからこれらを組み合わせれば複体の同型が得られる。
\end{Proof}

\begin{Proof}[複体のテンソルと環構造]
\itemprof
  \(\alpha^{i+1} \circ \alpha^{i} = 0\) かどうかについて。 \\
  これは \(x \otimes y\) : \(E^i \otimes F^j\) に対して \(\alpha^{i+1} \alpha^i (x \otimes y) = \alpha_E^{i+1} \alpha_E^{i} x \otimes y + (-1)^{i+1} \alpha_E^{i} x \otimes \alpha_F^{j} y + (-1)^{i} \alpha_E^i x \otimes \alpha_F^j y + (-1)^i x \otimes \alpha_F^{j+1} \alpha_F^{j} y = 0\) よりわかる。
\itemprof
  \(\otimes\) が \(K(X)\) 上に well-defined な写像を定めることについて。 \\
  \(E\) か \(F\) のどちらかの support が compact のとき \(E \otimes F\) の support も compact であることを示せば十分。
  これには \(x\) : \(X\) で \(\restr{E}{x} , \restr{F}{x}\) のどちらかが exact なら \(\restr{E \otimes F}{x}\) も exact であることを示せばよい。
  \(x \otimes y\) : \(E^i \otimes F^j\) が \(\alpha^{i+j} (x \otimes y) = 0\) を満たすなら、 \(\alpha_E^i x \otimes y = 0\) かつ \(x \otimes \alpha_F^j y = 0\) である。
  \(x = 0\) か \(y = 0\) の場合を除くと \(\alpha_E^i x = 0\) かつ \(\alpha_F^j y = 0\) とわかる。
  \(\alpha_E x^\prime = x\) か \(\alpha_F y^\prime = y\) となるような元が取れる。
  前者の場合は \(x^\prime \otimes y\) 、後者の場合は \(\pm x \otimes y^\prime\) を考えればこれの \(\alpha\) による行き先が \(x \otimes y\) である。
\itemprof
  環構造を定めることについて。 \\
  和構造は \(\oplus\) とする。
  ゼロ元は \(0\) 次元自明束とすると成り立つ。
  単位元は底空間が compact のときには \(1\) 次元自明束により与えられる。
\end{Proof}

\begin{Proof}[(1) の式]
\itemprof
  \(\xi\) を代表する元として \(U^+\) 上の複体 \(E\) であって support が \(U\) に含まれるものにより \(\restr{E}{U}\) で表されるとする。
  また \(\zeta\) を代表する \(F\) なる \(X\) 上の複体を考える。
  次のように示せる。
  \begin{align*}
    & i_*(i^* \zeta \otimes \xi) \\
    &=i_* [i^* F \otimes (\restr{E}{U})] \\
    &=i_* [\restr{(F \otimes \restr{(p^* E)}{X})}{U}] \\
    &= [F \otimes \restr{(p^* E)}{X}] \\
    &= \zeta \otimes i_* \xi
  \end{align*}
  ただし \(p := X^+ \to U^+\) である。
  途中で \(i_*[\restr{E}{U}] = E\) が \(E\) : cpt.supp.cpx over \(X\) で \(\text{supp} \, E \subset U\) を満たすものに対して成り立つことを用いた。
\end{Proof}

\end{document}