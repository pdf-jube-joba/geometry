\documentclass[dvipdfmx]{jsarticle}

% パッケージ
\usepackage{amsthm}
\usepackage{amsmath,amssymb,mathrsfs}
\usepackage{color}
\usepackage{tikz}

% 定理環境
%% 本体
\theoremstyle{definition}
\newtheorem*{tDefinition}{定義}
\newtheorem*{tTheorem}{定理}
\newtheorem*{tProof}{証明}
\newtheorem*{tNotation}{記法}
\newtheorem*{tRemark}{注意}
\newtheorem*{tWhen}{設定}

\newenvironment{Mini}{
  \begin{minipage}[t]{1\hsize}
  \setlength{\parindent}{10pt}
  \begin{itemize}
  \setlength{\labelsep}{10pt}
}{
  \end{itemize}
  \vspace{5pt}
  \end{minipage}
}

\newenvironment{Definition}[1][\quad]{
  \begin{tDefinition}
  #1 \\
  \begin{Mini}
}{
  \end{Mini}
  \end{tDefinition}
}

\newenvironment{Theorem}[1][\quad]{
  \begin{tTheorem}
  #1 \\
  \begin{Mini}
}{
  \end{Mini}
  \end{tTheorem}
}

\newenvironment{Proof}[1][\quad]{
  \begin{tProof}
  #1 \\
  \begin{Mini}
      }{
  \end{Mini}
  \end{tProof}
}

\newenvironment{When}{
  \begin{tWhen}
  \quad \\
  \begin{Mini}
}{
  \end{Mini}
  \end{tWhen}
}

\newenvironment{Remark}{
  \begin{tRemark}
}{
  \end{tRemark}
}

%% マーク
\newcommand{\itemwhen}{\item[\(\bigcirc\)]}
\newcommand{\itemnote}{\item[!]}

\newcommand{\itemdefi}{\item[\(\square\)]}
\newcommand{\itemprop}{\item[\(\vartriangleright\)]}
\newcommand{\itemand}{\item[\(-\)]}

\newcommand{\itemprof}{\item[\(\because\)]}
\newcommand{\itemthen}{\item[\(\rightsquigarrow\)]}

\newcommand{\itemenum}{\item[\(+\)]}
\newcommand{\itembase}{\item[\(\bullet\)]}
\newcommand{\itemwith}{\item[\(-\)]}

% 記述関係
\newenvironment{indentblock}{
  \\
  \hspace*{5mm}
  \begin{minipage}{0.8\textwidth}
}{
  \end{minipage}
  \\
}

%コマンド関連
\newcommand{\place}{\_}
\newcommand{\restr}[2]{\left. {#1} \right| _{#2}}
\newcommand{\txt}{\texttt}

%% 宣言
\newcommand{\declare}[1]{\textcolor[rgb]{0.1, 0.8, 0.2}{#1 }}
\newcommand{\For}{\declare{For}}
\newcommand{\Define}{\declare{Define}}
\newcommand{\Let}{\declare{Let}}
\newcommand{\IfHold}{\declare{If}}
\newcommand{\Then}{\declare{Then}}
\newcommand{\Take}{\declare{Take}}
\newcommand{\Fix}{\declare{Fix}}
\newcommand{\Return}{\declare{Return}}

%% 置く
\newcommand{\WIP}{\textcolor{red}{工事中}}
\newcommand{\SORRY}{\textcolor{red}{わかりませんでした}}
\newcommand{\ADMIT}{\textcolor{blue}{認めます}}
\newcommand{\AIMAI}[1]{\textit{#1}}

\begin{document}
\(K(X)\) がアーベル群になること、 \(\tilde{K}(X^{+}) \cong K(X)\) が任意の locally compact space に対して成り立つこと、 \(K(X)\) が普通の定義と一致することを見る。

\section{\(K(X)\) がアーベル群になること}
\begin{Proof}
\itemprof
  逆元の存在のみが非自明なためそれを示す。
  長さが \(n\) の複体 \(\xi = 0 \to E_0 \overset{\alpha_1}{\to} E_1 \overset{\alpha_2}{\to} \cdots E_n \to 0\) に対して、 長さが \(n+1\) の複体 \(0 \to 0 \to E_0 \overset{\alpha_1}{\to} E_1 \overset{\alpha_2}{\to} \cdots E_n \to 0\) を直和した複体に同型な次の複体
  \[
    \zeta = 0 \to 0 \oplus E_0 \overset{\beta^0}{\to} E_0 \oplus E_1 \overset{\beta^1}{\to} E_1 \oplus E_2 \cdots E_{n-1} \oplus E_n \overset{\beta^n}{\to} E_n \oplus 0 \to 0
  \]
  は、ある acyclic な複体に homotopic である。
  ここで、 \(\beta^i\) は \(\alpha_0 = \alpha_{n+1} = 0\) とおいたとき
  \(\beta^i = \begin{pmatrix}
    \alpha_i & 0 \\
    0 & \alpha_{i+1} 
  \end{pmatrix}\) である。
  そのような homotopy として \(\zeta\) の boundary map を \(t\) : \([0,1]\) に対して
  \[
    \beta^i_t = \begin{pmatrix}
    (1-t) \alpha_i & (-1)^i t \\
    0 & (1-t)\alpha_{i+1}
    \end{pmatrix}
  \]
  として定義される parametrize された複体 \(\zeta_t\) を考えればよいことが次のようにしてわかる。
\itemprof
  複体になっていることを示す。
  \[\begin{pmatrix}
    (1-t) \alpha_{i+1} & (-1)^{i+1} t \\
    0 & (1-t)\alpha_{i+2}
  \end{pmatrix} \begin{pmatrix}
    (1-t) \alpha_{i} & (-1)^{i} t \\
    0 & (1-t)\alpha_{i+1}
  \end{pmatrix} = \begin{pmatrix}
    (1-t)^2 \alpha_{i+1} \alpha_{i} & 0 \\
    0 & (1-t)^2 \alpha_{i+2} \alpha_{i+1}
  \end{pmatrix} = 0\] よりよい。
\itemprof
  compact support であることを示す。
  \(t=0\) の場合 support が \(\zeta\) と一致し、計算により元々の複体 \(\xi\) と一致することがわかる。
  次に \(t>0\) の場合を考える。
  このとき、任意の点で exact であることが次のようにしてわかる。
  \(v = \begin{pmatrix}
    v_{i} \\ v_{i+1}
  \end{pmatrix}\) : \(E_{i} \oplus E_{i+1}\) に対して \(\beta^i v = 0\) と仮定する。
  これは \((1-t) \alpha_{i} v_i + (-1)^i t v_{i+1} = 0 , (1-t) \alpha_{i+1} v_{i+1} = 0\) と同値である。
  このとき \(v^\prime = \begin{pmatrix}
    0 \\ (-1)^{i+1} \frac{1}{t} v_{i}
  \end{pmatrix}\) : \(E_{i-1} \oplus E_{i}\) を考えると
  \begin{align*}
    \beta^{i-1} v^\prime
    &= \begin{pmatrix}
      (1-t) \alpha_{i-1} & (-1)^{i-1} t \\
      0 & (1-t)\alpha_{i}
    \end{pmatrix} \begin{pmatrix}
      0 \\ \frac{1}{t} v_{i}
    \end{pmatrix} \\
    &= \begin{pmatrix}
      (-1)^{i-1} v_i \\ (1-t) \frac{1}{t} \alpha_{i} v_i
    \end{pmatrix} = \begin{pmatrix}
      (-1)^{i} v_i \\ (-1)^{i+1} v_{i+1} 
    \end{pmatrix} = \pm \begin{pmatrix}
      v_i \\ v_{i+1}
    \end{pmatrix} = \pm v
  \end{align*}
  したがってこの homotopy を与えている複体 \(\zeta\) の support は \(\text{supp} \, \xi \times \{0\}\) であり compact support である。
\itemprof
  最後に、 \(\zeta_t\) が \(\zeta\) と acyclic な複体の間の homotopy を与えていることを示す。
  \(\zeta = \zeta_0\) であるから前半はよい。
  \(\zeta_1\) をみると \(\beta^i_1\) が直和成分への同型を与えているため \(\zeta_1\) は acyclic である。
\end{Proof}

\newpage

\section{補題:代表元のとり方について}
\begin{Theorem}
\itemprop
  \For \(\xi = (0 \to E_0 \overset{\alpha_1}{\to} E_1 \to 0)\) : compact support complex \\
  \Then there exists \(\xi^\prime = (0 \to \epsilon^n \to E \to 0)\) : compact support complex such that \([\xi] = [\xi^\prime]\)
\itemprop
  \For \(0 \to E_0 \overset{\alpha_1}{\to} E_1 \overset{\alpha_2}{\to} E_2 \overset{\alpha_3}{\to} E_3 \to \cdots \to E_n \to 0\) : compact support complex \\
  \For \(0 \to E_0 \overset{\alpha_1}{\to} E_1 \to E \to 0\) : acyclic complex \\
  \Then there exists \(0 \to E \overset{\alpha}{\to} E_2 \overset{\alpha_3}{\to} E_3 \to \cdots \to E_n \to 0\) : compact support complex
\itemprop
  \For \(\xi = (0 \to E_0 \overset{\alpha_1}{\to} E_1 \overset{\alpha_2}{\to} E_2 \to \cdots \to E_n \to 0)\) \\
  \Then there exists \(\xi^\prime = (0 \to E^\prime_0 \overset{\alpha^\prime_1}{\to} E^\prime_1 \overset{\alpha^\prime_2}{\to} E^\prime_2 \to \cdots \to E^\prime_n \to 0)\) such that \([\xi] = [\xi^\prime]\) and \(\alpha^\prime_1\) is injective
\itemprop
  \For \(x\) : \(K(X)\) \\
  \Then there exists \(\xi = (0 \to \epsilon^n \to E \to 0)\) : compact support complex such that \([\xi] = x\)
\end{Theorem}

\begin{Proof}
\itemprof
  \(\text{supp} \, \xi\) が compact より \(s\) : \(E_0 \to \epsilon^n\) であって \(\text{supp} \,\xi\) 上単射となるものをとれば、 \(E_0 \to E_1 \oplus \epsilon^n\) は全体で単射となるから、
  \(
    0 \to E_0 \overset{
        \begin{pmatrix}
        \alpha_1 \\
        s
        \end{pmatrix}
      }{\to} E_1 \oplus \epsilon^n \overset{
        \begin{pmatrix}
          \pi_1 & \pi_2
        \end{pmatrix}
      }{\to} E \to 0
  \)
  なる完全な複体を得る。
  ここから
  \begin{align*}
    & 0 \to E_0 \overset{\alpha_1}{\to} E_1 \to 0 \\
    &\equiv 0 \to E_0 \overset{\alpha_1}{\to} E_1 \to 0 \\
    &\oplus 0 \to \epsilon^n \overset{1}{\to} \epsilon^n \to 0 \\
    &\oplus 0 \to 0 \to E \overset{1}{\to} E \to 0 \\
    &\simeq 0 \to E_0 \oplus \epsilon^n \overset{
      \begin{pmatrix}
        \alpha_1 & 0\\
        0 & 1 \\
        0 & 0
      \end{pmatrix}
    }{\to} E_1 \oplus \epsilon^n \oplus E \overset{
      \begin{pmatrix}
        0 & 0 & 1
      \end{pmatrix}
    }{\to} E \to 0 \\
    &\simeq 0 \to E_0 \oplus \epsilon^n \overset{
      \begin{pmatrix}
      \alpha_1 & 0 \\
      s & 0 \\
      0 & -\pi_2
      \end{pmatrix}
    }{\to} E_1 \oplus \epsilon^n \oplus E \overset{
      \begin{pmatrix}
        \pi_1 & \pi_2 & 0
      \end{pmatrix}
    }{\to} E \to 0 \\
    &\simeq 0 \to \epsilon^n \overset{-\pi_2}{\to} E \to 0 \\
    &\oplus 
    0 \to E_0 \overset{
      \begin{pmatrix}
      \alpha_1 \\
      s
      \end{pmatrix}
    }{\to} E_1 \oplus \epsilon^n \overset{
      \begin{pmatrix}
        \pi_1 & \pi_2
      \end{pmatrix}
    }{\to} E \to 0 \\
    &\equiv 0 \to \epsilon^n \overset{-\pi_2}{\to} E \to 0
  \end{align*}
  最後の複体の support を考えると \(\alpha_1\) が同型となる点では \(\pi_2\) も同型であるから support が compact である。
\end{Proof}

\begin{Proof}
\itemprof
  これは商空間と同様にして定義できるためよい。
\end{Proof}

\begin{Proof}
\itemprof
  \(s\) : \(E_0 \to \epsilon^n\) であって \(\text{supp} \, \xi\) 上単射かつ support が compact となるものをとれば、以前の証明と同様に \(E_0 \to E_1 \oplus \epsilon^n\) が単射となる。
  \(\rho\) : \(X \to \mathbb{R}\) であって compact set をのぞいて \(\rho = 1\) かつ \(s\) の support 上 \(0\) となるものをとる。
  \begin{align*}
    &0 \to E_0 \overset{\alpha_1}{\to} E_1 \overset{\alpha_2}{\to} E_2 \overset{\alpha_3}{\to} E_3 \to \cdots \\
    &\simeq 0 \to E_0 \overset{\alpha_1}{\to} E_1 \overset{\alpha_2}{\to} E_2 \overset{\alpha_3}{\to} E_3 \to \cdots \\
    &\oplus 0 \to 0 \to \epsilon^n \overset{1}{\to} \epsilon^n \to 0 \to \cdots \\
    &=0 \to E_0 \overset{
      \begin{pmatrix}
        \alpha_1 \\
        0
      \end{pmatrix}
    }{\to} E_1 \oplus \epsilon^n \overset{
      \begin{pmatrix}
        \alpha_2 & 0 \\
        0 & 1
      \end{pmatrix}
    }{\to} E_2 \oplus \epsilon^n \overset{
      \begin{pmatrix}
        \alpha_3 , 0
      \end{pmatrix}
    }{\to} E_3 \to \cdots \\
    &\simeq 0 \to E_0 \overset{
      \begin{pmatrix}
        \alpha_1 \\
        s
      \end{pmatrix}
    }{\to} E_1 \oplus \epsilon^n \overset{
      \begin{pmatrix}
        \alpha_2 & 0 \\
        0 & \rho
      \end{pmatrix}
    }{\to} E_2 \oplus \epsilon^n \overset{
      \begin{pmatrix}
        \alpha_3 , 0
      \end{pmatrix}
    }{\to} E_3 \to \cdots
  \end{align*}
  最後の複体は \(\rho = 1\) を満たす点では exact であるから compact support である。
  これが求める複体の条件を満たすためよい。
\end{Proof}

\newpage
\section{\(\tilde{K}(X^+) \cong K(X)\)}
\(\tilde{K}(X^+) \to K(X)\) を次のように構成する。
\begin{Definition}
  \itemdefi \Define \(\tilde{K}(X^+) \to K(X)\) :=
  \begin{indentblock}
    \For \(x\) : \(\tilde{K}(X^+)\) \\
    \Take \(\xi\) : \(x\) such that \(\xi_{\infty}\) is acyclic \\
    \Return \([\restr{\xi}{X}]\)
  \end{indentblock}
\end{Definition}
この写像の well-definedness を示す。
\begin{Proof}
\itemprof
  値域に対して well-defined に定まっていることを示す。
  \(\restr{\xi}{X}\) が compact support であることを示せばよいが、 \(\restr{\xi}{\infty}\) が acyclic なので \(\xi\) の support が \(X\) に含まれることからよい。
\itemprof
そのような条件を満たす \(\xi\) の存在を示す。
まず、 \(x\) : \(\tilde{K}(X^+)\) に対してこれを代表する \(X^+\) 上の複体 \(\xi\) を任意にとったとき \([\restr{\xi}{\infty}] = 0\) すなわち \(\restr{\xi}{\infty}\) が acyclic なものと homotopic であるという条件を満たしていることが \(\tilde{K}(X^+)\) の元であることからわかる。
この homotopy を全体に拡張する。
設定を整理すると(後者 2 点は \(X\) が locally compact であることからとった)
\begin{itemize}
  \item \(X\) : locally compact space , \(\xi = (E,f)\) : cpx. over \(X^+\)
  \item \(\alpha_t\) : homotopy of \(\restr{\xi}{\infty}\) and acyclic complex
  \item \((U,\phi)\) : trivialization of \(\xi\) near \(\infty\)
  \item \(\rho\) : \(X \to \mathbb{R}\) such that \(\rho(\infty) = 1\) and \(\text{supp} \, \rho \subset U\)
\end{itemize}
のような状況で \(\tilde{\alpha}_t\) : for \(t \in [0,1]\) , boundary map in \(\xi\)'s base であって \(\restr{\tilde{\alpha}_t}{\infty}\) が \(\alpha_t\) と一致するものがとれればよい。

次のように写像を構成すればできる。
\begin{itemize}
  \item \(\tilde{\phi}_x\) : for \(x \in U\) , \(\restr{\xi}{x} \to \restr{\xi}{\infty}\) := \(\phi^{-1}(\infty , \pi_2 \phi [\cdot])\)
  \item \(\tilde{H}_t\) : for \(x \in U\) , \(\text{boundary map in} \, \restr{\xi}{\infty}\) := \((1-\rho)(x) \text{Id} + \rho(x) \alpha_t\)
  \item \(\tilde{F}_t\) : \(\text{boundary map in} \, \restr{E}{U}\) := \(\restr{E}{x}\) 上では \(x \in U\) のとき \(\tilde{\phi}_x^{-1} \tilde{H}_t \tilde{\phi}_x\) 、それ以外では \(E\) と同じ。
\end{itemize}
\(\tilde{F}_t\) は連続であり \(\tilde{F}_1\) は \(\infty\) へ制限すると acyclic である。
この homotopy の \(t=1\) への制限が求める \(\xi\) を与える。

\itemprof
次にそのような \(\xi\) のとり方によらないことを示す。
これは明らか。
\end{Proof}
以上で定義された。

これが単射であることを示す。
この写像は明らかに準同型であるため、核を計算する形でよい。

\begin{Theorem}
\itemprop
  \For \(H = (0 \to \epsilon^n \to E \to 0)\) : cpt.supp.cpx over \(X \times I\) \\
  \IfHold there exists \(U\) : neighborhood of \(\infty\) \(\subset X^+\) such that \(\restr{H}{(U \backslash \infty) \times I}\) is acyclic \\
  \Then \(H\) has extension to \(X^+ \times I\) whose support equals that of \(H\)
\itemprop
  \For \(\xi\) : cpt.supp.cpx over \(X^+\) \\
  \IfHold \(\restr{\xi}{\infty}\) is acyclic , \(\restr{\xi}{X}\) is homotopic to acyclic \\
  \Then \(\xi\) is acyclic
\end{Theorem}

\begin{Proof}
\itemprof
  これは張り合わせによる。
  \(G = 0 \to \epsilon^n \to \epsilon^n \to 0\) なる \(U \times I\) 上の acyclic な複体を考えると \(G\) と \(H\) は同型なのでこれを貼り合わせて求める拡張が得られる。
\itemprof
  前の命題を homotopy として適用できるように、 homotopy 自体をよい複体で代表してよいことと、 compact 集合 \(\text{supp} \, H \subset X \times I\) の \(X\) への射影が compact のため \(\infty\) と開集合で分離できることを考えると、確かに成り立つ。
\end{Proof}

これが全射であることを示す。
\(K(X)\) の任意の元は \(0 \to \epsilon^n \to E \to 0\) の形で代表されるから、これを \(X^+\) へ拡張することができればよいが、これは確かに拡張できるためよい。
以上より同型である。

\section{この定義がふつうの \(K\) 理論と整合的であること}
これは長さが \(2\) の複体によって代表されることからわかる。

\end{document}