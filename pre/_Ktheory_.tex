\documentclass[dvipdfmx]{jsarticle}

% パッケージ
\usepackage{amsthm}
\usepackage{amsmath,amssymb,mathrsfs}
\usepackage{color}
\usepackage{tikz}

% 定理環境
%% 本体
\theoremstyle{definition}
\newtheorem*{tDefinition}{定義}
\newtheorem*{tTheorem}{定理}
\newtheorem*{tProof}{証明}
\newtheorem*{tNotation}{記法}
\newtheorem*{tRemark}{注意}
\newtheorem*{tWhen}{設定}

\newenvironment{Mini}{
  \begin{minipage}[t]{1\hsize}
  \setlength{\parindent}{10pt}
  \begin{itemize}
  \setlength{\labelsep}{10pt}
}{
  \end{itemize}
  \vspace{5pt}
  \end{minipage}
}

\newenvironment{Definition}[1][\quad]{
  \begin{tDefinition}
  #1 \\
  \begin{Mini}
}{
  \end{Mini}
  \end{tDefinition}
}

\newenvironment{Theorem}[1][\quad]{
  \begin{tTheorem}
  #1 \\
  \begin{Mini}
}{
  \end{Mini}
  \end{tTheorem}
}

\newenvironment{Proof}[1][\quad]{
  \begin{tProof}
  #1 \\
  \begin{Mini}
      }{
  \end{Mini}
  \end{tProof}
}

\newenvironment{When}{
  \begin{tWhen}
  \quad \\
  \begin{Mini}
}{
  \end{Mini}
  \end{tWhen}
}

\newenvironment{Remark}{
  \begin{tRemark}
}{
  \end{tRemark}
}

%% マーク
\newcommand{\itemwhen}{\item[\(\bigcirc\)]}
\newcommand{\itemnote}{\item[!]}

\newcommand{\itemdefi}{\item[\(\square\)]}
\newcommand{\itemprop}{\item[\(\vartriangleright\)]}
\newcommand{\itemand}{\item[\(-\)]}

\newcommand{\itemprof}{\item[\(\because\)]}
\newcommand{\itemthen}{\item[\(\rightsquigarrow\)]}

\newcommand{\itemenum}{\item[\(+\)]}
\newcommand{\itembase}{\item[\(\bullet\)]}
\newcommand{\itemwith}{\item[\(-\)]}

% 記述関係
\newenvironment{indentblock}{
  \\
  \hspace*{5mm}
  \begin{minipage}{0.8\textwidth}
}{
  \end{minipage}
  \\
}

%コマンド関連
\newcommand{\place}{\_}
\newcommand{\restr}[2]{\left. {#1} \right| _{#2}}
\newcommand{\txt}{\texttt}

%% 宣言
\newcommand{\declare}[1]{\textcolor[rgb]{0.1, 0.8, 0.2}{#1 }}
\newcommand{\For}{\declare{For}}
\newcommand{\Define}{\declare{Define}}
\newcommand{\Let}{\declare{Let}}
\newcommand{\IfHold}{\declare{If}}
\newcommand{\Then}{\declare{Then}}
\newcommand{\Take}{\declare{Take}}
\newcommand{\Fix}{\declare{Fix}}
\newcommand{\Return}{\declare{Return}}

%% 置く
\newcommand{\WIP}{\textcolor{red}{工事中}}
\newcommand{\SORRY}{\textcolor{red}{わかりませんでした}}
\newcommand{\ADMIT}{\textcolor{blue}{認めます}}
\newcommand{\AIMAI}[1]{\textit{#1}}

\begin{document}
\(K(X)\) がアーベル群になること、 \(\tilde{K}(X^{+}) \cong K(X)\) が任意の locally compact space に対して成り立つこと、 \(K(X)\) が普通の定義と一致することを見る。

\section{ホモトピーに関する補題}
\(X\) 上の二つのベクトル束の複体 \(\xi_i = 0 \to E^i_0 \to \ldots \to E^i_n \to 0\) が homotopic であることは、 \(X \times I\) 上のベクトル束の複体 \(\xi\) であって \(r_t : X \to X \times I := x \mapsto (x , t)\) によって \(r_i^* \xi \cong \xi_i\) を満たすものが存在することとして定義されていた。
ここでは二つの複体が \(E^1_j = E^2_j =: E_j\) を満たすさいに、 もし \(f_t^i : t \in [0,1] , E_i \to E_{i+1}\) であって \(t\) に対して連続なものが複体になればそこから homotopy が構成できることを示す。

\begin{Definition}
\itemwhen
  \Fix \(X\) : locally compact Hausdorff space
\itemdefi
  \For \(E , F\) : vector bundle over \(X\) , \(f_t\) : \(t \in [0,1] . \text{bundle morphism} \, E \to F\) \\
  \Define \(f_t\) is continuous on \(t\) := if \(t \mapsto f_t\) : \(I \to \text{Map}(E,F)\) is continuous
\itemprop
  \Then this is equivalent to \((t , x) \mapsto f_t(x)\) : \(I \times E \to F\) is continuous
\end{Definition}

\begin{Theorem}
\itemwhen
  \Fix \(X\) : locally compact Hausdorff space
\itemprop
  \For \(E_i\) : vector bundle over \(X\) , \(f^i_t\) : continuous on \(t \in [0,1]\). \(E_i \to E_{i+1}\) \\
  \IfHold for each \(t\) : \([0,1]\) , \(0 \to E_0 \overset{f^1_t}{\to} E_1 \overset{f^2_t}{\to} \cdots \overset{f^n_t}{\to} E_n \to 0\) is complex \\
  \Let \(\tilde{f^i_t}\) : \(E^i \times I \to E^{i+1} \times I\) := \((x,t) \mapsto (f_t(x) , t)\) \\
  \Then \(\xi = (0 \to E_0 \times I \overset{\tilde{f^1_t}}{\to} \to E_1 \times I \overset{\tilde{f^2_t}}{\to} \cdots \overset{\tilde{f^n_t}}{\to} E_n \times I \to 0)\) is complex
\itemprop
  \IfHold \(\xi\) is compact support \\
  \Then \(r_0^* \xi , r_1^* \xi\) is compact support and homotopic
\end{Theorem}

\begin{Proof}
\itemprof
  二乗して消えていることが条件よりわかる。
\itemprof
  \(\text{supp} \, \xi \cap X \times \{i\}\) が \(r_i^* \xi\) の support になるから成り立つ。
\end{Proof}

\section{\(K(X)\) がアーベル群になること}
\begin{Proof}
\itemprof
  逆元の存在のみが非自明なためそれを示す。
  長さが \(n\) の複体 \(\xi = 0 \to E_0 \overset{\alpha_1}{\to} E_1 \overset{\alpha_2}{\to} \cdots E_n \to 0\) に対して、 長さが \(n+1\) の複体 \(0 \to 0 \to E_0 \overset{\alpha_1}{\to} E_1 \overset{\alpha_2}{\to} \cdots E_n \to 0\) を直和した複体に同型な次の複体
  \[
    \zeta = 0 \to 0 \oplus E_0 \overset{\beta^0}{\to} E_0 \oplus E_1 \overset{\beta^1}{\to} E_1 \oplus E_2 \cdots E_{n-1} \oplus E_n \overset{\beta^n}{\to} E_n \oplus 0 \to 0
  \]
  は、ある acyclic な複体に homotopic である。
  ここで、 \(\beta^i\) は \(\alpha_0 = \alpha_{n+1} = 0\) とおいたとき
  \(\beta^i = \begin{pmatrix}
    \alpha_i & 0 \\
    0 & \alpha_{i+1} 
  \end{pmatrix}\) である。
  そのような homotopy として \(\zeta\) の boundary map を \(t\) : \([0,1]\) に対して
  \[
    \beta^i_t = \begin{pmatrix}
    (1-t) \alpha_i & (-1)^i t \\
    0 & (1-t)\alpha_{i+1}
    \end{pmatrix}
  \]
  として定義される parametrize された複体 \(\zeta_t\) を考えればよいことが次のようにしてわかる。
\itemprof
  複体になっていることを示す。
  \[\begin{pmatrix}
    (1-t) \alpha_{i+1} & (-1)^{i+1} t \\
    0 & (1-t)\alpha_{i+2}
  \end{pmatrix} \begin{pmatrix}
    (1-t) \alpha_{i} & (-1)^{i} t \\
    0 & (1-t)\alpha_{i+1}
  \end{pmatrix} = \begin{pmatrix}
    (1-t)^2 \alpha_{i+1} \alpha_{i} & 0 \\
    0 & (1-t)^2 \alpha_{i+2} \alpha_{i+1}
  \end{pmatrix} = 0\] よりよい。
\itemprof
  compact support であることを示す。
  \(t=0\) の場合 support が \(\zeta\) と一致し、計算により元々の複体 \(\xi\) と一致することがわかる。
  次に \(t>0\) の場合を考える。
  このとき、任意の点で exact であることが次のようにしてわかる。
  \(v = \begin{pmatrix}
    v_{i} \\ v_{i+1}
  \end{pmatrix}\) : \(E_{i} \oplus E_{i+1}\) に対して \(\beta^i v = 0\) と仮定する。
  これは \((1-t) \alpha_{i} v_i + (-1)^i t v_{i+1} = 0 , (1-t) \alpha_{i+1} v_{i+1} = 0\) と同値である。
  このとき \(v^\prime = \begin{pmatrix}
    0 \\ (-1)^{i+1} \frac{1}{t} v_{i}
  \end{pmatrix}\) : \(E_{i-1} \oplus E_{i}\) を考えると
  \begin{align*}
    \beta^{i-1} v^\prime
    &= \begin{pmatrix}
      (1-t) \alpha_{i-1} & (-1)^{i-1} t \\
      0 & (1-t)\alpha_{i}
    \end{pmatrix} \begin{pmatrix}
      0 \\ \frac{1}{t} v_{i}
    \end{pmatrix} \\
    &= \begin{pmatrix}
      (-1)^{i-1} v_i \\ (1-t) \frac{1}{t} \alpha_{i} v_i
    \end{pmatrix} = \begin{pmatrix}
      (-1)^{i} v_i \\ (-1)^{i+1} v_{i+1} 
    \end{pmatrix} = \pm \begin{pmatrix}
      v_i \\ v_{i+1}
    \end{pmatrix} = \pm v
  \end{align*}
  したがってこの homotopy を与えている複体 \(\zeta\) の support は \(\text{supp} \, \xi \times \{0\}\) であり compact support である。
\itemprof
  最後に、 \(\zeta_t\) が \(\zeta\) と acyclic な複体の間の homotopy を与えていることを示す。
  \(\zeta = \zeta_0\) であるから前半はよい。
  \(\zeta_1\) をみると \(\beta^i_1\) が直和成分への同型を与えているため \(\zeta_1\) は acyclic である。
\end{Proof}

\newpage

\section{補題:代表元のとり方について}
\begin{Theorem}
\itemprop
  \For \(\xi = (0 \to E_0 \overset{\alpha_1}{\to} E_1 \to 0)\) : compact support complex \\
  \Then there exists \(\xi^\prime = (0 \to \epsilon^n \to E \to 0)\) : compact support complex such that \([\xi] = [\xi^\prime]\)
\end{Theorem}

\begin{Proof}
\itemprof
  \(\text{supp} \, \xi\) は compact なので \(s\) : \(E_0 \to \epsilon^n\) であって \(\text{supp} \,\xi\) 上単射となるものがとれる。
  \(\alpha (x) = (\alpha_1(x) , s(x))\) : \(E_0 \to E_1 \oplus \epsilon^n\) は全体で単射となるから、
  \(
    \xi^\prime = (0 \to E_0 \overset{\alpha}{\to} E_1 \oplus \epsilon^n \overset{\beta}{\to} E \to 0)
  \)
  なる完全な複体を得る。
  \(\beta(x,y) = \pi_1(x) + \pi_2(y)\) とおく。
  このとき \(\alpha_1\) が同型となるような \(x \in X\) において \(\pi_2\) が同型となることが次のようにしてわかる。
  完全性により次元を考えると \(\pi_2\) が単射であることを示せばよい。
  \(x\) : \(\ker \pi_2\) をとると \(\beta(0,x) = 0\) より \(y\) : \(E_0\) が存在して \((\alpha_1(y) , s(y)) = (0,x)\) とかけるが \(\alpha_1\) の同型により \(y = 0\) とわかり \(x = 0\) と示せた。
\itemthen
  ここから \(K(X)\) の元として \(\xi\) は \(0 \to \epsilon^n \overset{- \pi_2}{\to} E \to 0\) に同値であることを示す。
  \(\xi\) に acyclic な複体 \(0 \to \epsilon^n \overset{1}{\to} \epsilon^n \to 0\) と \(0 \to 0 \to E \overset{1}{\to} E \to 0\) を足せば次の複体となる。
  \[\xi_0 = 0 \to E_0 \oplus \epsilon^n \overset{
    \begin{pmatrix}
      \alpha_1 & 0\\
      0 & 1 \\
      0 & 0
    \end{pmatrix}
  }{\to} E_1 \oplus \epsilon^n \oplus E \overset{
    \begin{pmatrix}
      0 & 0 & 1
    \end{pmatrix}
  }{\to} E \to 0\]
  次の \(t\) : \([0,1]\) で parametrize された列 が compact support な複体であることを示せば、
  \[0 \to E_0 \oplus \epsilon^n \overset{
    \begin{pmatrix}
      \alpha_1 & 0\\
      t s & 1 - t \\
      0 & - t \pi_2
    \end{pmatrix}
  }{\to} E_1 \oplus \epsilon^n \oplus E \overset{
    \begin{pmatrix}
      t^2 \pi_1 & t \pi_2 & 1 - t
    \end{pmatrix}
  }{\to} E \to 0\]
  これを homotopy として \(\xi_0\) は \(0 \to \epsilon^n \overset{-\pi_2}{\to} E \to 0\) と \(\xi ^\prime\) の直和へ homotopic とわかる。
  \(\xi^\prime\) が完全であるから、 \(K(X)\) の元として \(\xi\) は \(0 \to \epsilon^n \overset{-\pi_2}{\to} E \to 0\) に同値であることがわかる。
  この複体の support を考えると \(\alpha_1\) が同型となる点では \(\pi_2\) も同型であるから support が compact である。
\itemthen
  複体となっていることを示す。
  すなわち、写像の合成が \(0\) であることを示す。
  \[
    \begin{pmatrix}
      t^2 \pi_1 & t \pi_2 & 1 - t
    \end{pmatrix} \cdot \begin{pmatrix}
      \alpha_1 & 0\\
      t s & 1 - t \\
      0 & - t \pi_2
    \end{pmatrix} =
    \begin{pmatrix}
      t^2 \pi_1 \alpha + t^2 \pi_2 s + 0 & 0 + (1-t) t \pi_2 - (1-t) t \pi_2
    \end{pmatrix} = (0,0,0)
  \]
\itemthen
  homotopy の support が compact であることを示す。
  これには \((\text{supp} \, \xi)^c \times I\) 上で完全であることを示せばよい。
  \(t=0\) では確かに完全であるからよいため、 \(t \not = 0\) とする。
  このとき \(\alpha_1 , \pi_2\) は同型である。
  \((x,y,z)\) : \(E_1 \oplus \epsilon^n \oplus E\) が \(t^2 \pi_1(x) + t \pi_2(y) + (1-t) z = 0\) を満たすとする。
  \((u,v)\) : \(E_0 \oplus \epsilon^n\) であって \((x,y,z)\) にうつるものを探せばよい。
  \(z = \pi_2(\tilde{z}) , x = \alpha_1(\tilde{x})\) をとる。
  このとき \((u,v) = (\tilde{x} , - \frac{1}{t} \tilde{z})\) は \((\alpha_1(\tilde{x}) , t s(\tilde{x}) - (1-t) \frac{1}{t} \tilde{z} , \pi_2(\tilde{z})) = (x , y^\prime := t s(\tilde{x}) - (1-t) \frac{1}{t} \tilde{z} , z)\) にうつる。
  \(\pi_2\) の同型性により、 \(t^2 \pi_1(x) + t \pi_2(y^\prime) + (1-t) z = 0\) なら \(y^\prime = y\) とわかり示される。
  実際、 \(t \pi_2(y^\prime) = t^2 \pi_2 s(\tilde{x}) - (1-t) \pi_2(\tilde{z}) = - t^2 \pi_1 \alpha_1 (\tilde{x}) - (1-t) z\) よりよい。
\end{Proof}


\begin{Theorem}
\itemprop
  \For \(0 \to E_0 \overset{\alpha_1}{\to} E_1 \overset{\alpha_2}{\to} E_2 \overset{\alpha_3}{\to} E_3 \to \cdots \to E_n \to 0\) : compact support complex \\
  \For \(0 \to E_0 \overset{\alpha_1}{\to} E_1 \to E \to 0\) : acyclic complex \\
  \Then there exists \(0 \to E \overset{\alpha}{\to} E_2 \overset{\alpha_3}{\to} E_3 \to \cdots \to E_n \to 0\) : compact support complex
\itemprop
  \For \(\xi = (0 \to E_0 \overset{\alpha_1}{\to} E_1 \overset{\alpha_2}{\to} E_2 \to \cdots \to E_n \to 0)\) , \(n \leq 3\) \\
  \Then there exists \(\xi^\prime = (0 \to E^\prime_0 \overset{\alpha^\prime_1}{\to} E^\prime_1 \overset{\alpha^\prime_2}{\to} E^\prime_2 \to \cdots \to E^\prime_n \to 0)\) such that \([\xi] = [\xi^\prime]\) and \(\alpha^\prime_1\) is injective
\itemprop
  \For \(x\) : \(K(X)\) \\
  \Then there exists \(\xi = (0 \to \epsilon^n \to E \to 0)\) : compact support complex such that \([\xi] = x\)
\end{Theorem}

\begin{Proof}
\itemprof
  これは商空間と同様にして定義できるためよい。
\end{Proof}

\begin{Proof}
\itemprof
  \(s\) : \(E_0 \to \epsilon^n\) であって \(\text{supp} \, \xi\) 上単射かつ support が compact となるものをとれば、以前の証明と同様に \(E_0 \to E_1 \oplus \epsilon^n\) が単射となる。
  \(\rho\) : \(X \to \mathbb{R}\) であって compact set をのぞいて \(\rho = 1\) かつ \(s\) の support 上 \(0\) となるものをとる。
  \(\xi\) は \(K(X)\) 上で acyclic な複体 \(0 \to 0 \to \epsilon^n \overset{1}{\to} \epsilon^n \to 0 \to 0 \cdots\) を直和したものに同型である。
  後者の複体は \(\rho\) の定義を考えると \(0 \to 0 \to \epsilon^n \overset{\rho}{\to} \epsilon^n \to 0 \to 0 \cdots\) なる複体へ \((1 - t) + t \rho\) により homotopic である。
  この homotopy は \(\rho\) がある compact set を除き \(1\) に等しいことから compact support である。
  従って \(\xi\) は次の複体に同値である。
  \[0 \to E_0 \overset{
    \begin{pmatrix}
      \alpha_1 \\
      0
    \end{pmatrix}
  }{\to} E_1 \oplus \epsilon^n \overset{
    \begin{pmatrix}
      \alpha_2 & 0 \\
      0 & \rho
    \end{pmatrix}
  }{\to} E_2 \oplus \epsilon^n \overset{
    \begin{pmatrix}
      \alpha_3 , 0
    \end{pmatrix}
  }{\to} E_3 \to \cdots\]
  であるが、これは
  \[0 \to E_0 \overset{
    \begin{pmatrix}
      \alpha_1 \\
      s
    \end{pmatrix}
  }{\to} E_1 \oplus \epsilon^n \overset{
    \begin{pmatrix}
      \alpha_2 & 0 \\
      0 & \rho
    \end{pmatrix}
  }{\to} E_2 \oplus \epsilon^n \overset{
    \begin{pmatrix}
      \alpha_3 , 0
    \end{pmatrix}
  }{\to} E_3 \to \cdots\]
  へ
  \[0 \to E_0 \overset{
    \begin{pmatrix}
      \alpha_1 \\
      t s
    \end{pmatrix}
  }{\to} E_1 \oplus \epsilon^n \overset{
    \begin{pmatrix}
      \alpha_2 & 0 \\
      0 & \rho
    \end{pmatrix}
  }{\to} E_2 \oplus \epsilon^n \overset{
    \begin{pmatrix}
      \alpha_3 , 0
    \end{pmatrix}
  }{\to} E_3 \to \cdots\]
  により homotopic である。
  support について考察すると、この homotopy は \(\rho = 1\) を満たす点では \(s = 0\) より完全であるからよい。
\end{Proof}

\begin{Proof}
\itemprof
  前三つの命題を一般の複体に順々に適用することで得られる。
\end{Proof}

\newpage
\section{\(\tilde{K}(X^+) \cong K(X)\)}
\(\tilde{K}(X^+) \to K(X)\) を次のように構成する。

\begin{Definition}
  \itemdefi \Define \(\tilde{K}(X^+) \to K(X)\) :=
  \begin{indentblock}
    \For \(x\) : \(\tilde{K}(X^+)\) \\
    \Take \(\xi\) : \(x\) such that \(\xi_{\infty}\) is acyclic \\
    \Return \([\restr{\xi}{X}]\)
  \end{indentblock}
\end{Definition}

この写像の well-definedness を示す。

\begin{Proof}[値域に対して well-defined に定まっていること]
\itemprof
  \(\restr{\xi}{X}\) が compact support であることを示せばよいが、 \(\restr{\xi}{\infty}\) が acyclic なので \(\xi\) の support が \(X\) に含まれることからよい。
\end{Proof}

\begin{Proof}[条件を満たす \(\xi\) の存在]
  \itemprof
  まず、 \(x\) : \(\tilde{K}(X^+)\) に対してこれを代表する \(X^+\) 上の複体 \(\xi\) を任意にとったとき \([\restr{\xi}{\infty}] = 0\) すなわち \(\restr{\xi}{\infty}\) に対して \(\xi^\prime\) : acyclic.cpx over \(\infty\) で \(\restr{\xi}{\infty} \oplus \xi^\prime\) が acyclic なものと homotopic であるという条件を満たしていることが \(\tilde{K}(X^+)\) の元であることからわかる。
  \(\xi ^\prime\) を pull back により全体へ acyclic な複体として拡張すると、設定を整理することにより、
  \begin{itemize}
    \item \(\xi = (0 \to E_0 \overset{\alpha_1}{\to} E_1 \to \cdots)\) : cpt.supp.cpx. over \(X^+\)
    \item \(\alpha_i^t\) : for \(t \in [0,1]\) , \(\restr{E_{i-1}}{\infty} \to \restr{E_{i}}{\infty}\)
  \end{itemize}
    のような状況で \(\tilde{\alpha_i}^t\) : for \(t \in [0,1]\) , \(E_{i-1} \to E_{i}\) であって、各 \(t \in [0,1]\) で \(\restr{\tilde{\alpha_i}^t}{\infty}\) が複体となり、 制限が \(\alpha_i^t\) や \(\alpha_i\) と一致するものがとれれば、 \(\tilde{\alpha_i}^1\) によって定義される複体は \(\xi\) と同値であり \(\infty\) への制限が acyclic となるものであるからよい。
    このようなものを構成する。
  \begin{itemize}
    \item \((U, \phi_i)\) : trivialization of \(\xi\) near \(\infty\)
    \item \(\rho\) : \(X \to \mathbb{R}\) such that \(\rho(\infty) = 1\) and \(\text{supp} \, \rho \subset U\)
  \end{itemize}
  を \(X\) : locally compact によりとる。
  次のように写像を構成すればできる。
  \begin{itemize}
    \item \(\tilde{\phi}_x\) : for \(x \in U\) , \(\restr{\xi}{x} \to \restr{\xi}{\infty}\) := \(\phi^{-1}(\infty , \pi_2 \phi [\cdot])\)
    \item \(\tilde{H}_t\) : for \(x \in U\) , \(\text{boundary map in} \, \restr{\xi}{\infty}\) := \((1-\rho)(x) \text{Id} + \rho(x) \alpha_t\)
    \item \(\tilde{F}_t\) : \(\text{boundary map in} \, \restr{E}{U}\) := \(\restr{E}{x}\) 上では \(x \in U\) のとき \(\tilde{\phi}_x^{-1} \tilde{H}_t \tilde{\phi}_x\) 、それ以外では \(E\) と同じ。
\end{itemize}
\(\tilde{F}_t\) は連続であり \(\tilde{F}_1\) は \(\infty\) へ制限すると acyclic である。
この homotopy の \(t=1\) への制限が求める \(\xi\) を与える。
\end{Proof}

\begin{Proof}[\(\xi\) のとり方によらないこと]
\itemprof
  示すべきことは「 \(\xi_0 , \xi_1\) : cpt.supp.cpx / \(X^+\) が \(\restr{\xi_0}{\infty} , \restr{\xi_1}{\infty}\) が acyclic でありかつ \(\xi_0^\prime , \xi_1^\prime\) : acyclic.cpx / \(X^+\) と \(H\) : cpt.supp.cpx / \(X^+ \times I\) で \(\xi_i \oplus \xi_i^\prime \cong \restr{H}{X^+ \times \{i\}}\) が存在するならば、 \(\tilde{\xi_0} , \tilde{\xi_1}\) : acyclic.cpx / \(X\) と \(\tilde{H}\) : cpt.supp.cpx / \(X \times I\) で \(\restr{\xi_i}{X} \oplus \tilde{\xi_i} \cong \tilde{H}\) が存在する」ことである。
  この homotopy をそのまま制限したものは cpt.supp とは限らないので homotopy 自体を取り替えるように命題を考える。
  これには「 \(H\) : cpt.supp.cpx / \(X^+ \times I\) で \(\restr{H}{\infty \times \{0\}} , \restr{H}{\infty \times \{1\}}\) が acyclic ならば、 \(H^\prime\) : cpt.supp.cpx / \(X^+ \times I\) で \(\restr{H}{X^+ \times \{0\}} , \restr{H}{X^+ \times \{1\}}\) が acyclic なものと \(\tilde{H}\) : cpt.supp.cpx / \(X^+ \times I\) で \(\restr{H}{\infty \times I}\) が acyclic なものが存在して、 \(H \oplus H^\prime\) と \(\tilde{H}\) の"端点を固定した"homotopy が存在する」ことを示せばよい。
  次の二つの補題に分けた。
\end{Proof}

\begin{Theorem}
  \itemprop
  \For \(H\) : cpt.supp.cpx over \(\infty \times I\) such that \(\restr{H}{\infty \times 0} , \restr{H}{\infty \times 1}\) is acyclic \\
  \Then there exists \(H^\prime\) : cpt.supp.cpx over \(\infty \times I\) such that \(\restr{H^\prime}{\infty \times 0} , \restr{H^\prime}{\infty \times 1}\) is acyclic , \(H ^\prime\) : cpt.supp.cpx over \(\infty \times I \times I\) such that \(\restr{\tilde{H}}{\infty \times I \times 0} \cong H \oplus H^\prime\) and \(\restr{\tilde{H}}{}\)
\itemprop
  \For \(H\) : cpt.supp.cpx over \(\infty \times I \times I\) , \(H^\prime\) : cpt.supp.cpx over \(X^+ \times I\) \\
  \Then there exists \(\tilde{H}\) : cpt.supp.cpx over \(X^+ \times I \times I\) such that \(\restr{\tilde{H}}{X^+ \times I \times 0} \cong H , \restr{\tilde{H}}{\infty \times I \times I} \cong H^\prime\)
\end{Theorem}

以上で定義された。

これが単射であることを示す。
この写像は明らかに準同型であるため、核を計算する形でよい。
ゆえに次の命題を示せばよいとわかる。

\begin{Theorem}
\itemprop
  \For \(H = (0 \to \epsilon^n \to E \to 0)\) : cpt.supp.cpx over \(X \times I\) \\
  \IfHold there exists \(U\) : neighborhood of \(\infty\) \(\subset X^+\) such that \(\restr{H}{(U \backslash \infty) \times I}\) is acyclic \\
  \Then \(H\) has extension to \(X^+ \times I\) whose support equals that of \(H\)
\itemprop
  \For \(\xi = (0 \to \epsilon^n \to E \to 0)\) : cpt.supp.cpx over \(X^+\) \\
  \IfHold \(\restr{\xi}{\infty}\) is acyclic , \(\restr{\xi}{X}\) is homotopic to acyclic \\
  \Then \(\xi\) is acyclic
\end{Theorem}

\begin{Proof}
\itemprof
  これは張り合わせによる。
  \(G = 0 \to \epsilon^n \to \epsilon^n \to 0\) なる \(U \times I\) 上の acyclic な複体を考えると \(G\) と \(H\) は \(U \backslash \infty \times I\) 上同型なのでこれを貼り合わせて求める拡張が得られる。
\itemprof
  前の命題を homotopy として適用できるように、 homotopy 自体をよい複体で代表してよいことと、 compact 集合 \(\text{supp} \, H \subset X \times I\) の \(X\) への射影が compact のため \(\infty\) と開集合で分離できることを考えると、確かに成り立つ。
\end{Proof}

これが全射であることを示す。
\(K(X)\) の任意の元は \(0 \to \epsilon^n \to E \to 0\) の形で代表されるから、これを \(X^+\) へ拡張することができればよいが、これは確かに拡張できるためよい。
以上より同型である。

\section{この定義がふつうの \(K\) 理論と整合的であること}
これは長さが \(2\) の複体によって代表されることからわかる。

\end{document}