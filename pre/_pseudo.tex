\begin{Theorem}
\itemwhen
  \Fix \(p\) : symbol of order \(m\)
\itemprop
  \For \(u\) : \(\mathscr{S}\) \\
  \Then \(x \mapsto (Pu) (x) = (2\pi)^{-\frac{n}{2}} \int e^{i \langle x , \xi \rangle} p(x , \xi) \hat{u}(\xi) d \xi\) defines a element of \(\mathscr{S}\)
\itemprop
  \IfHold \(p\) has compact \(x\)-support (independent from \(\xi\)) \\
  \Then forall \(s\) , this operator has a continuous extension \(L^2_{s+m} \to L^2_s\)
\end{Theorem}

\begin{Proof}
\itemprof
  各 \(x\) ごとに \(\mathbb{C}^p\) の元が定まっているかを示す。
  「 \(\int \lvert f \rvert < \infty\) なら \(\int f < \infty\) 」より \(\int \lvert p(x ,\xi) \rvert \lvert \hat{u}(\xi) \rvert d \xi < \infty\) を示せばよい。
  \(\lvert p(x,\xi) \rvert \leq \text{const} (1 + \lvert \xi \rvert)^m\) なる \(C\) をとる。
  \(u \in \mathscr{S}\) なら \(\hat{u} \in \mathscr{S}\) であるから各整数 \(k\) に対して \(\lvert \hat{u}(\xi) \rvert \leq \text{const} (1 + \lvert \xi \rvert)^{-k}\) をとる。
  このとき、各整数に対して
  \[
    \cdots \leq \int \text{const} (1 + \lvert \xi \rvert)^{m} \text{const} (1 + \lvert \xi \rvert)^{-k} d \xi \leq \text{const} \int (1 + \lvert \xi \rvert)^{m - k} d\xi
  \]
  である。
  \(k\) をうまくとれば右辺は有限なのでよい。
\itemthen
  \(x \mapsto (2\pi)^{-\frac{n}{2}} \int e^{i \langle x , \xi \rangle} p(x , \xi) \hat{u}(\xi) d \xi\) が \(\mathbb{R}^n \to \mathbb{C}^p\) としてなめらかかどうかについて示す。
  \(f(x,\xi)\) := \(e^{i\langle x , \xi \rangle} p(x,\xi) \hat{u}(\xi)\) とするとこれは滑らかであり、 \(x \mapsto \int f(x,\xi) d\xi\) が滑らかかどうかを示すということである。
  各 \(x_0\) : \(\mathbb{R}^n\) と \(\alpha\) : multi index に対して \(\lim_{h \to \infty} \int \lvert (D^\alpha_x f)(x + h ,\xi) \rvert = \int \lvert (D^\alpha_x f)(x , \xi) \rvert \) を示せばよいことが次のようにしてわかる。
  もしこれが成り立てば \(\lim_{y \to x_0} \int (D^\alpha_x f)(y , \xi) d\xi = \int (D^\alpha_x f)(x_0 , \xi) d\xi\) である。
  一階微分が可能であることについては任意の \(h\) : \(\mathbb{R}\) に対して平均値の定理より \(0 \leq \theta (h) \leq h\) が存在して、 \(h_i\) := \((0 , \ldots , h , \ldots , 0)\) とすると
  \[\frac{1}{h}(\int f(x + h_i ,\xi) d\xi - \int f(x , \xi) d\xi) = \int \frac{f(x + h_i , \xi) - f(x , \xi)}{h} d\xi = \int \frac{\partial f}{\partial x_i} (x + \theta (h) , \xi) d\xi\]
  であるから、 \(h \to \infty\) のとき、
  \(\frac{\partial}{\partial x_i} \int f(x,\xi) d\xi = \lim_{h \to 0} \frac{1}{h}(\int f(x + h_i ,\xi) d\xi - \int f(x , \xi) d\xi) = \lim_{h \to 0} \int \frac{\partial f}{\partial x_i}(x + \theta(h) , \xi) d\xi = \int \frac{\partial f}{\partial x_i}(x , \xi) d\xi\) なのでよい。
  これを繰り返せば何回も微分できる。
\itemthen
  \(x_0\) と \(\alpha\) を固定する。
  \(h\) : \(\mathbb{R}\) に対して \(f_h(\xi) := (D^\alpha_x f)(x_0 + h , \xi)\) とすると、ルベーグの優収束定理により可積分な \(g(\xi)\) で \(\lvert f_h(\xi) \rvert\) を支配する関数があればよいとわかる。
  各整数 \(k\) に対して \(\xi , h\) の式として
  \begin{align*}
    \lvert f_h(\xi) \rvert
    &= D^\alpha_x [e^{i \langle x , \xi \rangle} p](x_0 + h , \xi) \hat{u}(\xi) \\
    &\leq \sum_{\alpha_1 + \alpha_2 = \alpha} \lvert \xi^{\alpha_1} \rvert \lvert (D^{\alpha_2}_x p)(x_0 + h , \xi) \rvert \lvert \hat{u}(\xi) \rvert
    \leq \sum_{\alpha_1 + \alpha_2 = \alpha} \text{const} \, (1 + \lvert \xi \rvert)^{\lvert \alpha_1 \rvert + m - k} \\
    &\leq \text{const} (1 + \lvert \xi \rvert)^{\lvert \alpha \rvert + m - k}
  \end{align*}
  右辺は \(h\) によらない関数であり \(k\) を十分に大きくとれば可積分なのでよい。
\itemthen
  \(u\) : \(\mathscr{S}\) に対して \(P u \in \mathscr{S}\) を示す。
  これにはどの pseudodifferential operator \(P\) についても各整数 \(k\) に対して \(x\) の式として \(\lvert (P u)(x) \rvert \leq \text{const} (1 + \lvert x \rvert)^{-k}\) が有界となること、 \(D^\alpha \circ P\) も pseudodifferential operator であることを示せばよい。
  一つ目は各整数 \(N > 0\) に対して
  \begin{align*}
    \lvert x \rvert^{2N} Pu(x)
    &= (-1)^N \int (\Delta^N_{\xi} e^{i \langle x , \xi \rangle}) p(x , \xi) \hat{u}(\xi) d\xi \\
    &= (-1)^N \int e^{i \langle x , \xi \rangle} (\Delta^N_{\xi} [p(x,\xi) \hat{u}(\xi)]) d\xi
  \end{align*}
  であるが最後の式は有界であることから、各整数 \(k\) に対して \(x\) の式として \(\lvert (P u) (x) \rvert \leq \text{const} (1 + \lvert x \rvert)^{-k}\) となるようにとることができる。
  二つ目は \(D^\alpha \circ P\) に対応する symbol の存在を示す。
  これは計算すれば \(\sum_{\beta + \beta^{\prime} = \alpha} \xi^\beta p(x,\xi) + D^{\beta^\prime}_x p(x,\xi)\) が symbol になることからしたがうがこれは成り立つ。
\end{Proof}

\begin{Proof}
\itemprof
  補題としていくつかの式を確認する。
  \(\zeta , \xi\) : \(\mathbb{R}^n\) に対して \(\zeta^\alpha \int e^{i \langle x , \zeta \rangle} p(x , \xi) d x = (-1)^{\lvert \alpha \rvert} \int e^{i \langle x , \zeta \rangle} D^\alpha_x p(x , \xi) dx\) であることが \(x\)-compact よりわかる。
  したがって各 \(\alpha\) に対して \(\zeta,\xi\) の式として
  \begin{align*}
    \lvert \int e^{i \langle x , \zeta \rangle} p(x , \xi) d x \rvert
    &\leq \lvert \zeta^\alpha \rvert^{-1} \int  \lvert D^\alpha_x p(x , \xi) \rvert dx = \lvert \zeta^\alpha \rvert^{-1} \int_{x\text{-support of} p} \lvert D^\alpha_x p(x , \xi) \rvert dx \\
    &\leq \text{const} \, (1 + \lvert \zeta \rvert)^{- \lvert \alpha \rvert}  \int_{x\text{-support of} p} (1 + \lvert \xi \rvert)^m dx \\
    &\leq \text{const} (1 + \lvert \xi \rvert)^m (1 + \lvert \zeta \rvert)^{- \lvert \alpha \rvert} \\
  \end{align*}
  ゆえに各正の整数 \(t\) に対して \(\zeta , \xi\) の式として \(\cdots \leq \text{const} (1 + \lvert \xi \rvert)^m (1 + \lvert \zeta \rvert)^{-t}\) とわかる。
  \(\Psi(\xi,\eta)\) := \(\lvert \int e^{i \langle x , \xi - \eta \rangle} p(x , \xi) dx \rvert (1 + \lvert \xi \rvert)^{-s-m} (1 + \lvert \eta \rvert)^s\) とする。
  \(\zeta , \eta , s\) の式として \((1 + \lvert \xi \rvert)^{-s} (1 + \lvert \eta \rvert)^s \leq \text{const} (1 + \lvert \xi - \eta \rvert)^{\lvert s \rvert}\) をとる。
  これがとれるのは \(s \geq 0\) に対して \(\text{max}_{x,y : [0 , \infty)} ((1 + x) / (1 + y) (1 + \lvert x - y \rvert))^s \leq 1\) となることからわかる。
  各実数 \(s\) と整数 \(t\) に対して \(\xi , \zeta\) の式として
  \begin{align*}
    \Psi(\xi,\zeta)
    &\leq \text{const} (1 + \lvert \xi \rvert)^{-s} (1 + \lvert \zeta \rvert)^{s} (1 + \lvert \xi - \zeta \rvert)^{-t} \\
    &\leq \text{const} (1 + \lvert \xi - \zeta \rvert)^{-t + \lvert s \rvert}
  \end{align*}
  したがって \(\int \Psi(\xi,\zeta) d\xi \leq C\) かつ \(\int \Psi(\xi , \zeta) d\zeta \leq C\) なる定数 \(C\) がとれる。
\itemthen
  \(u\) : \(L^2_{s + m}\) に対して \(P u \in L^2_s\) を示せばよいのだった。
  ここで \((u,v) = \int \hat{u}(\xi) \cdot \hat{v}(\xi) d\xi\) は \(L^2_s \times L^2_{-s}\) の perfect な pairing になるのであった。
  まず \(u , v\) : \(\mathscr{S}\) に対して定数倍を無視して変形すると
  \begin{align*}
    (Pu , v)
    &= \int \widehat{Pu}(\eta) \cdot \hat{v}(\eta) d\eta \\
    &= \int \int e^{-i \langle x , \eta \rangle} (Pu)(x) dx \hat{v}(\eta) d\eta \\
    &= \int \int \int e^{-i \langle x , \eta \rangle} \int e^{i \langle x , \xi \rangle} p(x,\xi)  \hat{u}(\xi) d\xi \hat{v}(\eta) dx d\eta \\
    &= \int \int \int e^{i \langle x ,\xi - \eta \rangle} p(x,\xi) dx \hat{u}(\xi) \hat{v}(\eta) d\xi d\eta
  \end{align*}
  これにより、
  \begin{align*}
    \lvert (P u , v) \rvert
    &\leq \int \int \Psi (\xi , \eta) \{\hat{u}(\xi) (1 + \lvert \xi \rvert)^{s+m}\} \{\hat{v}(\eta)(1 + \lvert \eta \rvert)^{-s}\} d\xi d\eta\\
    &\leq \left\{\int \int \Psi(\xi , \eta) (1 + \lvert \xi \rvert)^{2(s+m)} \lvert \hat{u}(\xi) \rvert ^2 d\xi d\eta \right\}^{\frac{1}{2}} \left\{\int \int \Psi(\xi , \eta) (1 + \lvert \xi \rvert)^{-2s} \lvert \hat{v}(\eta) \rvert ^2 d\eta d\xi \right\}^{\frac{1}{2}} \\
    &\leq C \lVert u \rVert_{s+m} \lVert v \rVert_{-s}
  \end{align*}
  これに pairing の完全性(双対性)を考えれば確かに成り立つ。
\end{Proof}

\begin{Definition}
\itemdefi
  \Define pseudodifferential operator of order \(m\) on \(\mathbb{R}^n\) := \\
  operator \(\mathscr{S} \to \mathscr{S}\) given by symbol of order \(m\) \\
  \Define \(\Psi DO_m\) := space of all such operator
\itemdefi
  \For \(m > 0\) , \\
  \Define smoothing of order \(m\) := pseudodifferential operator of order \(- m\) \\
  \Define infinitely smoothing operator := \\
  linear map \(\mathscr{S} \to \mathscr{S}\) which extends to a bounded linear map \(L_s \to L_{s+m}\) forall \(s,m\) : \(\mathbb{R}\)
\itemdefi
  \For \(P , P^{\prime}\) : pseudodifferential operator \\
  \Define \(P\) and \(P^{\prime}\) is equivalent := \(P-P^{\prime}\) is infinitely smoothing operator
\end{Definition}

\begin{Theorem}
\itemprop
  \For \(P\) : pseudodifferential operator with compact support \\
  \IfHold forall \(m > 0\) , \(P\) is smoothing of order \(m\) \\
  \Then \(P\) is pseudodifferential operator of any order and is infinitely smoothing operator
\itemprop
  \For \(P\) : infinitely smoothing operator \\
  \Then \(P(L^2_s) \subset C^\infty\) forall \(s\) : \(\mathbb{R}\)
\end{Theorem}

\begin{Proof}
\itemprof
  条件より任意の \(m\) : \(\mathbb{R}\) に対して pseudodifferential operator of order \(m\) であるとわかる。
  compact support であることと合わせて任意の \(s,m\) : \(\mathbb{R}\) に対して \(L^2_{s+m} \to L^2_{s}\) に拡張されるから、任意の \(s,m\) : \(\mathbb{R}\) に対して  \(L^2_{s} \to L^2_{s+m}\) に拡張される。
\itemprof
  Sobolev embedding theorem により任意の実数 \(s\) と整数 \(k\) に対して \(s > \frac{n}{2} + k\) が満たされていれば \(L^2_s \subset C^k\) であった。
  定義より \(u\) : \(L^2_{s}\) は任意の \(m\) : \(\mathbb{R}\) に対して \(P u \in L^2_{s + m}\) である。
  したがって任意にとった \(k\) に対して \(m\) を \(s + m > \frac{2}{n} + k\) を満たす程度に十分大きくとることで \(P u \in L^2_{s + m} \subset C^k\) とわかる。
\end{Proof}

\begin{Theorem}
\itemwhen
  \For \(m\) : \(\mathbb{R}\) \\
  \For \(a(x,y,\xi)\) : \(C^{\infty}(\mathbb{R}^n \times \mathbb{R}^n \times \mathbb{R}^n , M(l,\mathbb{C}))\) with compact \(x\)- and \(y\)-support \\
  \IfHold forall \(\alpha , \beta , \gamma\) , there exists \(C_{\alpha , \beta , \gamma}\) such that \(\lvert D^\alpha_x D^\beta_y D^\gamma_\xi a \rvert \leq C_{\alpha , \beta , \gamma} (1 + \lvert \xi \rvert)^{m - \lvert \gamma \rvert}\)
\itemprop
  \Then forall \(\alpha\) , function \((D^\alpha_\xi D^\alpha_y a)(x , x , \xi)\) of \(x,\xi\) is symbol of order \(m - \lvert \alpha \rvert\)
\itemprop
  \Then for \(u\) : \(\mathscr{S}\) , \(\left[x \mapsto (2\pi)^{-n} \int \int e^{i \langle x-y , \xi \rangle} a(x,y,\xi) u(y) dy d\xi \right]\) is in \(\mathscr{S}\)
\itemprop
  \Let \(K\) : \(\mathscr{S} \to \mathscr{S}\) := defined by \((Ku)(x) = (2\pi)^{-n} \int \int e^{i \langle x-y , \xi \rangle} a(x,y,\xi) u(y) dy d\xi\) \\
  \Then \(K\) is a pseudodifferential operator whose symbol \(k\) is order \(m\) \\
  \Then and has formal development \(k(x,\xi) \sim \sum_{\alpha} \frac{i ^ {\lvert \alpha \rvert}}{\alpha !} (D^\alpha_\xi D^\alpha_y a)(x , x , \xi)\)
\itemprop
  \IfHold \(a(x,y,\xi)\) vanishes for all \(x,y\) in a neightborhood of the diagonal \\
  \Then corresponding operator is infinitely smoothing
\end{Theorem}

\begin{Proof}
\itemprof
  実際に計算する。
  各 \(\alpha , \beta , \gamma\) について \(x,\xi\) の式として
  \begin{align*}
    \lvert D^\alpha_x D^\beta_\xi[(x,\xi) \mapsto (D^\gamma_\xi D^\gamma_y a)(x , x , \xi)](x,\xi) \rvert 
    &= \lvert D^\alpha_x D^\beta_\xi D^\gamma_\xi D^\gamma_y a (x , x , \xi)
      + D^\alpha_y D^\beta_\xi D^\gamma_\xi D^\gamma_y a (x , x , \xi) \rvert \\
    &\leq \text{const} (1 + \lvert \xi \rvert)^{m - \lvert \beta + \gamma \rvert}
    = \text{const} (1 + \lvert \xi \rvert)^{(m - \lvert \gamma \rvert) - \lvert \beta \rvert}
  \end{align*}
  よりよい。
\end{Proof}

\begin{Proof}
\itemprof
  smooth な \(k(x,\xi)\) で \((2\pi)^{-n} \int \int e^{i \langle x-y , \xi \rangle} a(x,y,\xi) u(y) dy d\xi = \int e^{i \langle x , \eta \rangle} k(x,\eta) \hat{u}(\eta) d\eta\) となるようなものが存在すれば、 \(k(x,\xi)\) が symbol となることを示せば \(\mathscr{S}\) の元と分かる。
  このような \(k(x,\xi)\) として \(k(x,\xi) = \int e^{i \langle x , \xi^\prime - \xi \rangle} \mathscr{F}_y[a] (x , \xi^\prime - \xi , \xi^\prime) d\xi^\prime\) を与えればよいことが次のようにしてわかる。
  また、定数倍を無視して書いている。
  \begin{align*}
    (Ku)(x)
    &= \int \int e^{i \langle x-y , \xi \rangle} a(x,y,\xi) u(y) dy d\xi \\
    &= \int e^{i \langle x , \xi \rangle} \int e^{- i \langle y , \xi \rangle} a(x,y,\xi) u(y) dy d\xi \\
    &= \int e^{i \langle x , \xi \rangle} \mathscr{F}[y \mapsto a(x,y,\xi) u(y)](\xi) d\xi \\
    &= \int e^{i \langle x , \xi \rangle} \left(\mathscr{F}[y \mapsto a(x,y,\xi)] * \mathscr{F}[y \mapsto u(y)]\right) (\xi) d\xi \\
    &= \int e^{i \langle x , \xi \rangle} \int \mathscr{F}[y \mapsto a(x,y,\xi)] (\xi - \eta)\mathscr{F}[y \mapsto u(y)](\eta) d\eta d\xi \\
    &= \int \int e^{i \langle x , \xi \rangle} \mathscr{F}_y[a](x , \xi - \eta , \xi) \hat{u}(\eta) d\eta d\xi \\
    &= \int e^{i \langle x , \eta \rangle} \int e^{i \langle x , \xi - \eta \rangle} \mathscr{F}_y[a](x , \xi - \eta , \xi) d\xi \hat{u}(\eta) d\eta \\
    &= \int e^{i \langle x , \eta \rangle} k(x,\eta) \hat{u}(\eta) d\eta
  \end{align*}
  積分順序の変更
  \(\Leftarrow\) \(\lvert \mathscr{F}_y[a](x,\xi - \eta , \xi) \rvert \lvert \hat{u}(\eta) \rvert\) が各 \(x\) に対して \((\xi , \eta)\)  の関数として可積分である
  \(\Leftarrow\) どの整数 \(l\) に対して \(x , \xi , \eta\) の式として \(\lvert \mathscr{F}_y[a](x , \xi - \eta , \xi) \rvert \lvert \hat{u}(\eta) \rvert \leq \text{const} (1 + \lvert \xi \rvert)^m (1 + \lvert \xi - \eta \rvert)^{-l} (1 + \lvert \eta \rvert)^{-l}\)
  \(\Leftarrow\) どの整数 \(l\) に対しても \(x,y,\xi\) の式として \(\lvert y \rvert^l \lvert \mathscr{F}_y[a](x,y,\xi) \rvert \leq \text{const} (1 + \lvert \xi \rvert)^m\) となるからこれを示す。
  \(\lvert \beta \rvert = l\) なる \(\beta\) をとれば
  \begin{align*}
    \lvert y^\beta \rvert \lvert \mathscr{F}_y[a] (x,y,\xi) \rvert
    &= \lvert \mathscr{F}_y[D^\beta_y a](x , y , \xi) \rvert
      = \lvert \int e^{i \langle y , y^\prime \rangle} (D^\beta_y a)(x , y^\prime , \xi) dy^\prime \rvert \\
    &\leq \int \lvert (D^\beta_y a)(x , y^\prime , \xi) \rvert dy^\prime
      \leq \int_{y-\text{supp}(a)} (1 + \lvert \xi \rvert)^m dy^\prime
      \leq \text{const} (1 + \lvert \xi \rvert)^m
  \end{align*}
  ゆえに正当化された。
\end{Proof}

\begin{Proof}
\itemprof
  これが symbol になることは以下で \(k(x,\xi) = \sum_{\lvert \alpha \rvert \leq l} \frac{i^{\lvert \alpha \rvert}}{\alpha !} (D^\alpha_y D^\alpha_\xi a)(x,x,\xi) + r_l(x,\xi)\) とかけて右辺が symbol の和になることから従う。
  order については \((D^\alpha_y D^\alpha_\xi a)(x,x,\xi)\) が \(m - \lvert \alpha \rvert\) で \(r_l(x,\xi)\) が \(m - (l + 1)\) であるからそれぞれ order \(m\) ではあるため \(k(x,\xi)\) は order \(m\) である。
\itemthen
  次に \(k(x,\xi)\) の asymptotic development が上の式で与えられることを示す。
  \(k(x,\xi) = \int e^{i \langle x , \zeta \rangle} \mathscr{F}_y[a](x , \zeta , \zeta + \xi) d \zeta\) であることに着目し、 \(\mathscr{F}_y[a](x , \zeta , \zeta + \xi)\) の Taylor 展開を次のように考える。
  \(\mathscr{F}_y[a](x , \zeta , \zeta + \xi) = [h \mapsto \mathscr{F}_y[a](x , \zeta , \xi + h)](\zeta)\) としてこの \(h\) についての Taylor 展開を行い \(\zeta\) を代入する。
  その結果はある関数 \(R_l(x,y,\xi)\) により次のようになる。
  それぞれ \(x , \zeta , \eta\) の式として
  \begin{align*}
    \mathscr{F}_y[a](x , \zeta , \zeta + \eta) = \sum_{\lvert \alpha \rvert \leq l} \frac{i^{\lvert \alpha \rvert}}{\alpha !}(D^\alpha_{\xi} [\mathscr{F}_y[a]])(x , \zeta , \eta) \zeta^\alpha + R_{l}(x , \zeta , \zeta + \eta) \\
    R_l(x , \zeta , \zeta + \eta) = (l + 1)i^{l + 1} \sum_{\lvert \mu \rvert = l + 1} \frac{1}{\mu !} \int_0^1 (D^\mu_\xi \mathscr{F}_y[a])(x , \zeta , t \zeta + \eta) \zeta^\mu (1 - t)^l dt
  \end{align*}
  \\ \ADMIT \\
  ところで \(\zeta , \eta\) の式として
  \begin{align*}
    \int e^{i \langle x ,\zeta \rangle} (D^\alpha_\xi[\mathscr{F}_y[a]])(x,\zeta,\eta)\zeta^\alpha d\zeta
    &= \int e^{i \langle x ,\zeta \rangle} (\mathscr{F}_y[(D^\alpha_\xi a)])(x,\zeta,\eta)\zeta^\alpha d\zeta \\
    &= \int e^{i \langle x ,\zeta \rangle} (\mathscr{F}_y[(D^\alpha_y D^\alpha_\xi a)])(x,\zeta,\eta) d\zeta \\
    &= (D^\alpha_y D^\alpha_\xi a)(x,x,\eta)
  \end{align*}
  が成り立つから \(r_l(x,\xi) = \int e^{i \langle x , \zeta \rangle} R_l(x,\zeta,\zeta + \xi) d\zeta\) と書けば \(k(x,\xi) = \sum_{\lvert \alpha \rvert \leq l} \frac{i^{\lvert \alpha \rvert}}{\alpha !} (D^\alpha_y D^\alpha_\xi a)(x,x,\xi) + r_l(x,\xi)\) が成り立つから、 \(r_l(x,\xi)\) が symbol of order \(m - (l + 1)\) になっていることを示せば \(k(x,\xi)\) なる symbol の formal development として \(\sum_{\lvert \alpha \rvert \leq l} \frac{i^{\lvert \alpha \rvert}}{\alpha !} (D^\alpha_y D^\alpha_\xi a)(x,x,\xi)\) がとれることがわかる。
\end{Proof}

\begin{Proof}
\itemthen
  補題として各 \(\alpha , \beta , k\) に対して \(\lvert (D^\alpha_x D^\beta_\xi \mathscr{F}_y[a])(x , \zeta , \tilde{\zeta} + \eta) \rvert \leq \text{const} (1 + \lvert \tilde{\zeta} + \eta \rvert)^{m - \lvert \beta \rvert} (1 + \lvert \zeta \rvert)^{-k}\) が存在することを示す。
  \(\lvert \gamma \rvert = k\) なる \(g\) をとれば \(x,\zeta,\tilde{\zeta},\eta\) の式として
  \begin{align*}
    \lvert \zeta^\gamma (D^\alpha_x D^\beta_\xi \mathscr{F}_y[a])(x , \zeta , \tilde{\zeta} + \eta) \rvert
    &= \lvert \int (D^\alpha_x D^\beta_\xi a)(x , y, \tilde{\zeta} + \eta) \, D^{\gamma}_y [e^{-i \langle y , \zeta \rangle}] dy \rvert \\
    &= \lvert \int (D^\alpha_x D^\gamma_y D^\beta_\xi a)(x , y, \tilde{\zeta} + \eta) e^{-i \langle y , \zeta \rangle} dy \rvert \\
    &\leq \int \lvert (D^\alpha_x D^\gamma_y D^\beta_\xi a)(x , y, \tilde{\zeta} + \eta) \rvert dy \\
    &\leq \text{const} \int_{y-\text{supp}(a)} (1 + \lvert \tilde{\zeta} + \eta \rvert)^{m - \lvert \beta \rvert} dy = \text{const} (1 + \lvert \tilde{\zeta} + \eta \rvert)^{m - \lvert \beta \rvert}
  \end{align*}
  途中で \(y\)-supp が compact であることを用いた。
  この変形を観察し左辺に \(\xi^\chi\) がついてもよいとわかる。
\itemthen
  \(r_l(x,\xi)\) が symbol of order \(m - (l + 1)\) であること、すなわち各 \(\alpha , \beta\) に対して \(x , \xi\) の式として \(\lvert D^\alpha_x D^\beta_\xi r_l \rvert \leq \text{const} (1 + \lvert \xi \rvert)^{m - (l + 1) - \lvert \beta \rvert}\) であることを示す。
  \((1 + \lvert \xi \rvert)^{-s} (1 + \lvert \eta \rvert)^{s} \leq \text{const} (1 + \lvert \xi - \eta \rvert)^{\lvert s \rvert}\) などが成り立っていた。
  整数 \(k\) に対して \(x,\xi\) の式として
  \begin{align*}
    \lvert D^\alpha_x D^\beta_\xi r_l \rvert
    &= \lvert D^{\alpha}_x D^{\beta}_\xi \int e^{i \langle x , \zeta \rangle} R_l(x , \zeta , \zeta + \xi) d\zeta \rvert
      = \int \sum_{\alpha_1 + \alpha_2 = \alpha} \lvert \zeta^{\alpha_1} (D^{\alpha_2}_x D^{\beta}_\xi [R_l(x , \zeta , \zeta + \xi)]) \rvert d\zeta \\
    &= \sum_{\alpha_1 + \alpha_2 = \alpha} \int \lvert \sum_{\lvert \mu \rvert = l + 1} \frac{1}{\mu !}
      \int_0^1 \zeta^{\alpha_1} D^{\alpha_2}_x D^\beta_\xi \left[(l + 1) D^\mu_\xi \mathscr{F}_y[a](x , \zeta , t \zeta + \xi) \zeta^\mu (1 - t)^l d t \right] \rvert d\zeta \\
    &\leq \text{const} \sum_{\alpha_1 + \alpha_2 = \alpha} \int \int_0^1 \lvert \zeta^{\alpha_1 + \mu} D^{\alpha_2}_x D^{\beta + \mu}_\xi \mathscr{F}_y[a](x , \zeta , t \zeta + \xi) \rvert \, \lvert (1 - t)^l \rvert dt d\zeta \\
    &\leq \text{const} \int \int_0^1 (1 + \lvert t \zeta + \xi \rvert)^{m - \lvert \beta + \mu \rvert} (1 + \lvert \zeta \rvert)^{-k} (1 - t)^l dt d\zeta \\
    &\leq \text{const} \int \int_0^1 (1 + \lvert t \zeta \rvert)^{\lvert m - (l + 1) - \lvert \beta \rvert \rvert} (1 + \lvert \xi \rvert)^{m - (l + 1) - \lvert \beta \rvert} (1 + \lvert \zeta \rvert)^{-k} (1 - t)^l dt d\zeta \\
    &= \text{const} \left[\int \int_0^1 (1 + t \lvert \zeta \rvert)^{\lvert m - (l + 1) - \lvert \beta \rvert \rvert} (1 + \lvert \zeta \rvert)^{-k} (1 - t)^l dt d\zeta \right] (1 + \lvert \xi \rvert)^{m - (l + 1) - \lvert \beta \rvert} \\
  \end{align*}
  この式を見れば、\(K := \lvert m - (l + 1) - \lvert \beta \rvert \rvert \geq 0\) と \(l\) に対して \(\int_0^\infty \int_0^1 (1 + x t)^{K} (1 + x)^{-k} (1 - t)^{l} dt dx\) が有限となる整数 \(k\) が存在することを示せばよい。
  これには \(k = K + k^\prime\) のように書くと
  \begin{align*}
    \int_0^\infty \int_0^1 \left(\frac{1 + x t}{1 + x}\right)^{K} (1 + x)^{-k^\prime} (1 - t)^l dt dx
    &\leq \int_0^\infty \int_0^1 (1 + x)^{-k^\prime} (1 - t)^l dt dx \\
    &= \int_0^\infty (1 + x)^{-k^\prime} dx \int_0^1 (1 - t)^l dt
  \end{align*}
  \(k^\prime\) を十分大きくとると有限である。
\end{Proof}

\begin{Proof}
\itemprof
  \(k(x,\xi) = \sum_{\lvert \alpha \rvert \leq l} \frac{i^{\lvert \alpha \rvert}}{\alpha !} (D^\alpha_y D^\alpha_\xi a)(x,x,\xi) + r_l(x,\xi)\) で \(r_l\) が \(\text{Sym}^{m-(l+1)}\) に含まれていることから、仮定と合わせて \(k(x,\xi) = r_l(x,\xi)\) であるからよい。
\end{Proof}

\begin{Definition}
\itemdefi
  \For \(A \subset \mathbb{R}^n\) , \(\epsilon > 0\) \\
  \Define \(A_{\epsilon}\) := \(\{x : \mathbb{R}^n \mid \text{distance}(x,A) \leq \epsilon\}\)
\itemdefi
  \For \(A\) : operator \\
  \Define \(P\) is \(\epsilon\)-local := \(\forall\) \(u\) : \(C^\infty_0\) , \(\text{supp} \, Pu \subset (\text{supp} \, u)_e\)
\end{Definition}

\begin{Theorem}
\itemprop
  \For \(P\) : \(\Psi DO_m\) whose symbol \(p\) has compact \(x\)-support , \(\epsilon > 0\) \\
  \Then there exists \(P_{\epsilon}\) : \(\Psi DO_m\) which equivalent to \(P\) and is \(\epsilon\)-local
\itemprop
  \For \(\chi_1 , \chi_2\) : real-valued function with compact support on \(\mathbb{R}^n\) , \(P\) : \(\Psi DO_m\) \\
  \Then \(u \mapsto \chi_1 P(\chi_2 u) \in \Psi DO_m\)
\end{Theorem}

\begin{Proof}
\itemprof
  \(\phi(x,y)\) : \(C^\infty(\mathbb{R}^n \times \mathbb{R}^n \to \mathbb{R})\) であって対角線の近傍で \(1\) かつ対角線から \(\epsilon\) 離れたところで \(0\) となるようなものを固定する。
  このとき \(\phi(x,y)\) は
  \(a(x,y,\xi) = \phi(x,y)p(x,\xi)\) とする。
  定理 3.5 を適用するための仮定を確かめる。
  \(x,y\)-supp が compact であることは確かに成り立つ。
  任意の \(\alpha , \beta , \gamma\) に対して \(x,y,\xi\) の式として 
  \begin{align*}
    \lvert D^\alpha_x D^\beta_y D^\gamma_\xi a(x,y,\xi) \rvert
    &\leq \sum_{\alpha_1 + \alpha_2 
    = \alpha} \lvert (D^{\alpha_1}_x D^\beta_y \phi) \rvert \lvert (D^{\alpha_2}_x D^\gamma_\xi p) \rvert \\
    &\leq \text{const} (1 + \lvert \xi \rvert)^{m - \lvert \gamma \rvert}
  \end{align*}
  となるのでよい。
  特にこのように定義した \(a(x,y,\xi)\) を定理 3.5 に適用して得られる \(P_{\epsilon}\) について、
  \begin{itemize}
    \itemenum ある order \(m\) の symbol \(p_{\epsilon}\) による pseudodifferential operator of order \(m\) である
    \itemenum \(p_{\epsilon}\) は formal development として \(\sum_{\alpha} \frac{i^{\lvert \alpha \rvert}}{\alpha !} (D^\alpha_y D^\alpha_\xi a)(x , x , \xi)\) を持つ。
  \end{itemize}
  などが成り立った。
\itemthen
  \(P_{\epsilon}\) が \(P\) と equivalent になることを示す。
  \[
    (D^\alpha_y D^\alpha_\xi [\phi(x,y) a(x,\xi)])(x,x,\xi) = (D^\alpha_y \phi)(x,x) (D^\alpha_\xi a)(x,\xi) = D^\alpha_\xi(x,\xi)
  \]
  なので formal development は \(\sum_{\alpha} \frac{i^{\lvert \alpha \rvert}}{\alpha !} (D^\alpha_\xi p)(x,\xi)\) となる。
  ところで \(p\) もまたこれを formal development に持つことがわかることからこのふたつの symbol から得られる pseudodifferential operator は equivalent である。
\itemthen
  \(P_{\epsilon}\) が \(\epsilon\)-local であることを示す。
  \(x\) に対して任意の \(y\) が \(u (y) = 0\) か \(\lvert x - y \rvert > \epsilon\) を満たしているとき \((P_{\epsilon} u)(x) = 0\) を満たすことからわかる。
\itemprof
  \(a(x,y,\xi) = \chi_1(x)\chi_2(y)p(x,\xi)\) とすると \(x,y\)-supp compact であり条件を満たすことが計算によりわかる。
  ゆえに定理 3.5 よりわかる。
\end{Proof}

\begin{Theorem}
\itemprop
  \For \(P\) : pseudodifferential operator , \(u\) : a function in its domain \(L^2_s\) , \(U\) : open subset in \(\mathbb{R}^n\) \\
  \Then \(\restr{u}{U} \in C^\infty(U)\) \(\rightarrow\) \(\restr{(P u)}{U} \in C^\infty(U)\)
\end{Theorem}

\begin{Proof}
\itemprof
  \(x_0\) : \(U\) を任意にとり fix し、 \((P u)\) が \(x_0\) の近傍でなめらかであることを示す。
  \(\chi_1 , \chi_2\) を \(x_0 \in \text{supp} \chi_1 \subset \text{supp} \chi_2\) で \(x_0\) の近傍で \(\chi_1 = 1\) かつ \(\text{supp} \chi_1\) の近傍で \(\chi_2 = 1\) となるようにとる。
  \(a(x,y,\xi) = \chi_1(x) (1 - \chi_2(y)) p(x,\xi)\) は対角線の近傍で \(0\) であるから定理 3.5 を適用するとここからえられる作用素は infinitely smoothing であり、 \(\chi_1 P ((1 - \chi_2) u)\) は smooth である。
  また \(\chi_2 u \in C^\infty_0\) のため \(\chi_1 P(\chi_2 u) \in C^\infty\) である。
  したがって \(\chi_1 P u = \chi_1 P ((1 - \chi_2) u) - \chi_1 P (\chi_2) u\) もまた滑らかであり、 \(P u\) は \(x_0\) の近傍で滑らかとわかる。
\end{Proof}