\documentclass[dvipdfmx]{jsarticle}
% パッケージ
\usepackage{amsthm}
\usepackage{amsmath,amssymb,mathrsfs}
\usepackage{color}
\usepackage{tikz}

% 定理環境
%% 本体
\theoremstyle{definition}
\newtheorem*{tDefinition}{定義}
\newtheorem*{tTheorem}{定理}
\newtheorem*{tProof}{証明}
\newtheorem*{tNotation}{記法}
\newtheorem*{tRemark}{注意}
\newtheorem*{tWhen}{設定}

\newenvironment{Mini}{
  \begin{minipage}[t]{1\hsize}
  \setlength{\parindent}{10pt}
  \begin{itemize}
  \setlength{\labelsep}{10pt}
}{
  \end{itemize}
  \vspace{5pt}
  \end{minipage}
}

\newenvironment{Definition}[1][\quad]{
  \begin{tDefinition}
  #1 \\
  \begin{Mini}
}{
  \end{Mini}
  \end{tDefinition}
}

\newenvironment{Theorem}[1][\quad]{
  \begin{tTheorem}
  #1 \\
  \begin{Mini}
}{
  \end{Mini}
  \end{tTheorem}
}

\newenvironment{Proof}[1][\quad]{
  \begin{tProof}
  #1 \\
  \begin{Mini}
      }{
  \end{Mini}
  \end{tProof}
}

\newenvironment{When}{
  \begin{tWhen}
  \quad \\
  \begin{Mini}
}{
  \end{Mini}
  \end{tWhen}
}

\newenvironment{Remark}{
  \begin{tRemark}
}{
  \end{tRemark}
}

%% マーク
\newcommand{\itemwhen}{\item[\(\bigcirc\)]}
\newcommand{\itemnote}{\item[!]}

\newcommand{\itemdefi}{\item[\(\square\)]}
\newcommand{\itemprop}{\item[\(\vartriangleright\)]}
\newcommand{\itemand}{\item[\(-\)]}

\newcommand{\itemprof}{\item[\(\because\)]}
\newcommand{\itemthen}{\item[\(\rightsquigarrow\)]}

\newcommand{\itemenum}{\item[\(+\)]}
\newcommand{\itembase}{\item[\(\bullet\)]}
\newcommand{\itemwith}{\item[\(-\)]}

% 記述関係
\newenvironment{indentblock}{
  \\
  \hspace*{5mm}
  \begin{minipage}{0.8\textwidth}
}{
  \end{minipage}
  \\
}

%コマンド関連
\newcommand{\place}{\_}
\newcommand{\restr}[2]{\left. {#1} \right| _{#2}}
\newcommand{\txt}{\texttt}

%% 宣言
\newcommand{\declare}[1]{\textcolor[rgb]{0.1, 0.8, 0.2}{#1 }}
\newcommand{\For}{\declare{For}}
\newcommand{\Define}{\declare{Define}}
\newcommand{\Let}{\declare{Let}}
\newcommand{\IfHold}{\declare{If}}
\newcommand{\Then}{\declare{Then}}
\newcommand{\Take}{\declare{Take}}
\newcommand{\Fix}{\declare{Fix}}
\newcommand{\Return}{\declare{Return}}

%% 置く
\newcommand{\WIP}{\textcolor{red}{工事中}}
\newcommand{\SORRY}{\textcolor{red}{わかりませんでした}}
\newcommand{\ADMIT}{\textcolor{blue}{認めます}}
\newcommand{\AIMAI}[1]{\textit{#1}}

\begin{document}
\section*{\(\tilde{K}(X^+) \cong K(X)\)}
ここでは普通の \(K\)-理論を \(K\) で書き、複体を用いた場合 \(D\) で書き、 Appendix にある定義を \(Q\) で書く。

\begin{Theorem}
\itemprop
  \For \((X,A)\) : locally compact pair \\
  \Then \(D(X,A) \cong L(X,A)\)
\itemprop
  \For \((X,A)\) : compact pair \\
  \Then \(K(X,A) \cong Q(X,A)\) 
\end{Theorem}

\begin{Proof}
\itemprof
  定義の同値性より。
\itemprof
  Proposition (A.1) より。
\end{Proof}

\begin{Theorem}
\itemprop
  \For \(X\) : locally compact space \\
  \Then \(0 \to D(X^+, \infty) \overset{\alpha}{\to} D(X^+) \overset{i^*}{\to} D(\infty) \to 0\) is exact
\end{Theorem}

\begin{Proof}
\itemprof
  Proposition (A.3) により \((X^+ , \infty)\) が compact pair であることから、 \(0 \to Q(X^+ , \infty) \to K(X^+) \to K(\infty) \to 0\) が完全である。
  これと上の同型を合わせて考えればよい。
\end{Proof}

\begin{Theorem}
\itemprop
  \(D(X^+, \infty) \overset{r^*}{\to} D(X^+ - \infty)\) is isomorphism
\end{Theorem}

\begin{Proof}
\itemprof
  Proposition (A.4) により \((X^+ , \infty)\) が compact pair であることから、 \(Q(X,A) \to Q(X-A)\) が同型である。
  これと上の同型を合わせて考えればよい。
\end{Proof}

\end{document}