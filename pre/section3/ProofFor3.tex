\documentclass[dvipdfmx]{jsarticle}
% パッケージ
\usepackage{amsthm}
\usepackage{amsmath,amssymb,mathrsfs}
\usepackage{color}
\usepackage{tikz}

% 定理環境
%% 本体
\theoremstyle{definition}
\newtheorem*{tDefinition}{定義}
\newtheorem*{tTheorem}{定理}
\newtheorem*{tProof}{証明}
\newtheorem*{tNotation}{記法}
\newtheorem*{tRemark}{注意}
\newtheorem*{tWhen}{設定}

\newenvironment{Mini}{
  \begin{minipage}[t]{1\hsize}
  \setlength{\parindent}{10pt}
  \begin{itemize}
  \setlength{\labelsep}{10pt}
}{
  \end{itemize}
  \vspace{5pt}
  \end{minipage}
}

\newenvironment{Definition}[1][\quad]{
  \begin{tDefinition}
  #1 \\
  \begin{Mini}
}{
  \end{Mini}
  \end{tDefinition}
}

\newenvironment{Theorem}[1][\quad]{
  \begin{tTheorem}
  #1 \\
  \begin{Mini}
}{
  \end{Mini}
  \end{tTheorem}
}

\newenvironment{Proof}[1][\quad]{
  \begin{tProof}
  #1 \\
  \begin{Mini}
      }{
  \end{Mini}
  \end{tProof}
}

\newenvironment{When}{
  \begin{tWhen}
  \quad \\
  \begin{Mini}
}{
  \end{Mini}
  \end{tWhen}
}

\newenvironment{Remark}{
  \begin{tRemark}
}{
  \end{tRemark}
}

%% マーク
\newcommand{\itemwhen}{\item[\(\bigcirc\)]}
\newcommand{\itemnote}{\item[!]}

\newcommand{\itemdefi}{\item[\(\square\)]}
\newcommand{\itemprop}{\item[\(\vartriangleright\)]}
\newcommand{\itemand}{\item[\(-\)]}

\newcommand{\itemprof}{\item[\(\because\)]}
\newcommand{\itemthen}{\item[\(\rightsquigarrow\)]}

\newcommand{\itemenum}{\item[\(+\)]}
\newcommand{\itembase}{\item[\(\bullet\)]}
\newcommand{\itemwith}{\item[\(-\)]}

% 記述関係
\newenvironment{indentblock}{
  \\
  \hspace*{5mm}
  \begin{minipage}{0.8\textwidth}
}{
  \end{minipage}
  \\
}

%コマンド関連
\newcommand{\place}{\_}
\newcommand{\restr}[2]{\left. {#1} \right| _{#2}}
\newcommand{\txt}{\texttt}

%% 宣言
\newcommand{\declare}[1]{\textcolor[rgb]{0.1, 0.8, 0.2}{#1 }}
\newcommand{\For}{\declare{For}}
\newcommand{\Define}{\declare{Define}}
\newcommand{\Let}{\declare{Let}}
\newcommand{\IfHold}{\declare{If}}
\newcommand{\Then}{\declare{Then}}
\newcommand{\Take}{\declare{Take}}
\newcommand{\Fix}{\declare{Fix}}
\newcommand{\Return}{\declare{Return}}

%% 置く
\newcommand{\WIP}{\textcolor{red}{工事中}}
\newcommand{\SORRY}{\textcolor{red}{わかりませんでした}}
\newcommand{\ADMIT}{\textcolor{blue}{認めます}}
\newcommand{\AIMAI}[1]{\textit{#1}}

\newcommand{\Ktheory}[1]{K_{\mathbb{K}}(#1)}
\newcommand{\KtheoryReduced}[1]{\tilde{K}_{\mathbb{K}}(#1)}
\newcommand{\KtheoryRel}[2]{K_{\mathbb{K}}(#1, #2)}
\newcommand{\Kcohomology}[3]{K_{\mathbb{K}}^{#1}(#2, #3)}
\newcommand{\KcohomologyReduced}[2]{\tilde{K}_{\mathbb{K}}^{#1}(#2)}
\newcommand{\KtheoryCpx}[1]{K_{\mathbb{K},cpt}(#1)}
\newcommand{\KtheoryCpxRel}[2]{K_{\mathbb{K},cpt}(#1, #2)}

\begin{document}

\section{Thom 同型の同型性について}
コンパクト空間の場合の \(K\)-理論に話を移し、 Bott 周期性との関連により示す。

\((\mathbb{K} , \mathbb{G} , k , V) \leftarrow (\mathbb{R}, \text{Spin} , 8 , \mathbb{R}^8), (\mathbb{C}, \text{Spin}^c , 2 , \mathbb{R}^2)\) とする。
ori.metrized.vec.bundle with \(\mathbb{G}\).st を \(\mathbb{G}\).st と書く。
また通常のコンパクト空間に対してのみ定義された \(K\) 理論と区別する必要がある(何がどっちで成り立っているかわかりにくい)ため、複体で定義したものを \(\KtheoryCpx{X}\) と書く。
pair of space で locally compact space と closed subset の空間対とする。

\begin{Definition}[Thom 複体と Thom 写像の定義]
\itemwhen
  \Fix \(\xi = (q : W \to X)\) : \(\mathbb{G}\).st where \(X\) : compact
\itemdefi
  \Define \(\mu(\xi)\) : cpt.supp.cpx over \(W\) := \\
  \(0 \to q^*S^+ \overset{\omega}{\to}q^*S^- \to 0\) \\
  \Define Thom class : \(\KtheoryCpx{W}\) := represented by \(\mu(\xi)\) \\
  \Define Thom homomorphism of \(\xi\) \ldots \(T_{\xi}\) : \(\KtheoryCpx{X} \to \KtheoryCpx{W}\) := \([x] \mapsto [q^*x \otimes \mu(\xi)]\)
\end{Definition}

%\begin{Theorem}
%\itemprop
%  \For \(W_1 \overset{q_1}{\to} X , W_2 \overset{q_2}{\to} X\) : \(\mathbb{G}\).st. , \(f\) : isomorphism of \(\mathbb{G}\).st. \(W_1 \to W_2\) \\
%  \Then \(T_{W_1 \overset{q_1}{\to} X} = f^* \circ T_{W_2 \overset{q_2}{\to} X}\) : \(\KtheoryCpx{X} \to \KtheoryCpx{W_1}\)
%\itemprop
%  \For \(\xi = (W \overset{q}{\to} X)\) : \(\mathbb{G}\).st , \(f\) : proper map of \(Y \to X\) \\
%  \Then \(\mu(f^* \xi) \cong f^* \mu(\xi)\) : cpt.supp.cpx over \(\text{Total}(f^* \xi)\)
%\itemprop
%  \For \(\xi_1 , \xi_2\) : \(\mathbb{G}\).st \\
%  \Then \([\mu(\xi_1 \times \xi_2)] = [\text{Proj}^* \mu(\xi_1) \otimes \text{Proj}^* \mu(\xi_2)]\) : \(\KtheoryCpx{\text{Total}(\xi_1) \times \text{Total}(\xi_2)}\)
%\end{Theorem}

%\begin{Proof}
%\itemprof
%  Thom 複体が誘導されるものと可換であることからよい。
%\itemprof
%  Thom 複体が誘導されるものと可換であることからよい。
%\itemprof
%  \WIP
%\end{Proof}

\begin{Theorem}[Thom 写像を移し替える]
%\itemprop
%  \Let \(\phi\) : \(W \to D(W) - S(W)\) := \(v \mapsto v / (1 + \lVert v \rVert)\) \\
%  \Then \(h\) : \(\KtheoryCpxRel{D(W)}{S(W)} \to \KtheoryCpx{W}\) := \([x] \mapsto [\phi^* (\restr{x}{D(W) - S(W)})]\) is isomorphism
%\itemprop
%  \Then \(h^{-1}T_{\xi} x = [q^* x] \cdot [\restr{\mu(W)}{(D(W), S(W))}]\)
\itemdefi
  \Let \(\beta\) : \(\KcohomologyReduced{*}{X} \to \KcohomologyReduced{*+k}{X}\) := by Bott periodicity \\
  \Let \(\tilde{\mu}(\xi)\) : \(\KcohomologyReduced{n}{\text{Total}(\xi)}\) := defined by \(\mu(\xi) \in \KtheoryCpx{W} \cong \KtheoryReduced{\text{Total}(\xi)} \cong \KcohomologyReduced{n}{\text{Total}(\xi)}\)
\itemprop
  \Let \(\tilde{T}_{\xi}\) : \(\Kcohomology{0}{X}{\emptyset} \to \KcohomologyReduced{n}{\text{Total}(\xi)}\) := induced by \\
  \(\KcohomologyReduced{0}{X} \cong \KtheoryCpx{X} \overset{T_{\xi}}{\to} \KtheoryCpx{\text{Total}(\xi)} \cong \KtheoryReduced{\text{Thom}(\xi)} \cong \KcohomologyReduced{n}{\text{Thom}(\xi)}\) \\
  \Then \(\tilde{T}_{\xi} = x \mapsto x \cdot \tilde{\mu}(\xi)\)
\itemprop
  \Then \(\tilde{\mu}(\xi)\) is \(K_{\mathbb{K}}\)-orientation
\end{Theorem}

\begin{Proof}
%\itemprof
%  \(\KtheoryCpxRel{W^+}{\infty}\) との同型を経由すれば、コンパクト空間の対の間の写像であり通常の \(K\) 理論での同型からわかる。
%\itemprof
%  \(\psi\) : \(W \to W\) := \(v \mapsto v / (1 + \lVert w \rVert)\) として
%  \(\psi^* \mu(W)\) と \(\mu(W)\) が同じ \(K(W)\) の元を定めていることを示せばよいことがわかる。
%  \(f\) : \(q^*(S^+) \to \psi^* q^*(S^+)\) と \(g\) : \(q^*(S^-) \to \psi^* q^*(S^-)\) を \((w,v) \mapsto (w / (1 + \lVert w \rVert), v)\) と定値すると、ベクトル束の同型であり次の図式が可換になるからよい。
%  \[\begin{matrix}
%    0 & \to & q^*(S^+) & \to & q^*(S^-) & \to & 0 \\
%     & & \downarrow f & & \downarrow g & & \\
%    0 & \to & \psi^*q^*(S^+) & \to & \psi^*q^*(S^-) & \to & 0
%  \end{matrix}\]
%\itemprof
%  これは積を持つ一般コホモロジーの一般論からわかる。
\itemprof
  これは定義による。
\itemprof
  \(i_x\) の定義(実際には \(\mathbb{R}^n \to \xi_x\) 分の定義の揺らぎがある)と Thom class の乗法性により、次の場合に帰着する。
  \(0 \to \mathbb{R}^{k} \times S^{+} \to \mathbb{R}^{k} \times S^{-} \to 0\) なる \(\KtheoryCpx{\mathbb{R}^{k}} \cong \KcohomologyReduced{0}{S^k}\) の元は Bott 周期性 \(\beta\) : \(\KcohomologyReduced{0}{S^0} \to \KcohomologyReduced{-k}{S^0} \cong \KcohomologyReduced{0}{S^k}\) により \(1 \in \KcohomologyReduced{0}{S^0}\) の移った元であることを示す。
\end{Proof}

\end{document}