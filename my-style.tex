% パッケージ
\usepackage{amsthm}
\usepackage{amsmath,amssymb}
\usepackage{color}
\usepackage{tikz}

% 定理環境
%% 本体
\theoremstyle{definition}
\newtheorem*{tDefinition}{定義}
\newtheorem*{tTheorem}{定理}
\newtheorem*{tProof}{証明}
\newtheorem*{tNotation}{記法}
\newtheorem*{tRemark}{注意}
\newtheorem*{tWhen}{設定}

\newenvironment{Mini}{
  \begin{minipage}[t]{0.9\hsize}
  \setlength{\parindent}{12pt}
  \begin{itemize}
  \setlength{\labelsep}{10pt}
}{
  \end{itemize}
  \vspace{5pt}
  \end{minipage}
}

\newenvironment{Definition}{
  \begin{tDefinition}
  \begin{Mini}
}{
  \end{Mini}
  \end{tDefinition}
}

\newenvironment{Theorem}{
  \begin{tTheorem}
  \begin{Mini}
}{
  \end{Mini}
  \end{tTheorem}
}

\newenvironment{Proof}{
  \begin{tProof}
  \begin{Mini}
      }{
  \end{Mini}
  \end{tProof}
}

\newenvironment{When}{
  \begin{tWhen}
  \begin{Mini}
}{
  \end{Mini}
  \end{tWhen}
}

%% マーク
\newcommand{\itemwhen}{\item[\(\bigcirc\)]}
\newcommand{\itemnote}{\item[!]}

\newcommand{\itemdefi}{\item[\(\square\)]}
\newcommand{\itemprop}{\item[\(\vartriangleright\)]}
\newcommand{\itemand}{\item[\(-\)]}

\newcommand{\itemprof}{\item[\(\because\)]}
\newcommand{\itemthen}{\item[\(\rightsquigarrow\)]}

\newcommand{\itemenum}{\item[\(\bullet\)]}
\newcommand{\itemwith}{\item[\(-\)]}

%コマンド関連

\newcommand{\WIP}{\textcolor{red}{工事中}}
\newcommand{\SORRY}{\textcolor{red}{わかりませんでした}}
\newcommand{\ADMIT}{\textcolor{blue}{認めます}}
\newcommand{\AIMAI}[1]{\textit{#1}}