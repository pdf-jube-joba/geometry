\begin{Definition}[一般の束]
\itemdefi
  \Define bundle := pair of
  \begin{itemize}
    \itemenum base space (\(B\)) : space
    \itemenum total space (\(E\)) : space
    \itemenum projection : \(E \to B\)
  \end{itemize}
\itemdefi
  \For \(\xi \leftarrow (B , E , \pi)\) : bundle , \(b\) : \(B\) \\
  \Define \(\xi_b\) : space := \(\pi^{-1}(x)\)
\end{Definition}

\begin{Definition}[束の間の写像や合成、束の同型]
\itemdefi
  \For \(\xi_1 \leftarrow (B_1 , E_1 , \pi_1)\) , \(\xi_2 \leftarrow (B_2 , E_2 , \pi_2)\) : bundle \\
  \Define \(\xi_1 \to \xi_2\) := pair of
  \begin{itemize}
    \itemenum \(f\) : \(B_1 \to B_2\)
    \itemenum \(\tilde{f}\) : \(E_1 \to E_2\)
    \itemwith commutativity : \(f \circ \pi_1 = \pi_2 \circ \tilde{f}\)
  \end{itemize}
\itemdefi
  \For \(\xi_1\) , \(\xi_2\) , \(\xi_3\) : bundle \\
  \For \(\phi_1 \leftarrow (f , \tilde{f})\) : \(\xi_1 \to \xi_2\) , \(\phi_2 \leftarrow (g , \tilde{g})\) : \(\xi_2 \to \xi_3\) \\
  \Define \(\phi_2 \circ \phi_1\) : \(\xi_1 \to \xi_3\) := \((g \circ f , \tilde{g} \circ \tilde{f})\)
\itemdefi
  \For \(\xi_1\) , \(\xi_2\) : bundle , \(f\) : \(\xi_1 \to \xi_2\) \\
  \Define \(f\) is isomorphism := \AIMAI{by natural definition}
\itemdefi
  \For \(\xi_1\) , \(\xi_2\) : bundle \\
  \Define \(\xi_1\) is isomorphic to \(\xi_2\) := \AIMAI{by natural definition}
\end{Definition}

\begin{Definition}[空間の上の束]
\itemdefi
  \For \(B\) : space \\
  \Define bundle over \(B\) := structure on \(B\) with
  \begin{itemize}
    \itemenum total space (\(E\)) : space
    \itemenum projection : \(E \to B\)
  \end{itemize}
\itemdefi
  \For \(\xi_1 \leftarrow (E_1 , \pi_1)\) , \(\xi_2 \leftarrow (E_2 , \pi_2)\) : bundle over \(B\) \\
  \Define \(\xi_1 \to \xi_2\) := pair of
  \begin{itemize}
    \itemenum \(\tilde{f}\) : \(E_1 \to E_2\)
    \itemwith commutativity : \(\pi_2 \circ \tilde{f} = \pi_1\)
  \end{itemize}
\end{Definition}

\begin{Remark}
「空間の上の束」と「束」は特に区別しなくてよい。
ここ以降あまり区別しない。
\end{Remark}

\begin{Definition}[積束・自明束・制限束・誘導束]
\itemdefi
  \For \(B , F\) : space \\
  \Define product bundle over \(B\) with fiber \(F\) : bundle over \(B\) := \((B \times F , \txt{projection})\)
\itemdefi
  \For \(\xi \leftarrow (B , \place , \place)\) : bundle \\
  \Define \(\xi\) is trivial := \AIMAI{\(\xi\) is isomorphic to some product bundle as bundle over \(B\)}
\itemdefi
  \For \(\xi \leftarrow (B , \place , \pi)\) : bundle , \(U\) : subset of B \\
  \Define restriction bundle (\(\restr{\xi}{U}\)) : bundle over \(U\) := \((\pi^{-1}(U) , \restr{\pi}{\pi^{-1}(U)})\)
\itemdefi
  \For \(\xi \leftarrow (B , E , \pi)\) : bundle , \(f\) : \(B^{\prime} \to B\) \\
  \Define induced bundle (\(f^*\xi\)) : bundle over \(B^{\prime}\) :=
  \begin{itemize}
    \itemenum total space := \(\{(b , e) \in B^{\prime} \times E \mid \pi(e) = f(b)\}\)
    \itemenum projection := restriction of projection \(B^{\prime} \times E \to B^{\prime}\)
  \end{itemize}
\end{Definition}

\begin{Theorem}[制限束と制限写像の誘導束は同型、誘導束の構成は関手的である(ただし同型を除く)]
\itemprop
  \For \(\xi \leftarrow (B , E ,\pi)\) : bundle , \(U\) : subset of B \\
  \Let \(i\) : \(U \to B\) := injection \\
  \Let \(E \to B \times E\) := \(e \mapsto (\pi(e) , e)\) \\
  \Then this becomes \(\txt{total space of} \, \restr{\xi}{U} \to \txt{total space of} \, i^*\xi\) \\
  \Then this is isomorphism
\itemnote
  We identify the two in this way
\itemprop
  \For \(\xi \leftarrow (B , E , \pi)\) : bundle \\
  \For \(f\) : \(B_2 \to B_1\) , \(g\) : \(B_1 \to B\) \\
  \Let \(B_2 \times B_1 \times E \to B_2 \times E\) := projection \\
  \Then this becomes \(\text{total space of} \, f^*g^* \xi \to \text{total space of} \, (g \circ f)^* \xi\) \\
  \Then this is isomorphism
\itemnote
  We identify the two in this way 
\end{Theorem}

\begin{Definition}[ファイバー束とその間の射や同型]
\itemdefi
  \Define fiber bundle := structure of
  \begin{itemize}
    \itembase \(\xi \leftarrow (B , E , \pi)\) : bundle
    \itemwith local triviality :
      \(\forall\) (\(b\) : \(B\)) ,\\
      \(\exists\) (\(U\) : open neighborhood of \(b\)) (\(\phi\) : \(\pi^{-1}(U) \to U \times \pi^{-1}(x)\)) , \\
      \(\phi\) is isomorphism and \(\txt{projection} \circ \phi = \pi\)
  \end{itemize}
\itemdefi
  \Define fiber bundle with typical fiber \(F\) := fiber bundle (\((B , E , \pi)\)) such that \(\forall\) (\(b\) : \(B\)) , \(E_b \cong F\)
\itemdefi
  \Define map of fiber bundle := map as bundle
\itemdefi
  \Define isomorphism of fiber bundle := \AIMAI{by natural definition}
\end{Definition}

\begin{Theorem}[ファイバー束と束に対する操作について]
\itemprop
  product bundle (\((B , B \times F , -)\)) becomes fiber bundle with fiber \(F\)
\itemprop
  restriction bundle of fiber bundle with fiber \(F\) becomes fiber bundle with fiber \(F\)
\itemprop
  induced bundle of fiber bundle with fiber \(F\) becomes fiber bundle with fiber \(F\)
\end{Theorem}

\begin{Definition}
\itemnote
  ベクトル束とその間の射や同型を定義する。
\itemdefi
  \Define vector bundle := structure of
  \begin{itemize}
    \itembase \((B , E , \pi)\) : bundle
    \itemenum fiberwise vector space structure :
      for (\(b\) : \(B\)) , vector space on \(E_b\)
    \itemwith local triviality :
      \(\forall\) (\(b\) : \(B\)) ,\\
      \(\exists\) (\(U\) : open neiborhood of \(b\)) (\(\phi\) : \(\pi^{-1}(U) \to U \times \pi^{-1}(x)\)) , \\
      (\(\forall\) (\(y\) : \(U\)) , \(\txt{projection} \, \circ \phi \mid \pi^{-1}(y)\) : \(E_y \to E_x\) is isomorphism as vector space) and \(\txt{projection} \, \circ \phi = \pi\)
  \end{itemize}
\itemdefi
  \Define \(n\)-dimensional vector bundle :=
  vector bundle (\((B , E , \pi)\)) such that \(\forall\) (\(b\) : B) , \(\dim \pi^{-1}(b) = n\)
\itemdefi
  \Define map of vector bundle := map as bundle such that \AIMAI{linear on each fiber}
\end{Definition}

\begin{Definition}
\itemnote
  ベクトル束構造が誘導束などに対して自然に誘導されることを示す。
\itemdefi
  \For \(\xi = (B , E , \pi)\) : vector bundle \\
  \For \(f\) : \(B^{\prime} \to B\) \\
  \Define induced vector bundle (\(f^*\xi = (B^{\prime} , E^{\prime} , \pi^{\prime})\)) : vector bundle :=
  \begin{itemize}
    \itembase \(f^*\xi\)
    \itemenum fiberwise vector space structure :=
      for (\(b\) : \(B^{\prime}\)) , \AIMAI{induced by vector space structure on \(E_{f(b)}\)}
  \end{itemize}
  hint \(E^{\prime}_b = \{(b^{\prime} , e) \in B^{\prime} \times E \mid f(b^{\prime}) = \pi(e) , \text{projection }(b^{\prime} , e) = b\} = \{b\} \times E_{f(b)}\)
\end{Definition}

\begin{Definition}
\itemnote
  主束やその間の射を定義する。
\itemdefi
  \For \(G\) : Lie group ,\\
  \Define principal \(G\) bundle := structure of
  \begin{itemize}
    \itembase \((B , P , \pi)\) : bundle
    \itemenum \(\place \cdot \place\) : right action of \(G\) on \(P\)
    \itemwith local triviality :
      \(\forall\) (\(b\) : \(B\)) ,\\
      \(\exists\) (\(U\) : open neiborhood of \(b\)) (\(\phi\) : \(\pi^{-1}(U) \to U \times G\)) , \\
      \(\phi\) is isomorphism as \(G\) space and \(\txt{projection} \circ \phi = \pi\)
  \end{itemize}
\itemdefi
  \Define map of principal \(G\) bundle := map as bundle such that \AIMAI{\(G\) morphism on each fiber}
\end{Definition}

\begin{Definition}
\itemnote
  主束構造が誘導束などに対して自然に誘導されることを示す。
\itemdefi
  \For \(G\) : Lie group \\
  \For \(\xi = (P , B , \pi)\) : principal \(G\) bundle \\
  \For \(f\) : \(B^{\prime} \to B\) \\
  \Define induced princiapl \(G\) bundle (\(f^*\xi = (B^{\prime} , E^{\prime} , \pi^{\prime})\)) : principal \(G\) bundle :=
  \begin{itemize}
    \itembase \(f^*\xi\)
    \itemenum \((\place \cdot \place)\) := induced by \((b , e) \cdot g \mapsto (b , e \cdot g)\)
  \end{itemize}
\end{Definition}

\begin{Definition}
\itemnote
  \(G\) 同変な束を定義する。
\itemwhen
  \Fix \(G\) : topological group
\itemdefi
  \Define \(G\) equivariant bundle := structure of
  \begin{itemize}
    \itembase \(\xi = (B , E , \pi)\) : fiber bundle
    \itemenum left action of \(G\) on B
    \itemenum left action of \(G\) on E
    \itemwith forall (\(g\) : \(G\)) , \(\pi \circ (\place \cdot g) = (\place \cdot g) \circ \pi\)
  \end{itemize}
\itemdefi
  \Define \(G\) equivariant vector bundle := structure of
  \begin{itemize}
    \itembase \(\xi = (B , E , \pi)\) : \(G\) equivariant bundle and vector bundle
    \itemwith
      forall (\(g\) : \(G\)) , \((- \cdot g)\) : \(\xi \to \xi\) is vector bundle map
  \end{itemize}
\end{Definition}

\begin{Definition}
\itemnote 代数束・加群束を定義する
\itemdefi 
  \Define algebra bundle :=
  \begin{itemize}
    \itembase \(\xi = (B , E , \pi)\) : vector bundle
    \itemenum \(\place \cdot \place\) :
      for (\(b\) : \(B\)) , algebra structure on \(E_b\)
    \itemwith \AIMAI{algebra structure is locally trivial}
  \end{itemize}
\itemdefi
  \Define module bundle of algebra bundle (\(B , E^{\prime} , \pi^{\prime}\)) :=
  \begin{itemize}
    \itembase \(\xi = (B , E , \pi)\) : vector bundle
    \itemenum \(\place \cdot \place\) :
      for (\(b\) : \(B\)) , \(E^{\prime}_b\) module structure on \(E_b\) 
    \itemwith \AIMAI{module structure is locally trivial}
  \end{itemize}
\end{Definition}