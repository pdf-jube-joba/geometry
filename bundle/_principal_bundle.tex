\begin{Definition}
\itemnote
  主束と表現から束を作る。
\itemwhen \((\pi : P \to M)\) : principal \(G\) bundle
\itemdefi
  for (\((F , \rho)\) : \(G\) space) ,\\
  \(P \times_{\rho} F\) : fiber bundle with fiber \(F\) :=
  \begin{itemize}
    \itemenum base space := \(B\)
    \itemenum total space := \(\{(p , f) \in P \times F\} / G\) where \\
      \(g \cdot (p , f) := (p \cdot g^{-1} , \rho(g) f)\)
  \end{itemize}
\itemdefi
  for (\(V , \rho\) : representation of \(G\)) ,\\
  \(P \times_{\rho} V\) : vector bundle :=
  \begin{itemize}
    \itemenum bundle := \(P \times_{\rho} V\)
    \itemenum vector structure := \AIMAI{induced by vector structure on \(V\)}
  \end{itemize}
\itemdefi
  for (\(H , \rho\) : homomorphism of Lie group) ,\\
  \(P \times_{\rho} H\) : principal \(H\) bundle :=
  \begin{itemize}
    \itemenum bundle := \(P \times_{\rho} H\)
    \itemenum right action of \(H\) := \\
    for (\(e\) : \(P \times_{\rho} H\)) (\(h\) : H) ,\\
    take (\(p\) : \(P\)) (\(h_1\) : H) such that \(e = [p , h_1]\) ,
    return \([p , h_1 \cdot h]\)
  \end{itemize}
\end{Definition}

\begin{Definition}
\itemnote
  部分束を定義する。
\itemdefi
  for (\((\pi : P \to M)\) : principal \(G\) bundle) (\(H\) : subgroup of \(G\))\\
  define \(P^{\prime}\) : subspace of \(P\) is principal \(H\) bundle \(\iff\)
  \(\pi : P^{\prime} \to M\) is principal \(H\) bundle under \AIMAI{restriction action}
\itemprop
  \(P\) has principal \(H\) subbundle \(\iff\) \AIMAI{座標が reduction できるとき}
\end{Definition}

\begin{Definition}
\itemnote
  主束をベクトル束から構成する。
\itemdefi
  for (\((\pi : E \to B)\) : \(n\) dim vector bundle) ,\\
  \(P_{GL}(E)\) : principal \(\text{GL}(n , \mathbb{R})\) bundle over \(B\) :=
  \begin{itemize}
    \itemenum total := \(\coprod_{x : B} \{f : \mathbb{R}^n \to E_x \mid f \text{ is invertible }\}\)
    \itemenum right \(G\) action := by formula ... \((\phi \cdot g)(x) = \phi(g \cdot x)\)
  \end{itemize}
\itemdefi
  for (\((\pi : E \to B)\) : oriented vector bundle) ,\\
  \(P_{GL+}(E)\) : principal \(\text{GL+}(n , \mathbb{R}^n)\) bundle over \(B\) :=
  \begin{itemize}
    \itemenum total := \(\coprod_{x : B} \{f : \mathbb{R}^n \to E_x \mid f \text{ is invertible and preserve orientation }\}\)
    \itemenum right \(G\) action := \AIMAI{define like the way as before}
  \end{itemize}
\itemdefi
  for (\((\pi : E \to B)\) : metrized vector bundle) ,\\
  \(P_{O}(E)\) : principal \(\text{O}(n)\) bundle over \(B\) :=
  \begin{itemize}
    \itemenum total := \(\coprod_{x : B} \{f : \mathbb{R}^n \to E_x \mid f \text{ is invertible and preserve metric }\}\)
    \itemenum right \(G\) action := \AIMAI{define like the way as before}
  \end{itemize}
\itemdefi
  for (\((\pi : E \to B)\) : oriented metrized vector bundle) ,\\
  \(P_{SO}(E)\) : principal \(\text{SO}(n)\) bundle over \(B\) :=
  \begin{itemize}
    \itemenum total := \(\coprod_{x : B} \{f : \mathbb{R}^n \to E_x \mid f \text{ is invertible and preserve orientation and metric }\}\)
    \itemenum right \(G\) action := \AIMAI{define like the way as before}
  \end{itemize}
\end{Definition}

\begin{Definition}
\itemnote
  主束の構成の二つ目
\itemdefi
  for (\((\pi : E \to B)\) : \(n\) dim vector bundle) ,\\
  \(P_{GL}(E)\) : principal \(G\) bundle := 
  \begin{itemize}
    \itemenum total := \(\coprod_{x : B} \{(v_1 , \ldots , v_n) \in E_x \times \cdots \times E_x \mid \{v_i \mid i\} \text{ is basis }\}\)
    \itemenum right \(G\) action := by formula ... \((v_1 , \ldots , v_n) \cdot (a_{ij})_{ij} = (\sum_j a_{1j} e_j , \ldots , \sum_j a_{nj} e_j)\)
  \end{itemize}
\itemdefi
  \AIMAI{似た感じで oriented , metric , oriented and metric の場合それぞれできる}
\end{Definition}

\begin{Theorem}
\itemnote
  当然二つの構成は自然に同型
\itemprop
  for (\((\pi : E \to M)\) : \(n\) dim vector budnel) ,\\
  there exists natural isomorphism on two definition
\end{Theorem}

\begin{Theorem}
\itemnote
  フレーム束からベクトル束を復元する。
\itemprop
  for (\((\pi : E \to M)\) : \(n\) dim vector bundle) ,\\
  let \(\rho\) : representation of \(\text{GL}(n , \mathbb{R}^n)\) to \(\mathbb{R}^n\) := standard definition
  then there exists natural isomorphism \(E \cong P_{GL}(E) \times_{\rho} \mathbb{R}^n\)
\end{Theorem}

\begin{Definition}
\itemnote
  主束の接ベクトルについて、水平を定義する
\itemwhen \((P , M , \pi)\) : principal \(G\) bundle
\itemwhen \(\mathfrak{g}\) := Lie algebra of \(G\)
\itemdefi
  (\(v\) : \(T_{x}P\)) is vertical \(\iff\) \(\forall\) (\(f\) : \(C(P)\)) , \(v(\pi^*f) = 0\)
\itemdefi
  \(\mathcal{V}P\) : vector bundle := induced by \(\{v \in TP \mid v \text{ is vertical }\}\)
\itemprop
  let \(P \times \mathfrak{g} \to TP\) := \((p , V) \mapsto \frac{d}{dt} \left|_{t = 0} p \cdots (exp t V) \right.\) \\
  then this is vector bundle hom and isomophism on \(\mathcal{V}P\)
\itemdefi
  for (\(V\) : \(\mathfrak{g}\)) ,\\
  \(\tilde{V}\) : \(\mathcal{X}(M)\) :=
  \begin{indentblock}
    for (\(p\) : \(P\)) ,\\
    let s : \(V_p P\) := corresponding to (\((p , V)\)) on \(P \times \mathfrak{g} \cong VP\) 
  \end{indentblock} 
\end{Definition}

\begin{Definition}
\itemnote
  リー微分を定義する
\itemwhen \(M\) : manifold
\itemdefi
  tensor bundle := vector bundle such that constructed from \(P_{GL}(TM) \times_{\rho} V\)
\itemdefi
  for (\(X\) : \(\mathcal{X}(M)\)) (\(s\) : \(\Gamma(P_{GL}(TM) \times_{\rho} V)\)) ,\\
  let \(\phi_t\) := one parameter group of \(X\) \\
  \(L(X)s\) : \(\Gamma(P_{GL}(TM) \times_{\rho} V)\) := \(\frac{d}{dt} \left|_{t=0} \phi_t \cdot s \right.\)
\end{Definition}

\begin{Theorem}
\itemnote
  リー微分、微分、縮約に関する性質
\itemprop
  \(L(X)d = dL(X)\)
\itemprop
  \(L(X) = d \iota(X) + \iota(X) d\)
\end{Theorem}

\begin{Definition}
\itemnote
  切断の対応を与える
\itemwhen \((P , M , \pi)\) : principal \(G\) bundle
\itemwhen \((\rho , E)\) : representation of \(G\)
\itemdefi
  \(C(P , E)^G\) : \(C(M)\) module := \(\{f : P \to E \mid s(p \cdot g) = \rho(g^{-1}) s(p)\}\)
\itemprop
  let \(C(P , E)^G \to \Gamma(M , P \times_G E)\) :=
  for \(f\) (\(x\) : \(M\)) ,\\
  take \(p\) : \(P_x\) ,\\
  return \([p , f(p)]\)
  then this is isomorphism
\itemprop
  \((\tilde{V} \cdot f)(x) + \rho(V)s(x) = 0\)
\end{Definition}

\begin{Definition}
\itemnote
  微分形式の対応を与える。
\itemdefi
  define \(G\) action on \(\Omega(P , E)\) := by formula \(g \cdot (\alpha \otimes e) = g \cdot \alpha \otimes g \cdot e\)
\itemdefi
  \(\Omega(P , E)^G\) := \(\{\alpha : \Omega(P , E) \mid g \cdot \alpha = \alpha\}\)
\itemdefi
  for (\(\alpha\) : \(\Omega(P , E)^G\)) ,\\
  \(L(X)\alpha + \rho(X)\alpha = 0\)
\itemdefi
  for (\(\alpha\) : \(\Omega(P , E)\)) ,\\
  \(\alpha\) is horizontal \(\iff\) \(\forall\) (\(X\) : vertical vector field on \(P\)) , \(\iota(X)\alpha = 0\)
\itemdefi
  for (\(\alpha\) : \(\Omega(P , E)\)) ,\\
  \(\alpha\) is basic differential form \(\iff\) \(\alpha\) is \(G\) invariant and horizontal
\itemdefi
  for (\(\alpha\) : \(\Omega^i(P , E)_{bas}\) ,\\
  define \(\alpha_M\) : \(\Omega^i(M , P \times_G E)\) :=
  \begin{indentblock}
    for (\(x\) : \(M\)) (\(X_1 , \ldots , X_k\) : \(T_x M\)) ,\\
    take (\(p\) : \(P_x\)) (\(\tilde{X_i}\) : \(T_p P\)) such that \(\pi_* \tilde{X_i} = X_i\) ,\\
    return \([p , \alpha_p(X_1 , \ldots X_q)]\)
  \end{indentblock}
\itemprop
  (\(\alpha \mapsto \alpha_M\) : \(\Omega^i(P , E)_{bas} \to \Omega^i(M , P \times_G E)\)) is isomorphism
\end{Definition}