\begin{Theorem}
\itemwhen
  \Fix \(p\) : symbol \\
  \Let \(P\) := pseudodifferential operator defined by \(p\)
\itemprop
  \IfHold \(p(x_0 , \xi) = 0\) forall \(\xi\) \\
  \Then \((Pu)(x_0) = 0\) forall \(u\)
\itemprop
  \IfHold \(p(x_0 , \xi_0) \not = 0\) \\
  \Then there exists \(u\) : \(C^\infty_0\) such that \((Pu)(x_0) \not = 0\)
\itemprop
  \Then for any \(K\) : compact subset which containing \(x_0\) as interior subset , above statement still hold if we restrict \(u\) so that has support in \(K\)
\end{Theorem}

\begin{Proof}
\itemprof
  計算すると確かにそうなる。
\itemprof
  \(C\) := \(e^{i \langle x_0 , \xi_0 \rangle} p(x_0 , \xi_0)\) とすると \(C \not = 0\) である。
  このとき、もし \(u\) : \(C^\infty_0\) であって、 \(\lvert C \int \hat{u}(\xi) d\xi \rvert > \int \lvert e^{i \langle x_0 , \xi \rangle} p(x_0 , \xi) - e^{i \langle x_0 , \xi_0 \rangle} p(x_0 , \xi_0) \rvert \lvert \hat{u}(\xi) \rvert d\xi\) を満たすものがあれば、 \(\lvert (Pu)(\xi_0) - C \int \hat{u}(\xi) \rvert \leq \int \lvert e^{i \langle x_0 , \xi \rangle} p(x_0 , \xi) - e^{i \langle x_0 , \xi_0 \rangle} p(x_0 , \xi_0) \rvert \lvert \hat{u}(\xi) \rvert d\xi\) より示すべき命題が示せる。
  \(C \mathbf{e}_i \not = 0\) なる \(\mathbf{e}_i\) をとればある \(f : \mathbb{R}^n \to \mathbb{R}_{>0}\) が存在して \(\hat{u}(\xi) = f(\xi) \mathbf{e}_i\) となるような \(u\) のみを考え
  \[
    \lvert C \int \hat{u}(\xi) d\xi \rvert = \lvert C \mathbf{e}_i \rvert \int f(\xi) d\xi > \int \lvert e^{i \langle x_0 , \xi \rangle} p(x_0 , \xi) - e^{i \langle x_0 , \xi_0 \rangle} p(x_0 , \xi_0) \rvert f(\xi) d\xi
  \]
  を満たす support in \(K\) な \(f\) の存在に帰着する。
  \(\lvert C \mathbf{e}_i \rvert > \epsilon > 0\) と \(r > 0\) であって \(\lvert \xi - \xi_0 \rvert < r\) なら \(\lvert e^{i \langle x_0 , \xi \rangle} p(x_0 , \xi) - e^{i \langle x_0 , \xi_0 \rangle} p(x_0 , \xi_0) \rvert < \epsilon\) を満たすものをとる。
  \(f(\xi)\) であって \(\lvert \xi - \xi_0 \rvert \geq 0\) なら \(f = 0\) となるような \(f \not = 0\) が存在すれば
  \begin{align*}
    &\int \lvert e^{i \langle x_0 , \xi \rangle} p(x_0 , \xi) - e^{i \langle x_0 , \xi_0 \rangle} p(x_0 , \xi_0) \rvert f(\xi) d\xi
    = \int_{\lvert \xi - \xi_0 \rvert < r} \lvert e^{i \langle x_0 , \xi \rangle} p(x_0 , \xi) - e^{i \langle x_0 , \xi_0 \rangle} p(x_0 , \xi_0) \rvert f(\xi) d\xi \\
    &\leq \epsilon \int_{\lvert \xi - \xi_0 \rvert < r} f(\xi) d\xi = \epsilon \int f(\xi) d\xi < \lvert K \mathbf{e}_i \rvert \int f(\xi) d\xi
  \end{align*}
  よりよいがこのような \(f\) は明らかに存在するので良い。
\itemprof
  \(f\) のとり方として support が \(K\) に含まれるようにとればよいが、これは \(x_0\) を内点に含むという条件から確かにとれる。
\end{Proof}

\begin{Theorem}
\itemprop
  \Then symbol of pseudodifferential operator is unique
\end{Theorem}

\begin{Definition}
\itemwhen
  \Fix \(U\) : subset of \(\mathbb{R}^n\)
\itemdefi
  \Define \(C^\infty_0(U)\) := \(\{u \in C^\infty_0 \mid \text{supp} \, u \subset U\}\)
\itemdefi
  \For \(P\) : pseudodifferential operator \\
  \Define support of \(P\) : subset of \(\mathbb{R}^n\) := \(x\)-support of symbol of \(P\)
\itemdefi
  \Define symbol support in \(U\) := symbol (\(p\)) such that \(\text{supp} \, p \subset U\)
\itemprop
  \For \(p\) : symbol support in \(U\) \\
  \Let \(P\) := pseudodifferential operator defined by \(p\) \\
  \Then \(P(\mathscr{S}) \subset C^\infty_0(U)\)
\itemprof
  略
\itemprop
  \Define pseudodifferential operator on \(U\) := \(\restr{P}{C^\infty_0(U)}\)
\end{Definition}

\begin{Definition}

\itemprop
  \Then forall \(u\) : \(\mathscr{S}\) , \(\text{supp} \, Pu \subset \text{supp} \, P\) \\
  \Then \(\cup_{u : C^\infty_0} \text{supp} \, Pu = \text{supp} \, P\)
\itemprof
  確かに成り立つ。
\itemdefi
  \For \(P\) : \(\Psi DO_m\) , \(K\) : compact set \\
  \Define \(P\) is said to have support in \(K\) := \\
  for \(u\) : \(C^\infty_0\) , \(\text{supp} \, Pu \subset K\) and \(\text{supp} \, u \cap K = \emptyset\) implies \(P u = 0\)
\itemprop
  \Then first statement is equivalent to \(K\) containing support of \(P\)
\itemprof
  確かに成り立つ。
\itemdefi
  \For \(p\) : symbol of order \(d\) with support in \(K\) \\
  \Define pseudodifferential of order \(m\) on \(K\) := \\
  restriction of pseudodifferential operator defined by \(p\) to the \(C^\infty_0(K) \to C^\infty_0(K)\)
\itemdefi
  \Define \(\Psi DO_{K,m}\) := linear space of such operators
\itemnote
  この定義により pseudodifferential operator on \(K\) の元を与えるには \(C^\infty_0(K)\) の元に対してのみ定義されていればよい。
\end{Definition}

\begin{Definition}
\itemdefi
  \For \(P , Q\) : \(\Psi DO_{K,m}\) \\
  \Define \(Q\) is formal adjoint of \(P\) := \\
  forall \(u,v\) : \(\mathscr{S}\) with support in \(K\) , \((Pu , v)_{L^2} = (u , Q v)_{L^2}\)
\end{Definition}

\begin{Theorem}
\itemprop
  \Then formal adjoint is unique
\end{Theorem}

\begin{Proof}
\itemprof
  \(P\) : \(\Psi DO_{K,m}\) であって任意の \(u,v\) : \(\mathscr{S}\) with support in \(K\) なものに対して \((Pu , v) = 0\) を満たすものが \(P = 0\) のみであることを示せばよい。
  条件より \(u\) : \(\mathscr{S}\) with support in \(K\) に対しては \(P u = 0\) とわかる。
  \(p(x,\xi) \not = 0\) を満たすような \(x\) があったとすると、 \(p(x,\xi)\) の連続性から \(x\) として \(K\) の内点か \(K\) の補集合の(内)点であるとしてよく、この二つの場合で矛盾を示す。
  前者の場合は \(u\) をうまくとると \(Pu \not = 0\) となるようにとれるがこれは \(u\) : \(\mathscr{S}\) with support in \(K\) に対しては \(P u = 0\) となることと矛盾する。
  後者の場合は \(u\) をうまくとることで \(\text{supp} \, u \cap K = \phi\) かつ \(P u \not = 0\) が満たされ矛盾する。
\end{Proof}

\begin{Theorem}
\itemprop
  \For \(P\) : \(\Psi DO_{K,m}\) with symbol \(p\) \\
  \Then there exists \(Q\) : \(\Psi DO_{K,m}\) with symbol \(q\) such that \\
  \(Q\) is formal adjoint of \(P\) and \(q \sim \sum_{\alpha} \frac{i ^ {\lvert \alpha \rvert}}{\alpha !} D^\alpha_{\xi} D^\alpha_x \bar{p}^t\)
\itemnote
  この構成方法を \(P^* , p^*\) のように書くこととする。
\end{Theorem}

\begin{Proof}
\itemprof
  \(\phi(x)\) : \(C^\infty_0\) であって \(K\) 上 \(\phi = 1\) となるものを固定する。
  任意の \(u\) : \(\mathscr{S}\) で support が \(K\) に含まれるものに対して、 \(\phi u = u\) である。
  特に、 \(u,v\) : \(\mathscr{S}\) で support が \(K\) に含まれるものに対して
  \begin{align*}
    (Pu , v)_{L^2}
    &= \int \langle \int e^{i \langle x , \xi \rangle} p(x , \xi) \mathscr{F}[u](\xi) d\xi , v(x) \rangle dx \\
    &= \int \int \int e^{i \langle x - y , \xi \rangle} \langle p(x , \xi) u(y) , v(x) \rangle dy d\xi dx \\
    &= \int \int \int \langle u(y) , e^{- i \langle x - y , \xi \rangle} \bar{p}^t(x,\xi) v(x) \rangle dy d\xi dx \\
    &= \langle u , x \mapsto \int e^{i \langle x - y , \xi \rangle} \bar{p}^t(y,\xi) v(y) dy d\xi \rangle \\
    &= \langle u , x \mapsto \int e^{i \langle x - y , \xi \rangle} \phi(x) \bar{p}^t(y,\xi) v(y) dy d\xi \rangle
    =: \langle u , Q v \rangle
  \end{align*}
  最後の行の変形が正当化されるのは、 \(u\) が support が \(K\) に含まれるような関数であることから、 \(\langle u , f \rangle = \langle u , \phi f \rangle\) となるためである。
  ( \(f\) 側が support in \(K\) となるかはわからない。)
  上で定義された operator \(Q\) は \(a(x,y,\xi) = \phi(x) \bar{p}^t(y,\xi)\) と置いたときの WorkHorse theorem により得られる operator であり、定義上 \(P\) の formal adjoint になることがわかる。
  従って、 WorkHorse theorem の適用のための前提を満たすこと、 theorem の帰結としてえられる formal development が求める formal development になること、 \(Q\) が support in \(K\) であることを示す。
\itemthen
  \(x,y\)-support compact であることが \(\phi(y)\) の y-support コンパクト性と \(p(x,\xi)\) の \(x\)-support コンパクト性よりわかる。
  また、微分に関する条件も、 \(\lvert D^\alpha_x D^\beta_y D^\gamma_\xi [\phi(y) \bar{p}^t(x,\xi) \rvert] = \lvert D^\alpha_x \phi(x) \rvert \lvert D^\alpha_y D^\gamma_\xi \bar{p}^t(y,\xi) \rvert \leq \text{const} \lvert D^\alpha_y D^\gamma_\xi \bar{p}^t(y,\xi) \rvert\) となり計算によりわかる。
  これによりえられる operator \(Q\) の symbol \(q\) は次の formal development を持つ。
  \begin{align*}
    \sum_{\alpha} \frac{i^{\lvert \alpha \rvert}}{\alpha !} D^\alpha_\xi D^\alpha_y [\phi(x) \bar{p}^t(y,\xi)](x,x,\xi)
    &= \sum_{\alpha} \frac{i^{\lvert \alpha \rvert}}{\alpha !} \phi(x) (D^\alpha_x D^\alpha_\xi \bar{p}^t)(x , \xi) \\
    &= \sum_{\alpha} \frac{i^{\lvert \alpha \rvert}}{\alpha !} (D^\alpha_x D^\alpha_\xi \bar{p}^t)(x , \xi)
  \end{align*}
  ただし、 \(p\) の \(x\)-support が \(K\) に含まれていることから \(\phi \bar{p}^t = \bar{p}^t\) を使った。
  \(Q\) が support in \(K\) であることを示す。
  \(u\) : \(C^\infty_0\) と \(x\) : not in \(K\) に対して \((Q u)(x) = \phi(x) \cdots = 0\) であるから \(\text{supp} (Pu) \subset K\) である。
  また、 \(\text{supp} u \cap K = \emptyset\) なら \(\phi(x)\bar{p}^t(y,\xi) u(y)\) は \(y \in K\) かどうかで場合分けをすると任意の \(y\) で \(0\) となることがわかるから \(Qu = 0\) となる。
\end{Proof}

\begin{Theorem}
\itemprop
  \For \(p,q\) : symbol \\
  \Let \(P , Q\) : \(\Psi DO_{K,m} , \Psi DO_{K,l}\) := define by \(p,q\) \\
  \Then \(P \circ Q\) is pseudo differential operator of order \(m + l\) with symbol 
  which has formal development \(\sum_{\alpha} \frac{i^{\lvert \alpha \rvert}}{\alpha !} (D^\alpha_{\xi} p) (D^\alpha_{x} q)\)
\end{Theorem}

\begin{Proof}
\itemprof
  \(Q = (Q^*)^*\) であるから
  \begin{align*}
    (Qu)(x)
    &= \int\int e^{i \langle x-y ,\xi \rangle} \bar{q^*}^t(y,\xi) u(y) dy d\xi \\
    &= \int e^{i\langle x,\xi \rangle} \left[\int e^{-i\langle y,\xi \rangle} \bar{q^*}^t(y,\xi) u(y) dy \right] d\xi \\
  \end{align*}
  のため Fourier inversion により
  \[
    \widehat{Qu}(\xi) = \int e^{-i \langle y,\xi \rangle} \bar{q^*}^t(y,\xi) u(y) dy
  \]
  とわかる。
  したがって
  \begin{align*}
    (PQu)(x)
    &= \int e^{i \langle x ,\xi \rangle} p(x,\xi) \widehat{Qu}(\xi) d\xi
      = \int e^{i \langle x ,\xi \rangle} p(x,\xi) \int e^{-i \langle y,\xi \rangle} \bar{q^*}^t(y,\xi) u(y) dy d\xi \\
    &= \int e^{i \langle x-y ,\xi \rangle} p(x,\xi) \bar{q^*}^t(y,\xi) dy d\xi
  \end{align*}
  ここで WorkHorse theorem を用いる。
\itemthen
  \(x,y\)-support のコンパクト性は確かに成り立つ。
  微分に関する条件も確かに計算により order が \(m + l\) となる。
  したがって定理の帰結として \(PQ\) は pseudodifferential operator でありその symbol が次の formal development を持つものとして得られる。
  formal development は各自然数 \(l\) で添え字付けられていた。
  \begin{align*}
    &\sum_{\lvert \alpha \rvert = l} \frac{i^{\lvert \alpha \rvert}}{\alpha !} (D^\alpha_\xi D^\alpha_y p(x,\xi) \bar{p^*}^t(y,\xi))(x,x,\xi) \\
    &= \sum_{\lvert \alpha \rvert = l} \frac{i^{\lvert \alpha \rvert}}{\alpha !}
    \sum_{\alpha_1 + \alpha_2 = \alpha} \frac{\alpha !}{\alpha_1 ! \alpha_2 !} (D^{\alpha_1}_\xi p) (D^{\alpha_2}_\xi D^{\alpha_1 + \alpha_2}_x \bar{p^*}^t(x,\xi)) \\
    &= \sum_{\lvert \alpha_1 \rvert , \lvert \alpha_2 \rvert = l} \frac{i^{\lvert \alpha \rvert }}{\alpha_1 !} \frac{i^{\lvert \alpha_2 \rvert}}{\alpha_2 !} (D^{\alpha_1}_\xi p) (D^{\alpha_2}_\xi D^{\alpha_1 + \alpha_2}_x \bar{p^*}^t(x,\xi)) \\
    &= \sum_{\lvert \beta \rvert = l} \frac{i^{\lvert \beta \rvert}}{\beta !} (D^{\beta}_\xi p) (D^{\beta}_x 
    \left[\sum_{\lvert \gamma \rvert = l} \frac{i^{\lvert \gamma \rvert}}{\gamma !} D^\gamma_\xi D^\gamma_x \bar{q^*}^t \right]) \\
    &\sim \sum_{\lvert \beta \rvert = l} \frac{i^{\lvert \beta \rvert}}{\beta !} (D^{\beta}_\xi p) (D^{\beta}_x q) \\
  \end{align*}
  途中で項の grouping の順番を変えた。
  最後は \(Q =(Q^*)^*\) より \(q\) の formal development が \(\sum_{\gamma} D^\gamma_\xi D^\gamma_x \bar{q^*}^t\) であることよりわかる。
\end{Proof}

\begin{Theorem}
\itemwhen
  \Fix \(U,V\) : open subset , \(\phi\) : \(U \to V\) , \(K\) : compact subset \(\subset U\)
\itemdefi
  \For \(u\) : \(C^\infty_0(\phi(K))\)
  \Define \(\phi_* u\) : \(C^\infty_0(K)\) := \(u \circ \phi\) if \(x \in U\) and \(0\) otherwise
\itemprop
  \For \(P\) : \(\Psi DO_{K,m}\) , \(u\) : \(\mathscr{S}\) , \(C^\infty_0(\phi(K))\) \\
  \Then \((\phi_* P)(u)\) : \(C^\infty_0(K)\) := \((\phi_*)^{-1} P(\phi_* u)\)
\itemprop
  \Define \(\phi_*\) : \(\Psi DO_{K,m} \to \Psi DO_{\phi(K) , m}\) := \(P \mapsto (\phi_* P)\) \\
  \Then \(\phi_*\) is isomorphism
\end{Theorem}

\begin{Proof}
\itemprof
\WIP
\end{Proof}

\begin{Theorem}
\itemprop
  \Then \AIMAI{the symbol of a pseudo differential operator transforms like a function on the cotangent bundle}.  i.e. Fourier transform \(\tilde{x} , \xi\) be considered as standard coordinates ofr \(1\)-forms \(\sum \xi_j d \tilde{x}_j\) 
\end{Theorem}

\begin{Definition}
\itemdefi
  \For \(P\) : \(\Psi DO_m\) with symbol \(p\)
  \Define principal symbol of \(P\) = \([p] \in \text{Sym}^m / \text{Sym}^{m-1}\)
\itemprop
  \Then \(\sigma(P)\) transforms under diffeomorphism like a function on the cotangent bundle of \(\mathbb{R}^n\)
\end{Definition}