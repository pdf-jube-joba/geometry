\begin{Definition}
\itemdefi
  \Define \(k\) : \(C_n \otimes \mathbb{C} \to C_{n+2}\) := \(C_n \otimes \mathbb{C} \cong C_n \otimes C_2^0 \to C_n \hat{\otimes} C_2 \cong C_{n+2}\)
\itemdefi
  \Define determinant : \(\text{Spin}^c(n) \to \text{SO}(2)\) := \(g \otimes \lambda \mapsto \lambda^2\)
\end{Definition}

\begin{Theorem}
\itemprop
  \Then \(k(u \otimes (a + bi)) = u(a + b e_{n+1} e_{n+2})\)
\itemprop
  \Then action of \(\text{Spin}^c\) on \(\mathbb{R}^{n} \times \mathbb{R}^2\) is \(k(g \otimes \lambda)(v,w) = (g v g^{-1} , \lambda^2 w)\)
\itemprop
  \For \(g \otimes \lambda\) : \(\text{Spin}^c(n)\) \\
  \Then \(k(g \otimes \lambda)\) \(\in\) \(\text{Spin}(n+2)\)
\itemprop
  \Then \(\rho \circ k = (g \otimes \lambda \mapsto (\rho(g), \lambda^2))\)
\end{Theorem}

\begin{Proof}
\itemprof
  \(u \times a + bi \mapsto u \times a + b e_1 e_2 \mapsto u (a + b e_{n+1} e_{n+2})\)
\itemprof
  \(k(g \otimes 1)(v,w) = (g) \cdot (v,w) = (g v g^{-1}, w)\) と \(k(1 \otimes \lambda)(v,w) = (v , \lambda w \lambda^{-1})\) が成り立った。
  \(w \in \mathbb{R}^2 \subset \mathbb{R}^{n+2}\) は \(e_{n+1} , e_{n+2}\) で貼られる空間にあるため \(e_{n+1} e_{n+2}\) と anti-commutative であり、 ( \(e_i (e_{n+1} e_{n+2}) = (e_{n+1} e_{n+2}) e_{i}\) が \(i=n+1,n+2\) で成り立つ) \(\lambda w \lambda^{-1} = \lambda w \bar{\lambda} = \lambda^2 w\) となる。
\itemprof
  \(g\) : \(\text{Spin}(n)\) より \(g \otimes \lambda \in C^0\) に入り、 \(k\) が計量を保つので \(\lVert g \otimes \lambda \rVert = \lVert g \rVert \lvert \lambda \rvert = \pm 1\)
\itemprof
  以上の命題から確かにそう表される。
\end{Proof}

\begin{Theorem}
\itemwhen \(n = 8m - 2\)
\itemprop
  \Let \(\Delta\) : \(C_{8m}\)-module := canonical one \\
  \Let \(\Delta\) : \(C_{8m-2} \otimes \mathbb{C}\)-module := \(C_{8m}\)-module \(\Delta\) via \(k\) \\
  \Then \(\Delta\) is canonically a complex vector space
\itemprop
  \Then \(\Delta_c \cong \Delta\) as \(C_{8m-2} \otimes \mathbb{C}\)-module \\
  \Then \(k(\Omega_c) = \Omega\)
\itemprop
  \Then splittings \(\Delta = \Delta^+ \oplus \Delta^-\) and \(\Delta_c = \Delta_c^+ \oplus \Delta_c^-\) agree as \(\text{Spin}^c(n)\)-module
\end{Theorem}

\begin{Proof}
\itemprof
  これは確かに定義より成り立つ
\itemprof
  次元を考えることでただ一つ表現が存在するはずとわかるからよい。
\itemprof
  \(k(\Omega_c) = k(i e_1 \cdots e_n) = e_1 \cdots e_n (e_{n+1} e_{n+2}) = \Omega\) より
\itemprof
  \(\Omega_c\) や \(\Omega\) の \(\pm 1\) 固有空間として splitting が与えられていたため。
\end{Proof}

\begin{Theorem}
\itemwhen
  \(n = 8 m - 2\)
\itemprop
  \For \((p : V \to X)\) : \(\text{Spin}^c\) vector bundle , \(P_c\) := principal bundle of this\\
  \Then \(\text{Det}(V) \cong P_c \times_{\text{Spin}^c} \mathbb{R}^2\)
\itemprop
  \Then \(P_c \times_{\text{Spin}^c(n)} \text{Spin}(n+2)\) is a canonical Spin structure on \(V \oplus \text{Det}(V) \)
\itemprop
  \Then \(r \mu_c(V)\) : \(KO(V)\) is presented by the complex \(0 \to p^* {S^+} \overset{\sigma}{\to} p^*{S^-} \to 0\) \\ where for \(v\) : \(V\) , \(\sigma_v\) := multiplication
\end{Theorem}