\documentclass[dvipdfmx]{jsarticle}
% パッケージ
\usepackage{amsthm}
\usepackage{amsmath,amssymb}
\usepackage{color}
\usepackage{tikz}

% 定理環境
%% 本体
\theoremstyle{definition}
\newtheorem*{tDefinition}{定義}
\newtheorem*{tTheorem}{定理}
\newtheorem*{tProof}{証明}
\newtheorem*{tNotation}{記法}
\newtheorem*{tRemark}{注意}
\newtheorem*{tWhen}{設定}

\newenvironment{Mini}{
  \begin{minipage}[t]{0.9\hsize}
  \setlength{\parindent}{12pt}
  \begin{itemize}
  \setlength{\labelsep}{10pt}
}{
  \end{itemize}
  \vspace{5pt}
  \end{minipage}
}

\newenvironment{Definition}{
  \begin{tDefinition}
  \begin{Mini}
}{
  \end{Mini}
  \end{tDefinition}
}

\newenvironment{Theorem}{
  \begin{tTheorem}
  \begin{Mini}
}{
  \end{Mini}
  \end{tTheorem}
}

\newenvironment{Proof}{
  \begin{tProof}
  \begin{Mini}
      }{
  \end{Mini}
  \end{tProof}
}

\newenvironment{When}{
  \begin{tWhen}
  \begin{Mini}
}{
  \end{Mini}
  \end{tWhen}
}

%% マーク
\newcommand{\itemwhen}{\item[\(\bigcirc\)]}
\newcommand{\itemnote}{\item[!]}

\newcommand{\itemdefi}{\item[\(\square\)]}
\newcommand{\itemprop}{\item[\(\vartriangleright\)]}
\newcommand{\itemand}{\item[\(-\)]}

\newcommand{\itemprof}{\item[\(\because\)]}
\newcommand{\itemthen}{\item[\(\rightsquigarrow\)]}

\newcommand{\itemenum}{\item[\(\bullet\)]}
\newcommand{\itemwith}{\item[\(-\)]}

% 記述関係
\newenvironment{indentblock}{
  \\
  \hspace*{5mm}
  \begin{minipage}{0.8\textwidth}
}{
  \end{minipage}
  \\
}

%コマンド関連

\newcommand{\WIP}{\textcolor{red}{工事中}}
\newcommand{\SORRY}{\textcolor{red}{わかりませんでした}}
\newcommand{\ADMIT}{\textcolor{blue}{認めます}}
\newcommand{\AIMAI}[1]{\textit{#1}}

\newcommand{\Ktheory}[1]{K_{\mathbb{K}}(#1)}
\newcommand{\KtheoryReduced}[1]{\tilde{K}_{\mathbb{K}}(#1)}
\newcommand{\KtheoryRel}[2]{K_{\mathbb{K}}(#1, #2)}
\newcommand{\Kcohomology}[3]{K_{\mathbb{K}}^{#1}(#2, #3)}
\newcommand{\KcohomologyReduced}[2]{\tilde{K}_{\mathbb{K}}^{#1}(#2)}
\newcommand{\KtheoryCpx}[1]{K_{\mathbb{K},cpt}(#1)}
\newcommand{\KtheoryCpxRel}[2]{K_{\mathbb{K},cpt}(#1, #2)}

\begin{document}

\section{Thom 同型の同型性について}
コンパクト空間の場合の \(K\)-理論に話を移し、 Bott 周期性との関連により示す。

\((\mathbb{K} , \mathbb{G} , k , V) \leftarrow (\mathbb{R}, \text{Spin} , 8 , \mathbb{R}^8), (\mathbb{C}, \text{Spin}^c , 2 , \mathbb{R}^2)\) とする。
ori.metrized.vec.bundle with \(\mathbb{G}\).st を \(\mathbb{G}\).st と書く。
また通常のコンパクト空間に対してのみ定義された \(K\) 理論と区別する必要がある(何がどっちで成り立っているかわかりにくい)ため、複体で定義したものを \(\KtheoryCpx{X}\) と書く。
pair of space で locally compact space と closed subset の空間対とする。
次のことを仮定した場合に示す。
\begin{itemize}
  \itemwith \(\eta := [0 \to \mathbb{R}^{k} \times \Delta^+ \to \mathbb{R}^{k} \times \Delta^- \to 0]\) : \(\KtheoryCpx{\mathbb{R}^k} \cong \KtheoryReduced{S^k} = \KcohomologyReduced{-k}{S^0}\)
  \itemwith \(\beta(x) = x \cdot \eta\) : \(\KcohomologyReduced{*}{X} \to \KcohomologyReduced{*-k}{X}\) := \(\beta\) is isomorphism
\end{itemize}

\begin{Definition}[Thom 複体と Thom 写像の定義]
\itemwhen
  \Fix \(\xi = (q : W \to X)\) : \(\mathbb{G}\).st where \(X\) : compact
\itemdefi
  \Define \(\mu(\xi)\) : cpt.supp.cpx over \(W\) := \\
  \(0 \to q^*S^+ \overset{\omega}{\to}q^*S^- \to 0\) \\
  \Define Thom class : \(\KtheoryCpx{W}\) := represented by \(\mu(\xi)\) \\
  \Define Thom homomorphism of \(\xi\) \ldots \(T_{\xi}\) : \(\KtheoryCpx{X} \to \KtheoryCpx{W}\) := \([x] \mapsto [q^*x \otimes \mu(\xi)]\)
\end{Definition}

%\begin{Theorem}
%\itemprop
%  \For \(W_1 \overset{q_1}{\to} X , W_2 \overset{q_2}{\to} X\) : \(\mathbb{G}\).st. , \(f\) : isomorphism of \(\mathbb{G}\).st. \(W_1 \to W_2\) \\
%  \Then \(T_{W_1 \overset{q_1}{\to} X} = f^* \circ T_{W_2 \overset{q_2}{\to} X}\) : \(\KtheoryCpx{X} \to \KtheoryCpx{W_1}\)
%\itemprop
%  \For \(\xi = (W \overset{q}{\to} X)\) : \(\mathbb{G}\).st , \(f\) : proper map of \(Y \to X\) \\
%  \Then \(\mu(f^* \xi) \cong f^* \mu(\xi)\) : cpt.supp.cpx over \(\text{Total}(f^* \xi)\)
%\itemprop
%  \For \(\xi_1 , \xi_2\) : \(\mathbb{G}\).st \\
%  \Then \([\mu(\xi_1 \times \xi_2)] = [\text{Proj}^* \mu(\xi_1) \otimes \text{Proj}^* \mu(\xi_2)]\) : \(\KtheoryCpx{\text{Total}(\xi_1) \times \text{Total}(\xi_2)}\)
%\end{Theorem}

%\begin{Proof}
%\itemprof
%  Thom 複体が誘導されるものと可換であることからよい。
%\itemprof
%  Thom 複体が誘導されるものと可換であることからよい。
%\itemprof
%  \WIP
%\end{Proof}

\begin{Theorem}[Thom 写像を移し替える]
%\itemprop
%  \Let \(\phi\) : \(W \to D(W) - S(W)\) := \(v \mapsto v / (1 + \lVert v \rVert)\) \\
%  \Then \(h\) : \(\KtheoryCpxRel{D(W)}{S(W)} \to \KtheoryCpx{W}\) := \([x] \mapsto [\phi^* (\restr{x}{D(W) - S(W)})]\) is isomorphism
%\itemprop
%  \Then \(h^{-1}T_{\xi} x = [q^* x] \cdot [\restr{\mu(W)}{(D(W), S(W))}]\)
\itemdefi
  \Let \(\tilde{\mu}(\xi)\) : \(\KcohomologyReduced{n}{\text{Total}(\xi)}\) := defined by \(\mu(\xi) \in \KtheoryCpx{W} \cong \KtheoryReduced{\text{Total}(\xi)} \cong \KcohomologyReduced{n}{\text{Total}(\xi)}\)
\itemprop
  \Let \(\tilde{T}_{\xi}\) : \(\Kcohomology{0}{X}{\emptyset} \to \KcohomologyReduced{n}{\text{Total}(\xi)}\) := induced by \\
  \(\KcohomologyReduced{0}{X} \cong \KtheoryCpx{X} \overset{T_{\xi}}{\to} \KtheoryCpx{\text{Total}(\xi)} \cong \KtheoryReduced{\text{Thom}(\xi)} \cong \KcohomologyReduced{n}{\text{Thom}(\xi)}\) \\
  \Then \(\tilde{T}_{\xi} = x \mapsto x \cdot \tilde{\mu}(\xi)\)
\itemprop
  \Then \(\tilde{\mu}(\xi)\) is \(K_{\mathbb{K}}\)-orientation
\itemprop
  \Then \(\tilde{T}_{\xi}\) is isomorphism
\end{Theorem}

\begin{Proof}
%\itemprof
%  \(\KtheoryCpxRel{W^+}{\infty}\) との同型を経由すれば、コンパクト空間の対の間の写像であり通常の \(K\) 理論での同型からわかる。
%\itemprof
%  \(\psi\) : \(W \to W\) := \(v \mapsto v / (1 + \lVert w \rVert)\) として
%  \(\psi^* \mu(W)\) と \(\mu(W)\) が同じ \(K(W)\) の元を定めていることを示せばよいことがわかる。
%  \(f\) : \(q^*(S^+) \to \psi^* q^*(S^+)\) と \(g\) : \(q^*(S^-) \to \psi^* q^*(S^-)\) を \((w,v) \mapsto (w / (1 + \lVert w \rVert), v)\) と定値すると、ベクトル束の同型であり次の図式が可換になるからよい。
%  \[\begin{matrix}
%    0 & \to & q^*(S^+) & \to & q^*(S^-) & \to & 0 \\
%     & & \downarrow f & & \downarrow g & & \\
%    0 & \to & \psi^*q^*(S^+) & \to & \psi^*q^*(S^-) & \to & 0
%  \end{matrix}\]
%\itemprof
%  これは積を持つ一般コホモロジーの一般論からわかる。
\itemprof
  これは定義による。
\itemprof
  まず状況を整理すると
  \begin{itemize}
    \item \(\xi = (q : W \to X)\) : vector bundle
    \item \(n = ik = \text{dim}(\xi)\)
    \item (\(P\): \(\text{Spin}(n)\)-bundle over \(X\), \(\phi\): \(P \times_{\text{Spin}(n)} \mathbb{R}^n \cong \xi\)) : Spin structure over \(\xi\)
    \item \(x\) : \(X\)
    \item \(i_x\) : \(\restr{\xi}{x} \to \xi\) := injection
  \end{itemize}
  という状況で
  \({i_x}^*\mu(\xi)\) が \(\mathbb{R}^n \cong \restr{\xi}{x}\) のもとで \(1 \cdot \eta \cdot \eta \cdots \eta\) : \(\tilde{K}^{0 - ik}(\text{point}) \cong K_{\text{cpt}}(\mathbb{R}^{ik})\)
  に等しいことを示せばよい。
  \({i_x}^*\mu(\xi)\) は \(0 \to \mathbb{R}^{ik} \times \Delta^+ \to \mathbb{R}^{ik} \times \Delta^- \to 0\) にあたることが次のようにしてわかる。
  \begin{align*}
    {i_x}^*[q^* (P \times_{\text{Spin}(n)} \Delta^+) &\to q^* (P \times_{\text{Spin}(n)} \Delta^-)] \\
    [\restr{\xi}{x} \times_{\{x\}} ((P \times_{\text{Spin}(n)} \Delta^+)) &\to \restr{\xi}{x} \times_{\{x\}} ((P \times_{\text{Spin}(n)} \Delta^+))] \\
    [(\restr{P}{x} \times_{\text{Spin}(n)} \mathbb{R}^n) \times (\restr{P}{x} \times_{\text{Spin}(n)} \Delta^+) &\to (\restr{P}{x} \times_{\text{Spin}(n)} \mathbb{R}^n) \times (\restr{P}{x} \times_{\text{Spin}(n)} \Delta^-)] \\
    [\mathbb{R}^{ik} \times \Delta^+ &\to \mathbb{R}^{ik} \times \Delta^-]
  \end{align*}
  写像が対応するかどうかについては、 \(q^* W \times q^* S^+ \to q^*S^-\) の写像の定義による。
  \(\mu(0 \to \mathbb{R}^{ik} \times \Delta^+ \to \mathbb{R}^{ik} \times \Delta^- \to 0)\) と \(\beta^i\) : \(\KtheoryReduced{S^0} \to \KtheoryReduced{S^{ik}}\) により \(\mu(\cdots) = \beta^i(1) \in \KtheoryReduced{S^0}\) であることを示せばよい。
  これは
  \[[\mathbb{R}^{ik} \times \Delta^+ \to \mathbb{R}^{ik} \times \Delta^-] = \hat{\otimes}^i [\mathbb{R}^{k} \times \Delta^+ \to \mathbb{R}^{k} \times \Delta^-]\] を示せば、 \(\mu(\cdots) = \mu(\cdots)^{i} = (1 \cdot \beta(1))^i = \eta^i = (1 \cdot \eta) \cdot \eta \cdots \eta = \beta(\cdots \beta(\beta(1)))\) よりよい。
\itemthen
  \([\mathbb{R}^{ik} \times \Delta^{+} \to \mathbb{R}^{ik} \times \Delta^-] \otimes [\mathbb{R}^{k} \times \Delta^+ \to \mathbb{R}^{k} \times \Delta^-] = [\mathbb{R}^{(i+1)k} \times \Delta^+ \to \mathbb{R}^{(i+1)k} \times \Delta^-]\) を示す。
  状況を整理すると、
  \begin{itemize}
    \item \(\Delta\) := \(\mathbb{R}^{16}\)
    \item identification of \(\text{Cl}_{8} \cong \text{End}(\Delta)\) := \AIMAI{by natural definition}
    \item volume element \(\Omega\) : \(\text{Cl}_{8}\) := \(e_{1} \cdots e_{8k}\)
    \item \(\Delta^{\pm}\) : subspace of \(\Delta\) := \(\pm 1\) eigenspace of \(\Omega \cdot \place\)
  \end{itemize}
  としたうえで \(\Delta_{8k}\) が次元のみ指定されていたため次のように定める。
  \begin{itemize}
    \item \(\Delta_{k}\) : vector space \(\text{dim}=2^{4k}\) := inductive definition by \(\Delta_{k+1}\) := \(\Delta_{k} \otimes \Delta\)
    \item isomorphism of \(\text{Cl}_{n+8} \to \text{Cl}_{n} \otimes \text{Cl}_{8}\) := \(e_{i} \mapsto e_{i} \otimes \Omega (1 \leq i \leq 8k), 1 \otimes e_{i-8k} (8k+1 \leq i 8(k+1))\)
    \item identification of \(\text{Cl}_{8k} \cong \text{End}(\Delta_{k})\) := inductive definition by \\
    \(\text{Cl}_{8(k+1)} \cong \text{Cl}_{8k} \otimes \text{Cl}_{8} \cong \text{End}(\Delta_{k}) \otimes \text{End}(\Delta) \cong \text{End}(\Delta_{k} \otimes \Delta)\)
    \item volume element \(\Omega_{k}\) : \(\text{Cl}_{8k}\) := \(e_{1} \cdots e_{8k}\)
    \item \({\Delta_{k}}^{\pm}\) : subspace of \(\Delta_{k}\) := \(\pm 1\) eigenspace of \(\Omega \cdot \place\)
  \end{itemize}
  このとき \((\Delta_{k+1})^{+} = \Delta_{k}^{+} \otimes \Delta_{k}^{+} \oplus \Delta_{k}^{-} \otimes \Delta_{k}^{-}, (\Delta_{k+1})^{-} = \Delta_{k}^{-} \otimes \Delta_{k}^{+} \oplus \Delta_{k}^{+} \otimes \Delta_{k}^{-}\) であることが、 \(\text{Cl}_{n+8} \cong \text{Cl}_{n} \cong \text{Cl}_{8}\) のもとで \(\Omega_{k+1} \mapsto \Omega_{k} \otimes \Omega\) と対応しそれぞれ次元を考えることでわかる。
  二つの複体を \((t,s) \in \mathbb{R}^{8k} \times \mathbb{R}^{8}\) のファイバー上で比較する。
  前者は
  \[
    0 \to
    \Delta_{k}^{+} \otimes \Delta_{k}^{+}
    \overset{\begin{pmatrix} t \otimes 1 \\ 1 \otimes s \end{pmatrix}}{\to}
    \Delta_{k}^{-} \otimes \Delta_{k}^{+} \oplus \Delta_{k}^{+} \otimes \Delta_{k}^{-}
    \overset{\begin{pmatrix} -1 \otimes s, t \otimes 1 \end{pmatrix}}{\to}
    \Delta_{k}^{-} \otimes \Delta_{k}^{-}
    \to 0
  \]
  であり、後者は \((t,s) \in \mathbb{R}^{8k} \times \mathbb{R}^8\) が \(\text{Cl}_{8k} \otimes \text{Cl}_{8}\) において \(t \otimes \Omega + 1 \otimes s\) に対応することから、
  \[
    0 \to
    \Delta_{k}^{+} \otimes \Delta_{k}^{+} \oplus \Delta_{k}^{-} \otimes \Delta_{k}^{-}
    \overset{t \otimes \Omega + 1 \otimes s}{\to}
    \Delta_{k}^{-} \otimes \Delta_{k}^{+} \oplus \Delta_{k}^{+} \otimes \Delta_{k}^{-}
    \to 0
  \]
  となる。
  今、 \(0 \to E \overset{f}{\to} F \overset{g}{\to} G \to 0 \simeq 0 \to E \oplus G \overset{(f,g^{\dagger})}{\to} F \to 0\) を認めれば、 \(\Delta^{+} \overset{t \cdot \place}{\to} \Delta^{-}\) の随伴が \(\langle t w_1 , w_2 \rangle = \langle w_1 , -t w_2 \rangle\) より \(t^{\dagger} = -t\) となることと合わせて、前者の複体が次と同値である。
  \[
    0 \to
    \Delta_{k}^{+} \otimes \Delta_{k}^{+} \oplus \Delta_{k}^{-} \otimes \Delta_{k}^{-}
    \overset{\begin{pmatrix} t \otimes 1 & 1 \otimes s \\ 1 \otimes s & - t \otimes 1 \end{pmatrix}}{\to}
    \Delta_{k}^{-} \otimes \Delta_{k}^{+} \oplus \Delta_{k}^{+} \otimes \Delta_{k}^{-}
    \to 0
  \]
  成分を見れば \(\Omega\) の作用から二つの複体は等しいとわかる。
\itemprof
  これは積を持つ一般コホモロジーの一般論からわかる。
\end{Proof}

\end{document}