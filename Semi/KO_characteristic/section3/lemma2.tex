\documentclass[dvipdfmx]{jsarticle}
% パッケージ
\usepackage{amsthm}
\usepackage{amsmath,amssymb,mathrsfs}
\usepackage{color}
\usepackage{tikz}

% 定理環境
%% 本体
\theoremstyle{definition}
\newtheorem*{tDefinition}{定義}
\newtheorem*{tTheorem}{定理}
\newtheorem*{tProof}{証明}
\newtheorem*{tNotation}{記法}
\newtheorem*{tRemark}{注意}
\newtheorem*{tWhen}{設定}

\newenvironment{Mini}{
  \begin{minipage}[t]{1\hsize}
  \setlength{\parindent}{10pt}
  \begin{itemize}
  \setlength{\labelsep}{10pt}
}{
  \end{itemize}
  \vspace{5pt}
  \end{minipage}
}

\newenvironment{Definition}[1][\quad]{
  \begin{tDefinition}
  #1 \\
  \begin{Mini}
}{
  \end{Mini}
  \end{tDefinition}
}

\newenvironment{Theorem}[1][\quad]{
  \begin{tTheorem}
  #1 \\
  \begin{Mini}
}{
  \end{Mini}
  \end{tTheorem}
}

\newenvironment{Proof}[1][\quad]{
  \begin{tProof}
  #1 \\
  \begin{Mini}
      }{
  \end{Mini}
  \end{tProof}
}

\newenvironment{When}{
  \begin{tWhen}
  \quad \\
  \begin{Mini}
}{
  \end{Mini}
  \end{tWhen}
}

\newenvironment{Remark}{
  \begin{tRemark}
}{
  \end{tRemark}
}

%% マーク
\newcommand{\itemwhen}{\item[\(\bigcirc\)]}
\newcommand{\itemnote}{\item[!]}

\newcommand{\itemdefi}{\item[\(\square\)]}
\newcommand{\itemprop}{\item[\(\vartriangleright\)]}
\newcommand{\itemand}{\item[\(-\)]}

\newcommand{\itemprof}{\item[\(\because\)]}
\newcommand{\itemthen}{\item[\(\rightsquigarrow\)]}

\newcommand{\itemenum}{\item[\(+\)]}
\newcommand{\itembase}{\item[\(\bullet\)]}
\newcommand{\itemwith}{\item[\(-\)]}

% 記述関係
\newenvironment{indentblock}{
  \\
  \hspace*{5mm}
  \begin{minipage}{0.8\textwidth}
}{
  \end{minipage}
  \\
}

%コマンド関連
\newcommand{\place}{\_}
\newcommand{\restr}[2]{\left. {#1} \right| _{#2}}
\newcommand{\txt}{\texttt}

%% 宣言
\newcommand{\declare}[1]{\textcolor[rgb]{0.1, 0.8, 0.2}{#1 }}
\newcommand{\For}{\declare{For}}
\newcommand{\Define}{\declare{Define}}
\newcommand{\Let}{\declare{Let}}
\newcommand{\IfHold}{\declare{If}}
\newcommand{\Then}{\declare{Then}}
\newcommand{\Take}{\declare{Take}}
\newcommand{\Fix}{\declare{Fix}}
\newcommand{\Return}{\declare{Return}}

%% 置く
\newcommand{\WIP}{\textcolor{red}{工事中}}
\newcommand{\SORRY}{\textcolor{red}{わかりませんでした}}
\newcommand{\ADMIT}{\textcolor{blue}{認めます}}
\newcommand{\AIMAI}[1]{\textit{#1}}

\begin{document}

\section*{示すことについて}

次のことは FACT とする。
\begin{Theorem}
\itemprop
  \Let \(\eta\) : \(\mathbb{C}\) vector bundle over \(S^2\) := reduced Hopf bundle \\
  \Then \(\sum_{k \geq 0} KO^{-k}(\text{point})\) is naturaly isomorphic to \(\mathbb{Z}[\eta]\)
\itemprop
  \Then \(\exists\) \(y\) : \(KO^{-8}(\text{point})\) , \(\sum_{k \geq 0} K^{-k}(\text{point})\) contains \(\mathbb{Z}[y]\) and complexification of \(y\) is \(x^4\) \\
  \Then \(\exists\) \(a,b,z\) : \(KO^{-1}(\text{point}),KO^{-2}(\text{point}),KO^{-4}(\text{point})\) as a \(\mathbb{Z}[y]\)-module , \(\sum_{k \geq 0} KO^{-k}(\text{point})\) is naturaly isomorphic to \((\text{freely generated by}\langle a , b , z \rangle / \langle 2a , 2b \rangle)\)
\end{Theorem}
次のことを示したい。

\begin{Theorem}
\itemdefi \(\eta := [0 \to \mathbb{R}^{k} \times \Delta^+ \to \mathbb{R}^{k} \times \Delta^- \to 0]\) : \(K_{\mathbb{K},cpt}(\mathbb{R}^k) \cong \tilde{K}_{\mathbb{K}}^{0}(S^k) = \tilde{K}_{\mathbb{K}}^{-k}(S^k)\)
\itemprop \(\beta(x) = x \cdot \eta\) : \(\tilde{K}_{\mathbb{K}}^{*}(X) \to \tilde{K}_{\mathbb{K}}^{*-k}(X)\) := \(\beta\) is isomorphism
\end{Theorem}

\section*{Bott 周期性について}
このセクションでは Bott 周期性 \(\beta\) : \(\tilde{K}_{\mathbb{K}}(X) \to \tilde{K}_{\mathbb{K}}^{-k}(X)\)を構成する。
KR理論を用いる。

\begin{Definition}
\itemdefi
  \Define space with involution , real space :=
  \begin{itemize}
    \itemenum Base \(\ldots\) \(X\) : space
    \itemenum \(\tau\) , \(\bar{\cdot}\) : \(X \to X\) s.t. \(\tau^2 = 1\)
  \end{itemize}
\itemdefi
  \For \(f\) : maps between real space \(X \to Y\) \\
  \Define \(f\) is real := commutes with the involutions
\itemdefi
  \For \(X,Y\) : real space \\
  \Define \(X \times Y\) : real space := Base \(X \times Y\) with involution \((x,y) \mapsto (\bar{x}, \bar{y})\)
\itemdefi
  \Define real vector bundle over \(X\) : real space :=
  \begin{itemize}
    \itemenum Base \(\ldots\) \(E\) : complex vector bundle over \(X\) and also real space
    \itemwith projection is real
    \itemwith for \(x\) : \(X\) , \(y \in E_{x} \to \bar{y} \in E_{\bar{x}}\) is anti-linear
  \end{itemize}
\itemdefi
  \For \(X\) : real space , \(E,F\) : real vector bundle over \(X\) \\
  \Define \(\)
\itemdefi
  \For \(X\) : real space \\
  \Define \(\mathscr{E}(X)\) : category := \(\mathbb{R}\) vector bundles over \(X\) \\
  \Define \(\mathscr{F}(X)\) : category := real vector bundles over \(X\)
\end{Definition}

\begin{Theorem}
\itemprop
  \IfHold \(X\) is trivial \\
  \Then \(\mathscr{E}(X) \cong \mathscr{F}(X)\)
\item 
\end{Theorem}

\begin{Theorem}
\itemprop
  \Then vector bundle in the category of real spaces and complex vector bundle in the category of \(\mathbb{Z}_2\)-spaces
\itemprop
  \For \(E\) : real vector bundle in the category of \(\mathbb{Z}_2\)-spaces \\
  \Then 
\end{Theorem}

\section*{\(\tilde{\mu}(\xi)\) が \(K_{\mathbb{K}}\)-orientation になることについて}
このセクションでは次のことを示す。
\begin{Theorem}
\itemprop
  \(0 \to \mathbb{R}^k \times S^+ \to \mathbb{R}^k \times S^- \to 0\) なる複体により得られる \(K_{\mathbb{K},cpt}(\mathbb{R}^k)\) の元は \(\cong \tilde{K}_{\mathbb{K}}^0(S^k)\) とみなしたとき Bott 周期性 \(\beta\) : \(\tilde{K}_{\mathbb{K}}(S^0) \to \tilde{K}_{\mathbb{K}}(S^k)\) により \(1 \in \tilde{K}_{\mathbb{K}}(S^0)\) が移った元である。
\end{Theorem}

ABS 構成を定義しこれによりわかる。

\begin{Definition}
\itemdefi
  \For \(\xi\) : real Euclidean vector bundle \\
  \Define \(\text{Cli}(\xi)\) : algebra bundle := \AIMAI{Clifford algebra bundle of \(\xi\)} \\
  \Define graded Clifford module of \(\xi\) := \(\text{Cli}(\xi)\)-module bundle
\itemdefi
  \For \(E\) : graded Clifford module of \(\xi\) \\
  \Let \(\pi\) : \(V \to X\) := projection of \(\xi\) \\
  \Define \(\sigma(E)\) : \(\pi^* E^1 \to \pi^* E^0\) := defined by \(\sigma(E)_v(e) = -v \cdot e\) \\
  \Define \(\chi_V(E)\) : cpt.supp.cpx over \((D(V), S(V))\) := \(0 \to \pi^* E^1 \overset{\sigma(E)}{\to} \pi^* E^0 \to 0\)
\end{Definition}

\begin{Definition}
\itemdefi
  \For \(\xi\) : real Euclidean vector bundle \\
  \Define \(M(\xi)\) : Abelian group := Grothendieck group of graded \(C(\xi)\)-modules \\
  \Let \(M(\xi \oplus 1) \to M(\xi)\) := \WIP \\
  \Define \(A(\xi)\) := cockernel of this \\
  \Define \(\chi_{\xi}\) : \(A(\xi) \to \tilde{KO}(\text{Thom}(\xi))\) 
\end{Definition}

\begin{Theorem}
\itemprop
  \For \(\xi_i\) : Euclidean vector bundle , \(E_i\) : \(\text{Cli}(\xi_i)\)-module bundle \\
  \(\chi_{\xi_1 \oplus \xi_2}(E_1 \hat{\otimes} E_2) = \chi_{\xi_1}(E_1) \chi_{\xi_2}(E_2)\)
\end{Theorem}

\begin{Theorem}
\itemprop
  \For \(P\) : \(\text{Spin}(k)\)-bundle over \(X\) , \(M\) : graded \(C_k\)-module \\
  \Let \(V\) := \(P \times_{\text{Spin}(k)} \mathbb{R}^k\) , \(E\) := \(P \times_{\text{Spin}(k)} M\) \\
  \Then \(E\) has a natural \(V\)-module structure
\itemdefi
  \Define \(\beta_P\) : \(A_k \to A(V)\) := in this way \\
\itemprop
  \For \(P_i\) : \(\text{Spin}(k_i)\)-bundle over \(X_i\) \\
  \Let \(P\) : \(\text{Spin}(k_1 + k_2)\)-bundle over \(X_1 \times X_2\) := \(P_1 \times P_2\) by the standard homomorphism \\
  \For \(a_i\) : \(A_{k_i}\) \\
  \Then \(\beta_P(a_1 a_2) = \beta_{P_1}(a_1) \beta_{P_2}(a_2)\)
\end{Theorem}

\begin{Definition}
\itemdefi
  \For \(P\) : above \\
  \Define \(\alpha_{P}\) := \(\xi_{V} \beta_{P}\)
\itemdefi
  \Define \(\alpha\) : \(A_{k} \to \tilde{KO}(S^k)\) := defined by \(\alpha_{P}\) where \(X\) is point
\end{Definition}

\begin{Theorem}
\itemprop
  \Then \(\alpha\) : \(A_{k} \to KO^{-k}(pt)\) := \(\sum \alpha\) is isomorphism
\end{Theorem}

\end{document}