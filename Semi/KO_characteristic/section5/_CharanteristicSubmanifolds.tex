\begin{Theorem}
\itemwhen
  \For \((p : V \to X)\) : \(\text{Spin}^c\) bundle over closed manifold \\
  \Let \(D\) : vector bundle over \(X\) := determinant bundle of this \\
  \For \(s\) : section of \(D\) transverse to \(X\) \\
  \Let \(Y\) : sub manifold of \(X\) := \(s^{-1}(X)\) \\
  \Let \(i\) : \(Y \to X\) := inclusion
\itemprop
  \Then normal bundle of \(Y\) in \(X\) is canonically isomorphic to the bundle \(\restr{D}{Y} \to Y\)
\itemwhen
  \Fix \(U\) : tubular neighborhood of \(Y\) in \(X\), identification of \(U\) and \(B(\restr{D}{Y})\)
\end{Theorem}
  
\begin{Proof}
\itemprof
  \ADMIT
\end{Proof}

\begin{Theorem}
\itemwhen
  \Let \(\tilde{\pi}\) : \(\restr{D}{Y} \to Y\) := restriction
\itemprop
  \Define \(\tau\) : \(\Gamma(B(\restr{D}{Y}); B(\restr{D}{Y}) \times_{\tilde{\pi}} \restr{D}{Y})\) := \AIMAI{canonical section} \\
  \Then bundle \(\restr{D}{U} \to U\) can be identified with the bundle \(B(\restr{D}{Y}) \times_{\tilde{\pi}} \restr{D}{Y} \to B(\restr{D}{Y})\) \\
  \Then this identification coincidents with \(\restr{s}{U} , \tau\)
\itemprop
  \Then \(\restr{V}{U} \to U\) and \(B(\restr{D}{Y}) \times_{\tilde{\pi}}\restr{V}{Y} \to B(\restr{D}{Y})\) can be identified
\itemprop
  \Then \(\restr{V}{Y} \oplus \restr{D}{Y} \overset{\pi_2}{\to} \restr{D}{Y}\) and \(\restr{D}{Y} \times_{\tilde{\pi}} \restr{V}{Y} \to \restr{D}{Y}\) can be identified
\end{Theorem}

\begin{Proof}
\itemprof
  \ADMIT
\itemprof
  \ADMIT
\itemprof
  これは確かに成り立つ。
\end{Proof}

\begin{Theorem}
\itemdefi
  \Define \(j\) : \(B(\restr{V}{Y} \oplus \restr{D}{Y}) \to V\) := resulting open embedding
\itemprop
  \Then \(j_* \mu(i^* V \oplus i^* D) = r \mu_c(V)\)
\end{Theorem}

\begin{Proof}
\itemprof
  状況としては、
  \begin{itemize}
    \item \(S^{\pm}\) : vector bundle over \(X\) := defined by spin structure on \(V \oplus D\)
    \item \(\Delta\) : \(\Gamma((V \oplus D);q^*(V \oplus D) = (V \oplus D) \times_{X} (V \oplus D))\) := \(v \mapsto (v, v)\)
    \item \(m\) : \(q^*(V \oplus D) \times_{V \oplus D} q^*(S^{+}) \to q^*(S^{-}) = ((V \oplus D) \times_{X} (V \oplus D)) \times_{V \oplus D} ((V \oplus D) \times_{X} S^{+})\) := multiplication
    \item 
    \[\begin{matrix}
      \restr{D}{U} &\to& U \\
      \downarrow \phi_{D} & & \downarrow \phi \\
      B(\restr{D}{Y}) \times_{\tilde{\pi}} \restr{D}{Y} &\to& B(\restr{D}{Y})
    \end{matrix}\]
    \item 
    \[\begin{matrix}
      \restr{V}{U} &\to& U \\
      \downarrow \phi_{V} & & \downarrow \phi \\
      B(\restr{D}{Y}) \times_{\tilde{\pi}} \restr{V}{Y} &\to& B(\restr{D}{Y})
    \end{matrix}\]
    \item 
    \[\begin{matrix}
      \restr{S^{\pm}}{U} &\to& U \\
      \downarrow \phi_{S^{\pm}} & & \downarrow \phi \\
      B(\restr{D}{Y}) \times_{\tilde{\pi}} \restr{S^{\pm}}{Y} &\to& B(\restr{D}{Y})
    \end{matrix}\]
    \item 
    \[\begin{matrix}
      \restr{V}{Y} \oplus \restr{D}{Y} &\overset{\pi_2}{\to}& \restr{D}{Y} \\
      \downarrow h & & | \\
      \restr{D}{Y} \times_{\tilde{\pi}} \restr{V}{Y} &\to& \restr{D}{Y}
    \end{matrix}\]
  \end{itemize}
  となっていた。
  \(\restr{V}{Y} \oplus \restr{D}{Y}\) に対応する構成によるベクトル束が \(\restr{S^{+}}{Y}\) となることから、
  \begin{align*}
    (\restr{V}{Y} \oplus \restr{D}{Y}) \times_{Y} \restr{S^{+}}{Y} &\to (\restr{V}{Y} \oplus \restr{D}{Y}) \times_{Y} \restr{S^{-}}{Y} 
    & \text{over} \, \restr{V}{Y} \oplus \restr{D}{Y} \\
    (v_1 , v_3) & \mapsto (v_1, m(\Delta(v_1), v_3))
  \end{align*}
  のようになっていた。
  これが \(\restr{V}{U}\) 上の複体となるように次のように移す。
  対応するベクトル束とその写像を書いた。
  \begin{itemize}
    \item \((\restr{V}{Y} \oplus \restr{D}{Y}) \times_{Y} \restr{S^{\pm}}{Y} \to (\restr{V}{Y} \oplus \restr{D}{Y}) \times_{Y} \restr{S^{-}}{Y}\) : over \(\restr{V}{Y} \oplus \restr{D}{Y}\) \\
    \((v_1, v_3) \mapsto (v_1, m(\Delta(v_1), v_3))\)
    \item \((\restr{D}{Y} \times_{Y} \restr{V}{Y}) \times_{Y} \restr{S^{\pm}}{Y}\) : over \(\restr{D}{Y} \times_{Y} \restr{V}{Y}\) \\
    \((v_1, v_3) \mapsto (v_1, m(\Delta(h^{-1} v_1), v_3))\)
    \item \((B(\restr{D}{Y}) \times_{Y} \restr{V}{Y}) \times_{Y} \restr{S^{\pm}}{Y}\) : over \(B(\restr{D}{Y}) \times_{Y} \restr{V}{Y}\) \\
    \((v_1, v_3) \mapsto (v_1, m(\Delta(h^{-1} v_1), v_3))\)
    \item \((B(\restr{D}{Y}) \times_{Y} \restr{V}{Y}) \times_{B(\restr{D}{Y})} (B(\restr{D}{Y}) \times_{Y} \restr{S^{\pm}}{Y})\) \\
    \((v_1, (v_2, v_3)) \mapsto (v_1, (v_2, m(\Delta(h^{-1} v_1), v_3)))\) ただし \(v_2 = \text{proj}\,v_1\) \\
    \((v_1, v_3) \mapsto (v_1, (\text{proj}_1 v_3, m(\Delta(h^{-1} v_1, \text{proj}_2 v_3))))\)
    \item \(\restr{V}{U} \times_{U} \restr{S^{\pm}}{U}\) \\
    \((v_1, v_3) \mapsto (v_1, {\phi_{S^{-}}}^{-1} (\text{proj}_1 \phi_{S^{+}} v_3, m(\Delta(h^{-1} \phi_{V} v_1), \text{proj}_2 \phi_{S^{+}} v_3)))\)
  \end{itemize}
  ここで最後の関数について、次を示す。
  \((v_1, v_3) \in \restr{V}{U} \times_{U} \restr{S}{U}\) のとき
  \[{\phi_{S^{-}}}^{-1} (\text{proj}_1 v_3, m(\Delta(h^{-1} \phi_{V} v_1), \text{proj}_2 \phi_{S^{+}} v_3)) = m(\Delta(v_1, sp(v_1)), v_3)\]
\end{Proof}

\begin{Proof}
\itemprof
  \(\omega\) := canonical section of \(p^*(V \oplus D) \to V \oplus D\) とするとき、
  \(\alpha\) : \(p^*(\restr{S^{+}}{U}) \to p^*(\restr{S^{-}}{U})\) := at \(v\), Clifford multiplication by \(\omega (v + s p(v))\) として
  このとき \(\mu({i^*} V \oplus {i^*} D)\) が次の complex で表せることに注意する。
  \[
    0 \to {p^*} (\restr{S^{+}}{U}) \overset{\alpha}{\to} {p^*} (\restr{S^{-}}{U}) \to 0
  \]
  これにより section 1 での結果から \({j_*} \mu({i^*} V \oplus {i^*} D)\) は次の complex で表される。
  \(\alpha\) : \({p^*} S^{+} \to {p^*} S^{-}\) := at \(v\), Clifford multiplication by \(\omega (v + s p (v))\) 
  \[
    0 \to {p^*} S^{+} \overset{\alpha}{\to} {p^*} S^{-} \to 0
  \]
  一方で \(r \mu_c (V)\) がどのように表されるかが Theorem 1 にあり、 \({p^*} S^{+} \to {p^*} S^{-}\) なる写像のみが上で述べた複体と異なる。
  従って、二つの写像が homotopic であることを示せばよいが、
  \(F_t\) := for \(t \in [0,1]\), at \(v \in {p^*} S^{+}\), Clifford multiplication by \(\omega (v + t s p(v))\) とすればこれが homotopy を与える。
\end{Proof}