\begin{Theorem}
\itemwhen
  \For \(\xi = (p : V \to X)\) : \(\text{Spin}^c\) bundle over closed manifold \\
  \Let \(\eta = (\pi : D \to X)\) : vector bundle over \(X\) := determinant bundle of \(\xi\) \\
  \For \(s\) : section of \(\eta\) transverse to \(X\) \\
  \Let \(Y\) : sub manifold of \(X\) := \(s^{-1}(X)\) \\
  \Let \(i\) : \(Y \to X\) := inclusion
\itemprop
  \Then normal bundle of \(Y\) in \(X\) is canonically isomorphic to the bundle \(\restr{\eta}{Y}\)
\end{Theorem}
  
\begin{Proof}
\itemprof
  以下の構成において、 \(\xi \leftarrow \zeta, \ldots\) とおけばよい。
\end{Proof}

\begin{Definition}
\itemwhen
  \For \(\xi = (p : E \to M)\), \(s\) : \(\Gamma(\xi)\) s.t. transverse to \(0\) \\
  \Let \(Y\) := \(s^{-1}(0)\) \\
  \Let \(\Phi\) : isomorphism \(\tau(M/E) \to \xi\) := \AIMAI{canonical} 
\itemdefi
  \Define \(J\) : isomorphism \(\tau(Y/M) \to \restr{\xi}{Y}\) := at \(x \in Y\),
  \begin{indentblock}
    \For \(v\) : \(\tau_{x}(Y/M)\) \\
    \Return \(\Phi \restr{(s_* v)}{\tau(M/E)}\) 
  \end{indentblock}
\end{Definition}

\begin{Proof}
\itemprof
  Well-defined 性について
  \(s\) が smooth であることから局所的にはグラフとして書けて、 \(v\) : \(\tau(Y/M) \subset T_x M\) は \(s_*\) により \(T_x E\) の元として確かに定義される。
  \(T_x E = T_x M \oplus \tau(M/E)\) のため直和分解して確かにつじつまはあっている。
\itemprof
  isomorphism であること
  次元を考えると単射であることを示せばよい。
  \(\Phi\) の同型性と \(s_*\) の単射性を加味すると、 \(v\) : \(T_x S\) が \(\restr{v}{\tau(M/E)} = 0\) を満たすならば \(v = 0\) であることを示せばよい。
  \(S\) が横断的だから確かに成り立つ。
\end{Proof}

\begin{Theorem}
\itemwhen
  \For same assumptions as before \\
  \Let \(r\) : \(U \to Y\) := canonical retraction \\
  \Let \(R_0\) : \(B(\restr{\xi}{U}) \to B(\tau(Y/M))\) := induced by \(J^{-1} \circ p\) \\
  \Let \(\pi\) := projection of \(\tau(Y/M)\)
\itemprop
  \Then \((R_0, r)\) is \(((D(\restr{\xi}{U})) \to U) \to (D(\tau(Y/M)) \to Y)\) \\
  \Then \(R_0 \circ s\) is diffeomorphism \(U \to D(\tau(Y/M))\)
\itemwhen
  \Let \(q = R_0 \circ s\) \\
  \Let \(Q\) : \(E(\restr{\xi}{U}) \to E((\restr{\pi}{D(\tau(Y/M))})^* \tau(Y/M))\) := \(y \mapsto (R_0 s p y, R_0 y)\)
\itemprop
  \Then \((Q, q)\) is \(\restr{\xi}{U} \to (\restr{\pi}{D(\tau(Y/M))})^* \tau(Y/M)\) \\
  \Then \((Q, q)\) is isomorphism
\itemprop
  \Then under isomorphism of \((Q, q)\), \(\restr{s}{U}\) : \(\Gamma(\restr{\xi}{U})\) correspond to \(x \mapsto (x, x)\) : \(\Gamma((\restr{\pi}{D(\tau(Y/M))})^* \tau(Y/M))\)
\end{Theorem}

\begin{Proof}
\itemprof
  
\end{Proof}

\begin{Theorem}
\itemprop
  \Let \(\gamma = (q : V \oplus D \to X)\) : vec with spin st over \(X\) := \(\xi \oplus \eta\) with natural st \\
  \Define \(\tau\) : \(\Gamma(q^* \gamma)\) := \AIMAI{canonical section} \\
  \Then \(\restr{\eta}{U}\) can be identified with \((\restr{\pi}{B(\restr{\eta}{Y})})^* \restr{\eta}{Y}\) \\
  \Then this identification coincidents with \(\restr{s}{U} , \tau\)
\itemprop
  \Then \(\restr{\xi}{U}\) can be identified with \((\restr{\pi}{B(\restr{\eta}{Y})})^* \restr{\xi}{Y}\) 
\itemprop
  \Then \(\restr{V}{Y} \oplus \restr{D}{Y} \overset{\pi_2}{\to} \restr{D}{Y}\) and \(\restr{D}{Y} \times_{\tilde{\pi}} \restr{V}{Y} \to \restr{D}{Y}\) can be identified
\end{Theorem}

\begin{Proof}
\itemprof
  \ADMIT
\itemprof
  \ADMIT
\itemprof
  これは確かに成り立つ。
\end{Proof}

\begin{Theorem}[補題]
\itemprop
  \For \(\xi = (p : V \to X)\) : vector bundle, \(s\) : \(\Gamma(V)\) \\
  \Then \((x, t) \in X \times \mathbb{K} \mapsto t * s(x) \in V\) is vec.bnd.hom of \(1 \to \xi\)
\end{Theorem}

\begin{Theorem}
\itemdefi
  \Define \(j\) : \(B(\restr{V}{Y} \oplus \restr{D}{Y}) \to V\) := resulting open embedding
\itemprop
  \Then \(j_* \mu(i^* V \oplus i^* D) = r \mu_c(V)\)
\end{Theorem}

\begin{Proof}
\itemprof
  状況としては
  \begin{itemize}
    \itemenum \(\xi = (p : V \to X)\) : \(\text{Spin}^c\)-bundle
    \itemenum \(\eta = (\pi : D \to X)\) : vector bundle over \(X\) := determinant bundle of \(\xi\)
    \itemenum \(\gamma = (q : V \oplus D \to X)\) : \(\text{Spin}\)-bundle
    \itemenum \(\omega^{\pm} = (r : S^{\pm} \to X)\) : vector bundle over \(X\) := spinor bundle of \(\eta\)
    \itemenum \(s\) : \(\Gamma(\eta)\)
    \itemenum \(\tau\) : \((\restr{\pi}{B(\restr{\eta}{Y})})^*\eta\) := canonical one
    \itemenum \(\delta\) : \(\Gamma(q^* \gamma)\) := canonical one
    \itemenum \(m\) : \(\gamma \otimes \omega^{+} \to \omega^{-}\) := multiplication
    \itemenum \(\phi\) : homeo \(U \to B(\restr{\eta}{U})\)
    \itemenum \((\phi_{\eta}, \phi)\) : \(\restr{\eta}{U} \cong (\restr{\pi}{B(\restr{\eta}{Y})})^* (\restr{\eta}{Y})\)
    \itemenum \((\phi_{\xi}, \phi)\) : \(\restr{\xi}{U} \cong (\restr{\pi}{B(\restr{\eta}{Y})})^* (\restr{\xi}{Y})\)
    \itemenum \((\phi_{\omega^{\pm}}, \phi)\) : \(\restr{\omega^{\pm}}{U} \cong (\restr{\pi}{B(\restr{\eta}{Y})})^* (\restr{\omega^{\pm}}{Y})\)
    \itemenum \(h\) : \(V \oplus D \to p^*\eta\) := \AIMAI{by natural definition}
  \end{itemize}
  ここで \(\mu(\restr{\gamma}{Y}) \in KO(\restr{V}{Y} \oplus \restr{D}{Y})\) がどのような複体で表されたかを考えると、
  \[
    (\restr{q}{Y})^*(\restr{\omega^{+}}{Y}) \overset{\alpha}{\to} (\restr{q}{Y})^*(\restr{\omega^{-}}{Y})
  \]
  ここで \(\alpha\) は
  \begin{align*}
    (\restr{q}{Y})^*(\restr{\omega^{+}}{Y}) \\
    \downarrow \cong \\
    1 \otimes (\restr{q}{Y})^*(\restr{\omega^{+}}{Y}) \\
    \downarrow \restr{\tau}{Y} \otimes 1 \\
    (\restr{q}{Y})^*(\restr{\gamma}{Y}) \otimes (\restr{q}{Y})^*(\restr{\omega^{+}}{Y}) \\
    \downarrow \cong \\
    (\restr{q}{Y})^*(\restr{\gamma}{Y} \otimes \restr{\omega^{+}}{Y}) \\
    \downarrow {\restr{q}{Y}}^* m \\
    (\restr{q}{Y})^*(\restr{\omega^{-}}{Y})
  \end{align*}
  と表された。これを \(\restr{V}{U} \overset{\phi_{\xi}}{\to} B(\restr{\eta}{Y}) \times_{Y} \restr{V}{Y} \overset{h}{\to} \restr{V}{Y} \oplus \restr{D}{Y}\) と引き戻したものを考えたとき、次の \(\restr{V}{U}\) 上の複体と同型であることを示す。。
  \[
    (\restr{p}{U})^*(\restr{\omega^{+}}{U}) \overset{\beta}{\to} (\restr{p}{U})^*(\restr{\omega^{-}}{U})
  \]
  ここで \(\beta\) は \(sp := v \in \restr{V}{U} \to (v, (v, spv)) \in \restr{V}{U} \times_{U} (\restr{V}{U} \times_{U} \restr{D}{U})\) なる \(\Gamma((\restr{p}{U})^*(\restr{\gamma}{U}))\) の元を \(1 \to (\restr{p}{U})^*(\restr{\gamma}{U})\) と考えたとき、
  \begin{align*}
    (\restr{p}{U})^* (\restr{\omega^{+}}{U}) \\
    \downarrow \cong \\
    1 \otimes (\restr{p}{U})^* (\restr{\omega^{+}}{U}) \\
    \downarrow sp \otimes 1 \\
    (\restr{p}{U})^* (\restr{\gamma}{U}) \otimes (\restr{p}{U})^* (\restr{\omega^{+}}{U}) \\
    \downarrow \cong \\
    (\restr{p}{U})^* (\restr{\gamma}{U} \otimes \restr{\omega^{+}}{U}) \\
    \downarrow (\restr{p}{U})^* m \\
    (\restr{p}{U})^* (\restr{\omega^{-}}{U})
  \end{align*}
  と表される。
\itemthen
  可換な図式として次のものがある
  \[\begin{matrix}
    \restr{V}{U} & \overset{\phi_{\xi}}{\to} & B(\restr{\eta}{Y}) \times_{Y} \restr{V}{Y} & \overset{h}{\to} & \restr{V}{Y} \oplus \restr{D}{Y} & \overset{\text{proj}_2}{\to} & \restr{D}{Y} \\
    \downarrow p & & \downarrow \text{proj}_1 & & \downarrow \restr{q}{Y} & & \downarrow \restr{\pi}{Y} \\
    U & \overset{\phi}{\to} & B(\restr{\eta}{Y}) & \overset{\restr{\pi}{B(\restr{\eta}{Y})}}{\to} & Y & = & Y
  \end{matrix}\]
  まず、 \(\xi, \eta, \gamma, \omega^{\pm}\) のいずれもベクトル束としての同型 \((\restr{\pi}{B(\restr{\eta}{Y})})^* ({\restr{\place}{Y}}) \cong \restr{\place}{U}\) から得られる \(U\) 上のベクトル束の同型 \(\phi^* (\restr{\pi}{B(\restr{\eta}{Y})})^* ({\restr{\place}{Y}}) \cong \restr{\place}{U}\) があることに注意すると、 \((h \phi_{\xi})^* (\restr{q}{Y})^* {\restr{\place}{Y}} \cong p^* \phi^* (\restr{\pi}{B(\restr{\eta}{Y})})^* \restr{\place}{Y} \cong p^* \restr{\place}{U}\) によりベクトル束の複体のうち対応するベクトル束の同型はこれにより得られる。
  次にこの引き戻しと同型により写像が対応するかどうかについて議論すると、 natural に得られる同型の部分と \(\place ^* m\) のように書かれている部分が対応しあっているので、切断同士が対応するかどうかについて議論すればよい。
  すなわち、
  \[
    \restr{\tau}{Y} \in \Gamma((\restr{q}{Y})^* \restr{\gamma}{Y}) := v \in \restr{V}{Y} \oplus \restr{D}{Y} \mapsto (v,v) \in \restr{V}{Y} \oplus \restr{D}{Y} \times_{Y} \restr{V}{Y} \oplus \restr{D}{Y}
  \]
  を \((h \phi_{\xi})^*\) で引き戻し、 \((h \phi_{\xi})^* (\restr{q}{Y})^* {\restr{\gamma}{Y}} \cong p^* \phi^* (\restr{\pi}{B(\restr{\eta}{Y})})^* \restr{\gamma}{Y} \cong p^* \restr{\gamma}{U}\) なる同型で \(\Gamma(p^* \restr{\gamma}{U})\) の元として考えたとき、
  \[
    v \in \restr{V}{U} \to (v, spv) \in \restr{V}{U} \times_{U} \restr{D}{U}
  \]
  になることを示す。
  \begin{align*}
    &\restr{\tau}{Y} : \Gamma((\restr{q}{Y})^* \restr{\gamma}{Y}) \\
    &= v \in v \in \restr{V}{Y} \oplus \restr{D}{Y} \mapsto (v,v) \in \restr{V}{Y} \oplus \restr{D}{Y} \times_{Y} \restr{V}{Y} \oplus \restr{D}{Y} \\
    &(h \phi_{\xi})^* \restr{\tau}{Y} : \Gamma((h \phi_{\xi})^* (\restr{q}{Y})^* \restr{\gamma}{Y}) \\
    &= v \in \restr{V}{U} \mapsto (v, \restr{\tau}{Y} (h \phi_{\xi} v)) \in \restr{V}{U} \times_{\restr{V}{Y} \oplus \restr{D}{Y}} (\restr{V}{Y} \oplus \restr{D}{Y} \times_{Y} \restr{V}{Y} \oplus \restr{D}{Y}) \\
    &\place : \Gamma((\restr{q}{Y} h \phi_{\xi})^* \restr{\gamma}{Y}) \\
    &= v \in \restr{V}{U} \mapsto (v, h \phi_{\xi} v) \in \restr{V}{U} \times_{Y} \restr{V}{Y} \oplus \restr{D}{Y} \\
    &\place : \Gamma(p^* (\phi^* (\restr{\pi}{B(\restr{\eta}{Y})})^* \restr{\gamma}{Y})) \\
    &= v \in \restr{V}{U} \mapsto (v, p v, \phi p v, h \phi_{\xi} v) \in \restr{V}{U} \times_{U} U \times_{B(\restr{\eta}{Y})} B(\restr{\eta}{Y}) \times_{Y} (\restr{V}{Y} \oplus \restr{D}{Y}) \\
    &\place : \Gamma(p^* \left( (\phi^* (\restr{\pi}{B(\restr{\eta}{Y})})^* \restr{\xi}{Y}) \oplus (\phi^* (\restr{\pi}{B(\restr{\eta}{Y})})^* \restr{\eta}{Y}) \right)) \\
    &\text{as in} \, \restr{V}{U} \times_{U} \left(U \times_{B(\restr{\eta}{Y})} B(\restr{\eta}{Y}) \times_{Y} \restr{V}{Y}\right) \oplus \left(U \times_{B(\restr{\eta}{Y})} B(\restr{\eta}{Y}) \times_{Y} \restr{D}{Y} \right) \\
    &= v \in \restr{V}{U} \mapsto (v, (p v, \phi p v, \text{proj}_1 h \phi_{\xi} v), (v, (p v, \phi p v, \text{proj}_2 h \phi_{\xi} v))) \\
    &= v \in \restr{V}{U} \mapsto (v, (p v, \phi_{\xi} v), (p v = \pi s p v, \restr{\tau}{Y} \phi p v = \phi_{\eta} s p v)) \\
    &\place : \Gamma(p^* (\restr{\xi}{U} \oplus \restr{\eta}{Y})) \\
    &\text{as in} \, \restr{V}{U} \times_{U} (\restr{V}{U} \oplus \restr{D}{U}) \\
    &= v \in \restr{V}{U} \mapsto (v, (v, sp v)) \\
  \end{align*}
  示された。
\itemthen
  こうして得られた \(\restr{V}{U}\) 上の複体 \((\restr{p}{U})^* (\restr{\omega^{+}}{U}) \overset{\beta}{\to} (\restr{p}{U})^* (\restr{\omega^{-}}{U})\) は \(v \in \restr{V}{U}\) では \((v, sp(v)) \in \restr{V}{U} \oplus \restr{D}{U}\) による multiplication として \(\restr{S^{+}}{pv} \to \restr{S^{-}}{pv}\) のように定義された。
  与えられた複体の support を考えると、 \(Y\) の外側では \(s \not = 0\) より \(Y\) に含まれるから compact である。
  この複体が \(\restr{V}{U} \to V\) という open embeding によって pushout されたとき、複体自体が自明に compact support な複体への拡張を持つことから、定理 1 より \(j_* \mu(\restr{\gamma}{Y}) = [p^* \omega^{+} \to p^* \omega^{-}]\) として書ける。
  \(r \mu_c(V)\) の記述を以前に行ったためこれと比較を行うと、 \(q^* \omega^{+} \to q^* \omega^{-}\) としては \(v \in V\) 上で \(\omega(v,0), \omega((v,sp(v)))\) のどちらで multiplication を行うかの違いがある。
  これらは \(\omega(v, t* spv)\) で homotopic であるからよい。
  したがって示された。
\end{Proof}