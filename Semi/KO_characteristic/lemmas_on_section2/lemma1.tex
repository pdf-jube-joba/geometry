\documentclass[dvipdfmx]{jsarticle}
% パッケージ
\usepackage{amsthm}
\usepackage{amsmath,amssymb,mathrsfs}
\usepackage{color}
\usepackage{tikz}

% 定理環境
%% 本体
\theoremstyle{definition}
\newtheorem*{tDefinition}{定義}
\newtheorem*{tTheorem}{定理}
\newtheorem*{tProof}{証明}
\newtheorem*{tNotation}{記法}
\newtheorem*{tRemark}{注意}
\newtheorem*{tWhen}{設定}

\newenvironment{Mini}{
  \begin{minipage}[t]{1\hsize}
  \setlength{\parindent}{10pt}
  \begin{itemize}
  \setlength{\labelsep}{10pt}
}{
  \end{itemize}
  \vspace{5pt}
  \end{minipage}
}

\newenvironment{Definition}[1][\quad]{
  \begin{tDefinition}
  #1 \\
  \begin{Mini}
}{
  \end{Mini}
  \end{tDefinition}
}

\newenvironment{Theorem}[1][\quad]{
  \begin{tTheorem}
  #1 \\
  \begin{Mini}
}{
  \end{Mini}
  \end{tTheorem}
}

\newenvironment{Proof}[1][\quad]{
  \begin{tProof}
  #1 \\
  \begin{Mini}
      }{
  \end{Mini}
  \end{tProof}
}

\newenvironment{When}{
  \begin{tWhen}
  \quad \\
  \begin{Mini}
}{
  \end{Mini}
  \end{tWhen}
}

\newenvironment{Remark}{
  \begin{tRemark}
}{
  \end{tRemark}
}

%% マーク
\newcommand{\itemwhen}{\item[\(\bigcirc\)]}
\newcommand{\itemnote}{\item[!]}

\newcommand{\itemdefi}{\item[\(\square\)]}
\newcommand{\itemprop}{\item[\(\vartriangleright\)]}
\newcommand{\itemand}{\item[\(-\)]}

\newcommand{\itemprof}{\item[\(\because\)]}
\newcommand{\itemthen}{\item[\(\rightsquigarrow\)]}

\newcommand{\itemenum}{\item[\(+\)]}
\newcommand{\itembase}{\item[\(\bullet\)]}
\newcommand{\itemwith}{\item[\(-\)]}

% 記述関係
\newenvironment{indentblock}{
  \\
  \hspace*{5mm}
  \begin{minipage}{0.8\textwidth}
}{
  \end{minipage}
  \\
}

%コマンド関連
\newcommand{\place}{\_}
\newcommand{\restr}[2]{\left. {#1} \right| _{#2}}
\newcommand{\txt}{\texttt}

%% 宣言
\newcommand{\declare}[1]{\textcolor[rgb]{0.1, 0.8, 0.2}{#1 }}
\newcommand{\For}{\declare{For}}
\newcommand{\Define}{\declare{Define}}
\newcommand{\Let}{\declare{Let}}
\newcommand{\IfHold}{\declare{If}}
\newcommand{\Then}{\declare{Then}}
\newcommand{\Take}{\declare{Take}}
\newcommand{\Fix}{\declare{Fix}}
\newcommand{\Return}{\declare{Return}}

%% 置く
\newcommand{\WIP}{\textcolor{red}{工事中}}
\newcommand{\SORRY}{\textcolor{red}{わかりませんでした}}
\newcommand{\ADMIT}{\textcolor{blue}{認めます}}
\newcommand{\AIMAI}[1]{\textit{#1}}

\begin{document}

\section*{一般コホモロジーの向き付け}
積のある一般コホモロジーにおいて orientation を定義し、 orientation が Thom 同型を導くことを示す。

\begin{When}
\itemwhen
  \Fix \(E\) : cohomology theory with product \\
\end{When}

\begin{Definition}
\itemdefi
  \For \(\xi\) : vector bundle \\
  \Let \(\text{Thom}(\xi)\) := Thom space of \(\xi\) \\
  \Let \(n\) := \(\text{dim}(\xi)\) \\
  \For \(i_{x}\) : for \(x\) : \(\text{Base}(\xi)\) , \(S^n \to \text{Thom}(\xi)\) s.t. induced by some inclusion \(\mathbb{R}^n \to \text{Total}(\restr{\xi}{x}) \to \text{Total}(\xi)\) \\
  \Let \(\sigma\) : \(\tilde{E}^{0}(S^0) \to \tilde{E}^{n}(S^n)\) := suspension isomorphism \\
  \Define \(E\)-orientation of \(\xi\) := \(\mu\) : \(\tilde{E}^{n}(\text{Thom}(\xi))\) s.t. \(\forall\) \(x\) : \(\text{Base}(\xi)\) , \({i_x}^* (\mu) = \sigma(1)\)
\end{Definition}

\begin{Theorem}
\itemprop
  \For \(\xi\) : vector bundle , \(\mu\) : \(E\)-orientation , \(f\) : \(X \to \text{Base}(\xi)\)  \\
  \Let \(f_\dagger\) : \(\text{Thom}(f^* \xi) \to \text{Thom}(\xi)\) := induced by \((x,v) \mapsto v\) \\
  \Then \(f_\dagger^* \mu\) is \(E\)-orientation of \(f^*\xi\)
\itemprop
  \For \(\xi_1 , \xi_2\) : vector bundle , \(\mu_i\) : \(E\)-orientation of \(\xi_i\) \\
  \Let \(i\) : \(\text{Thom}(\xi_1 \times \xi_2) \to \text{Thom}(\xi_1) \wedge \text{Thom}(\xi_2)\) := induced by \((e_1,e_2) \mapsto e_1 \wedge e_2\) \\
  \Then \(i^*(\mu_1 \times \mu_2)\) is \(E\)-orientation of \(\xi_1 \times \xi_2\)
\end{Theorem}

\begin{Proof}
\itemprof
  \(f_\dagger\) が誘導されることを示す。
  \(i\) := \((x,e) \in \text{Total}(f^* \xi) \to e \in \text{Total}(\xi)\) が proper であることを示せばよい。
  \(K\) : compact subset in \(\text{Total}(\xi)\) をとる。
  \(i^{-1}(K)\) が compact であることは、 \(f^{-1}(\pi(K)) \times K\) なる compact 集合に含まれているから良い。

  \(f_{\dagger}^* \mu\) が \(E\)-orientation であることについては \(f^*\xi\) のファイバーを考えれば確かに成り立つ。
\itemprof
  \(i\) が誘導されることを示す。
  \(E_i := \text{Total}(\xi_i)\) とする。
  \(O\) : \(E_1^\infty \wedge E_2^\infty\) の基点の近傍、に対して、 \(i^{-1}(O^c)\) が compact であることを示せばよい。
  またさらに、 \(E_1^\infty \times E_2^\infty\) の \(E_1^\infty \times \infty \cup \infty \times E_2^\infty\) を含む開集合 \(O\) について、 \(O \cap E_1 \times E_2\) の補集合が compact であることを示せばよい。
  \(E_1^\infty , E_2^\infty\) の compact 性から tube lemma により \(O_i\) : \(E_i^\infty\) の中の \(\infty\) の近傍で \(O \supset E_1^\infty \times \O_2\cup O_1 \times E_2^\infty\) がとれて定義により \(O_1^c, O_2^c\) は compact のため \(O^c \subset O_1^c \times O_2^c\) と合わせて \(O^c\) が compact とわかる。

  \(i_*(\mu_1 \times \mu_2)\) が \(E\)-orientation になることについては上と同様にファイバーごとに考えると \((x,y)\) : \(\text{Base}(\xi_1) \times \text{Base}(\xi_2)\) について \({i_x}^* i^*(\mu_1 \times \mu_2)\) が \(S^{n+m} \to S^n \wedge S^m\) のように与えられているためよい。
\end{Proof}

\begin{Theorem}
\itemprop
  \For \(\xi\) : vector bundle where \(X\) : compact , \(\mu\) : \(E\)-orientation of \(\xi\) \\
  \Then \(\tilde{E}^{*}(\text{Thom}(\xi))\) is a free \(E^{*}(\text{Base}(\xi))\)-module with generator \(\mu\)
\end{Theorem}

\begin{Proof}
\itemprof
  \(\xi\) が product bundle の場合に示す。
  \(E\)-orientation は \(\tilde{E}^n(\text{Thom}(X \times \mathbb{R}^n))\) の元である。
  \(\text{Thom}(X \times \mathbb{R}^n) \cong X^\infty \wedge S^n\) により、 \(x \in E^{*}(X) \cong \tilde{E}^{*}(X^+)\) に対して、 \(x \cdot \mu \in \tilde{E}^{*+n}(\text{Thom}(X \times \mathbb{R}^n))\) なる元と \(x \cdot \sigma(1) \in \tilde{E}^{*}(X^+ \wedge S^n)\) が対応すれば、 \(x \in \tilde{E}^{*}(X^+) \mapsto x \cdot 1 \in \tilde{E}^{*}(X^+)\) が明らかに同型であることと \(\sigma\) が同型であることからわかる。
  この対応は \(\mu\) の定義により成り立つ(符号を入れ替えるように行う必要がある)。
\itemprof
  \(\xi\) が有限型のときに示す。
  前に示した命題により \(X=X_1 \cap X_2\) として \(\restr{\mu}{X_1} , \restr{\mu}{X_2} , \restr{\mu}{X_1 \cap X_2}\) が制限束の \(E\)-orientation となるから、これらに対して成り立つときに \(X\) で成り立つことを示せば自明束の場合に帰着し示せる。
  \[\begin{matrix}
    E^{*}(X_1) \oplus E^{*}(X_2) & \to & E^{*+n}(\text{Thom}(\restr{\xi}{X_1})) \oplus E^{*+n}(\text{Thom}(\restr{\xi}{X_2})) \\
    \downarrow & & \downarrow \\
    E^{*}(X_1 \cap X_2) & \to & E^{*+n}(\text{Thom}(\restr{\xi}{X_1 \cap X_2})) \\
    \downarrow & & \downarrow \\
    E^{*+1}(X) & \to & E^{*+1+n}(\text{Thom}(\xi)) \\
    \downarrow & & \downarrow \\
    E^{*+1}(X_1) \oplus E^{*+1}(X_2) & \to & E^{*+1+n}(\text{Thom}(\restr{\xi}{X_1})) \oplus E^{*+1+n}(\text{Thom}(\restr{\xi}{X_2})) \\
    \downarrow & & \downarrow \\
    E^{*+1}(X_1 \cap X_2) & \to & E^{*+1+n}(\text{Thom}(\restr{\xi}{X_1 \cap X_2})) \\
    \downarrow & & \downarrow \\
  \end{matrix}\]
  より five-lemma を適用することでわかる。
\end{Proof}

\end{document}