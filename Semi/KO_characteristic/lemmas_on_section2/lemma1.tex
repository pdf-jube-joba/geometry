\documentclass[dvipdfmx]{jsarticle}
% パッケージ
\usepackage{amsthm}
\usepackage{amsmath,amssymb,mathrsfs}
\usepackage{color}
\usepackage{tikz}

% 定理環境
%% 本体
\theoremstyle{definition}
\newtheorem*{tDefinition}{定義}
\newtheorem*{tTheorem}{定理}
\newtheorem*{tProof}{証明}
\newtheorem*{tNotation}{記法}
\newtheorem*{tRemark}{注意}
\newtheorem*{tWhen}{設定}

\newenvironment{Mini}{
  \begin{minipage}[t]{1\hsize}
  \setlength{\parindent}{10pt}
  \begin{itemize}
  \setlength{\labelsep}{10pt}
}{
  \end{itemize}
  \vspace{5pt}
  \end{minipage}
}

\newenvironment{Definition}[1][\quad]{
  \begin{tDefinition}
  #1 \\
  \begin{Mini}
}{
  \end{Mini}
  \end{tDefinition}
}

\newenvironment{Theorem}[1][\quad]{
  \begin{tTheorem}
  #1 \\
  \begin{Mini}
}{
  \end{Mini}
  \end{tTheorem}
}

\newenvironment{Proof}[1][\quad]{
  \begin{tProof}
  #1 \\
  \begin{Mini}
      }{
  \end{Mini}
  \end{tProof}
}

\newenvironment{When}{
  \begin{tWhen}
  \quad \\
  \begin{Mini}
}{
  \end{Mini}
  \end{tWhen}
}

\newenvironment{Remark}{
  \begin{tRemark}
}{
  \end{tRemark}
}

%% マーク
\newcommand{\itemwhen}{\item[\(\bigcirc\)]}
\newcommand{\itemnote}{\item[!]}

\newcommand{\itemdefi}{\item[\(\square\)]}
\newcommand{\itemprop}{\item[\(\vartriangleright\)]}
\newcommand{\itemand}{\item[\(-\)]}

\newcommand{\itemprof}{\item[\(\because\)]}
\newcommand{\itemthen}{\item[\(\rightsquigarrow\)]}

\newcommand{\itemenum}{\item[\(+\)]}
\newcommand{\itembase}{\item[\(\bullet\)]}
\newcommand{\itemwith}{\item[\(-\)]}

% 記述関係
\newenvironment{indentblock}{
  \\
  \hspace*{5mm}
  \begin{minipage}{0.8\textwidth}
}{
  \end{minipage}
  \\
}

%コマンド関連
\newcommand{\place}{\_}
\newcommand{\restr}[2]{\left. {#1} \right| _{#2}}
\newcommand{\txt}{\texttt}

%% 宣言
\newcommand{\declare}[1]{\textcolor[rgb]{0.1, 0.8, 0.2}{#1 }}
\newcommand{\For}{\declare{For}}
\newcommand{\Define}{\declare{Define}}
\newcommand{\Let}{\declare{Let}}
\newcommand{\IfHold}{\declare{If}}
\newcommand{\Then}{\declare{Then}}
\newcommand{\Take}{\declare{Take}}
\newcommand{\Fix}{\declare{Fix}}
\newcommand{\Return}{\declare{Return}}

%% 置く
\newcommand{\WIP}{\textcolor{red}{工事中}}
\newcommand{\SORRY}{\textcolor{red}{わかりませんでした}}
\newcommand{\ADMIT}{\textcolor{blue}{認めます}}
\newcommand{\AIMAI}[1]{\textit{#1}}

\begin{document}
\section*{ホモトピーに関する補題}
\paragraph*{概要}
ホモトピーを考察するうえで使いやすい補題をまとめる。
\(X\) 上の二つのベクトル束の複体 \(\xi_i = 0 \to E^i_0 \to \ldots \to E^i_n \to 0\) が homotopic であることは、 \(X \times I\) 上のベクトル束の複体 \(\xi\) であって \(r_t : X \to X \times I := x \mapsto (x , t)\) によって \(r_i^* \xi \cong \xi_i\) を満たすものが存在することとして定義されていた。
ここでは二つの複体が \(E^1_j = E^2_j =: E_j\) を満たすさいに、 もし \(f_t^i : t \in [0,1] , E_i \to E_{i+1}\) であって \(t\) に対して連続なものが複体になればそこから homotopy が構成できることを示す。

\begin{Definition}
  \itemwhen
  \Fix \(X\) : locally compact Hausdorff space
  \itemdefi
  \For \(E , F\) : vector bundle over \(X\) , \(f_t\) : \(t \in [0,1] . \text{bundle morphism} \, E \to F\) \\
  \Define \(f_t\) is continuous on \(t\) := if \(t \mapsto f_t\) : \(I \to \text{Map}(E,F)\) is continuous
\itemprop
\Then this is equivalent to \((t , x) \mapsto f_t(x)\) : \(I \times E \to F\) is continuous
\end{Definition}

\begin{Theorem}
  \itemwhen
  \Fix \(X\) : locally compact Hausdorff space
  \itemprop
  \For \(E_i\) : vector bundle over \(X\) , \(f^i_t\) : continuous on \(t \in [0,1]\). \(E_i \to E_{i+1}\) \\
  \IfHold for each \(t\) : \([0,1]\) , \(0 \to E_0 \overset{f^1_t}{\to} E_1 \overset{f^2_t}{\to} \cdots \overset{f^n_t}{\to} E_n \to 0\) is complex \\
  \Let \(\tilde{f^i_t}\) : \(E^i \times I \to E^{i+1} \times I\) := \((x,t) \mapsto (f_t(x) , t)\) \\
  \Then \(\xi = (0 \to E_0 \times I \overset{\tilde{f^1_t}}{\to} \to E_1 \times I \overset{\tilde{f^2_t}}{\to} \cdots \overset{\tilde{f^n_t}}{\to} E_n \times I \to 0)\) is complex
  \itemprop
  \IfHold \(\xi\) is compact support \\
  \Then \(r_0^* \xi , r_1^* \xi\) is compact support and homotopic
\end{Theorem}

\begin{Proof}
  \itemprof
  二乗して消えていることが条件よりわかる。
  \itemprof
  \(\text{supp} \, \xi \cap X \times \{i\}\) が \(r_i^* \xi\) の support になるから成り立つ。
\end{Proof}

\end{document}