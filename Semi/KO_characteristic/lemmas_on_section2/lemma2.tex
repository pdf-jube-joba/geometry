\documentclass[dvipdfmx]{jsarticle}
% パッケージ
\usepackage{amsthm}
\usepackage{amsmath,amssymb}
\usepackage{color}
\usepackage{tikz}

% 定理環境
%% 本体
\theoremstyle{definition}
\newtheorem*{tDefinition}{定義}
\newtheorem*{tTheorem}{定理}
\newtheorem*{tProof}{証明}
\newtheorem*{tNotation}{記法}
\newtheorem*{tRemark}{注意}
\newtheorem*{tWhen}{設定}

\newenvironment{Mini}{
  \begin{minipage}[t]{0.9\hsize}
  \setlength{\parindent}{12pt}
  \begin{itemize}
  \setlength{\labelsep}{10pt}
}{
  \end{itemize}
  \vspace{5pt}
  \end{minipage}
}

\newenvironment{Definition}{
  \begin{tDefinition}
  \begin{Mini}
}{
  \end{Mini}
  \end{tDefinition}
}

\newenvironment{Theorem}{
  \begin{tTheorem}
  \begin{Mini}
}{
  \end{Mini}
  \end{tTheorem}
}

\newenvironment{Proof}{
  \begin{tProof}
  \begin{Mini}
      }{
  \end{Mini}
  \end{tProof}
}

\newenvironment{When}{
  \begin{tWhen}
  \begin{Mini}
}{
  \end{Mini}
  \end{tWhen}
}

%% マーク
\newcommand{\itemwhen}{\item[\(\bigcirc\)]}
\newcommand{\itemnote}{\item[!]}

\newcommand{\itemdefi}{\item[\(\square\)]}
\newcommand{\itemprop}{\item[\(\vartriangleright\)]}
\newcommand{\itemand}{\item[\(-\)]}

\newcommand{\itemprof}{\item[\(\because\)]}
\newcommand{\itemthen}{\item[\(\rightsquigarrow\)]}

\newcommand{\itemenum}{\item[\(\bullet\)]}
\newcommand{\itemwith}{\item[\(-\)]}

% 記述関係
\newenvironment{indentblock}{
  \\
  \hspace*{5mm}
  \begin{minipage}{0.8\textwidth}
}{
  \end{minipage}
  \\
}

%コマンド関連

\newcommand{\WIP}{\textcolor{red}{工事中}}
\newcommand{\SORRY}{\textcolor{red}{わかりませんでした}}
\newcommand{\ADMIT}{\textcolor{blue}{認めます}}
\newcommand{\AIMAI}[1]{\textit{#1}}

\begin{document}
\section*{代表元のとり方についての命題}
\begin{Theorem}[複体と代表元]
\itemprop
  \For \(\xi = (0 \to E_0 \overset{\alpha_1}{\to} E_1 \to 0)\) : cpt.supp. cpx over \((X,A)\) \\
  \Then \(\exists\) \(\xi^\prime = (0 \to \epsilon^n \to E \to 0)\) : cpt.supp. cpx over \((X,A)\) s.t. \(\xi \sim_{D}^{(X,A)} \xi^\prime\)
\itemprop
  \For \(\xi = (0 \to E_0 \overset{\alpha_1}{\to} E_1 \overset{\alpha_2}{\to} E_2 \overset{\alpha_3}{\to} E_3 \to \cdots \to E_n \to 0)\) : cpt.supp cpx over \((X,A)\) \\
  \For \(\xi^\prime = (0 \to E_0 \overset{\alpha_1}{\to} E_1 \to E \to 0)\) : acyclic complex \\
  \Then \(\exists\) \(\zeta = (0 \to E \overset{\alpha}{\to} E_2 \overset{\alpha_3}{\to} E_3 \to \cdots \to E_n \to 0)\) : cpt.supp cpx over \((X,A)\) s.t. \(\xi \sim_{D}^{(X,A)} \zeta\)
\itemprop
  \For \(\xi = (0 \to E_0 \overset{\alpha_1}{\to} E_1 \overset{\alpha_2}{\to} E_2 \to \cdots \to E_n \to 0)\) : cpt.supp cpx over \((X,A)\) , \(n \leq 3\) \\
  \Then \(\exists\) \(\xi^\prime = (0 \to E^\prime_0 \overset{\alpha^\prime_1}{\to} E^\prime_1 \overset{\alpha^\prime_2}{\to} E^\prime_2 \to \cdots \to E^\prime_n \to 0)\) s.t. \(\xi \sim_{D}^{(X,A)} \xi^\prime\) and \(\alpha^\prime_1\) is injective
\itemprop
  \For \(x\) : \(K(X)\) \\
  \Then there exists \(\xi = (0 \to \epsilon^n \to E \to 0)\) : cpt.supp cpx over \((X,A)\) such that \([\xi] = x\)
\end{Theorem}

\newpage
\section*{証明}
\begin{Proof}[複体と代表元(1)]
\itemprof
  \(\text{supp} \, \xi\) は compact なので \(s\) : \(E_0 \to \epsilon^n\) であって \(\text{supp} \,\xi\) 上単射となるものがとれる。
  \(\alpha (x) = (\alpha_1(x) , s(x))\) : \(E_0 \to E_1 \oplus \epsilon^n\) は全体で単射となるから、
  \(
    \xi^\prime = (0 \to E_0 \overset{\alpha}{\to} E_1 \oplus \epsilon^n \overset{\beta}{\to} E \to 0)
  \)
  なる完全な複体を得る。
  \(\beta(x,y) = \pi_1(x) + \pi_2(y)\) とおく。
  このとき \(\alpha_1\) が同型となるような \(x \in X\) において \(\pi_2\) が同型となることが次のようにしてわかる。
  完全性により次元を考えると \(\pi_2\) が単射であることを示せばよい。
  \(x\) : \(\ker \pi_2\) をとると \(\beta(0,x) = 0\) より \(y\) : \(E_0\) が存在して \((\alpha_1(y) , s(y)) = (0,x)\) とかけるが \(\alpha_1\) の同型により \(y = 0\) とわかり \(x = 0\) と示せた。
  特に、 \(\xi\) が完全な点では \(0 \to \epsilon^n \overset{- \pi_2}{\to} E \to 0\) も完全であり、 \((X,A)\) 上の複体となる。
\itemthen
  \(\xi \sim_{D}^{(X,A)} 0 \to \epsilon^n \overset{- \pi_2}{\to} E \to 0\) を示す。
  \(\xi\) に acyclic な複体 \(0 \to \epsilon^n \overset{1}{\to} \epsilon^n \to 0\) と \(0 \to 0 \to E \overset{1}{\to} E \to 0\) を足せば次の複体となる。
  \[\xi_0 = 0 \to E_0 \oplus \epsilon^n \overset{
    \begin{pmatrix}
      \alpha_1 & 0\\
      0 & 1 \\
      0 & 0
    \end{pmatrix}
  }{\to} E_1 \oplus \epsilon^n \oplus E \overset{
    \begin{pmatrix}
      0 & 0 & 1
    \end{pmatrix}
  }{\to} E \to 0\]
  次の \(t\) : \([0,1]\) で parametrize された列 が \((X,A)\) 上の compact support な複体であることを示せば、
  \[0 \to E_0 \oplus \epsilon^n \overset{
    \begin{pmatrix}
      \alpha_1 & 0\\
      t s & 1 - t \\
      0 & - t \pi_2
    \end{pmatrix}
  }{\to} E_1 \oplus \epsilon^n \oplus E \overset{
    \begin{pmatrix}
      t^2 \pi_1 & t \pi_2 & 1 - t
    \end{pmatrix}
  }{\to} E \to 0\]
  これを homotopy として \(\xi_0\) は \(0 \to \epsilon^n \overset{-\pi_2}{\to} E \to 0\) と \(\xi ^\prime\) の直和へ homotopic とわかる。
  \(\xi^\prime\) が完全であるから、 \(K(X)\) の元として \(\xi\) は \(0 \to \epsilon^n \overset{-\pi_2}{\to} E \to 0\) に同値であることがわかる。
  この複体の support を考えると \(\alpha_1\) が同型となる点では \(\pi_2\) も同型であるから support が compact である。
\itemthen
  複体となっていることを示す。
  すなわち、写像の合成が \(0\) であることを示す。
  \[
    \begin{pmatrix}
      t^2 \pi_1 & t \pi_2 & 1 - t
    \end{pmatrix} \cdot \begin{pmatrix}
      \alpha_1 & 0\\
      t s & 1 - t \\
      0 & - t \pi_2
    \end{pmatrix} =
    \begin{pmatrix}
      t^2 \pi_1 \alpha + t^2 \pi_2 s + 0 & 0 + (1-t) t \pi_2 - (1-t) t \pi_2
    \end{pmatrix} = (0,0,0)
  \]
\itemthen
  homotopy の support が compact であることを示す。
  これには \((\text{supp} \, \xi)^c \times I\) 上で完全であることを示せばよい。
  \(t=0\) では確かに完全であるからよいため、 \(t \not = 0\) とする。
  このとき \(\alpha_1 , \pi_2\) は同型である。
  \((x,y,z)\) : \(E_1 \oplus \epsilon^n \oplus E\) が \(t^2 \pi_1(x) + t \pi_2(y) + (1-t) z = 0\) を満たすとする。
  \((u,v)\) : \(E_0 \oplus \epsilon^n\) であって \((x,y,z)\) にうつるものを探せばよい。
  \(z = \pi_2(\tilde{z}) , x = \alpha_1(\tilde{x})\) をとる。
  このとき \((u,v) = (\tilde{x} , - \frac{1}{t} \tilde{z})\) は \((\alpha_1(\tilde{x}) , t s(\tilde{x}) - (1-t) \frac{1}{t} \tilde{z} , \pi_2(\tilde{z})) = (x , y^\prime := t s(\tilde{x}) - (1-t) \frac{1}{t} \tilde{z} , z)\) にうつる。
  \(\pi_2\) の同型性により、 \(t^2 \pi_1(x) + t \pi_2(y^\prime) + (1-t) z = 0\) なら \(y^\prime = y\) とわかり示される。
  実際、 \(t \pi_2(y^\prime) = t^2 \pi_2 s(\tilde{x}) - (1-t) \pi_2(\tilde{z}) = - t^2 \pi_1 \alpha_1 (\tilde{x}) - (1-t) z\) よりよい。
\end{Proof}

\begin{Proof}[(2)]
\itemprof
  これは商空間と同様にして定義できるためよい。
\end{Proof}

\begin{Proof}[(3)]
\itemprof
  \(s\) : \(E_0 \to \epsilon^n\) であって \(\text{supp} \, \xi\) 上単射かつ support が compact となるものをとれば、以前の証明と同様に \(E_0 \to E_1 \oplus \epsilon^n\) が単射となる。
  \(\rho\) : \(X \to \mathbb{R}\) であって compact set をのぞいて \(\rho = 1\) かつ \(s\) の support 上 \(0\) となるものをとる。
  \(\xi\) は \(K(X)\) 上で acyclic な複体 \(0 \to 0 \to \epsilon^n \overset{1}{\to} \epsilon^n \to 0 \to 0 \cdots\) を直和したものに同型である。
  後者の複体は \(\rho\) の定義を考えると \(0 \to 0 \to \epsilon^n \overset{\rho}{\to} \epsilon^n \to 0 \to 0 \cdots\) なる複体へ \((1 - t) + t \rho\) により homotopic である。
  この homotopy は \(\rho\) がある compact set を除き \(1\) に等しいことから compact support である。
  従って \(\xi\) は次の複体に同値である。
  \[0 \to E_0 \overset{
    \begin{pmatrix}
      \alpha_1 \\
      0
    \end{pmatrix}
  }{\to} E_1 \oplus \epsilon^n \overset{
    \begin{pmatrix}
      \alpha_2 & 0 \\
      0 & \rho
    \end{pmatrix}
  }{\to} E_2 \oplus \epsilon^n \overset{
    \begin{pmatrix}
      \alpha_3 , 0
    \end{pmatrix}
  }{\to} E_3 \to \cdots\]
  であるが、これは
  \[0 \to E_0 \overset{
    \begin{pmatrix}
      \alpha_1 \\
      s
    \end{pmatrix}
  }{\to} E_1 \oplus \epsilon^n \overset{
    \begin{pmatrix}
      \alpha_2 & 0 \\
      0 & \rho
    \end{pmatrix}
  }{\to} E_2 \oplus \epsilon^n \overset{
    \begin{pmatrix}
      \alpha_3 , 0
    \end{pmatrix}
  }{\to} E_3 \to \cdots\]
  へ
  \[0 \to E_0 \overset{
    \begin{pmatrix}
      \alpha_1 \\
      t s
    \end{pmatrix}
  }{\to} E_1 \oplus \epsilon^n \overset{
    \begin{pmatrix}
      \alpha_2 & 0 \\
      0 & \rho
    \end{pmatrix}
  }{\to} E_2 \oplus \epsilon^n \overset{
    \begin{pmatrix}
      \alpha_3 , 0
    \end{pmatrix}
  }{\to} E_3 \to \cdots\]
  により homotopic である。
  support について考察すると、この homotopy は \(\rho = 1\) を満たす点では \(s = 0\) より完全であるからよい。
\end{Proof}

\begin{Proof}[(4)]
\itemprof
  前三つの命題を一般の複体に順々に適用することで得られる。
\end{Proof}

\begin{Theorem}
\itemprop
  \For \(\xi = (\cdots E^{n} \overset{f_{n}}{\to} E^{n+1} \cdots)\) : cpt.supp cpx over \((X,A)\) with metric \\
  \Let \(\alpha\) : \(\oplus_n \xi^{2n} \to \oplus_n \xi^{2n+1}\) := \(\sum_{n} f_{2n} + f_{2n+1}^{\dagger}\) \\
  \Let \(\chi(\xi)\) : cpt.supp cpx over \((X,A)\) := \(0 \to \oplus_n \xi^{2n} \overset{\alpha}{\to} \oplus_n \xi^{2n+1} \to 0\) \\
  \Then \(\xi \sim_{D}^{(X,A)} \chi(\xi)\)
\end{Theorem}

\begin{Proof}
\itemprof
  \(\chi\) : \(L_{n}(X,A) \to L_{1}(X,A)\) が metric が homotopy を保つことから定まり、 \(L_{n}(X,A) \to L_{n+1}(X,A)\) が同型であることと \(L_{1}(X,A) \to L_{n}(X,A)\) の left inverse が \(\chi\) になることから確かにわかる。
\end{Proof}

\end{document}