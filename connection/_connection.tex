\begin{Definition}
\itemnote
  一般のファイバー束の場合に接続(connection)まわりを定める。
\itemwhen \((\pi , E , B)\) : fiber bundle
\itemdefi
  connection := splitting of \(0 \to \mathcal{V}E \to TE \to \pi^*TM\) : vector bundle exact sequence on \(E\)
\itemdefi
  connection \(1\) form := \(\omega\) : \(\Omega^(E , \mathcal{V}E)\) such that
  forall (\(X\) : vertical vector fields) , \(\iota(X)\omega = X\)
\itemwhen \(\omega\) : connection \(1\) form
\itemdefi
  define horizontal vector field := \(X\) : \(\Gamma(TE)\) such that \(\omega(X) = 0\) \\
  define \(\mathcal{H}E\) := kernel of  \(\omega\) as \(TE \to \mathcal{V}E\) \\
\itemprop
  let \(\mathcal{H}E \to \pi^* TB\) := restriction of \(TE \to \pi^*TB\) \\
  then this is isomorphism
\itemdefi
  for (\(X\) : \(\mathcal{X}(B)\)) ,\\
  horizontal lift of \(X\) (\(X_E\)) : \(\Gamma(\mathcal{H}E)\) := unique by \(\pi_* X_E = X\)
\end{Definition}

\begin{Definition}
\itemnote
  この接続に対して曲率を定義する。
\itemwhen \((\pi , E , B)\) : fiber bundle
\itemwhen \(\omega\) : connection \(1\) form
\itemdefi
  curvature (\(\Omega\)) : \(\Gamma(\pi^*T^*B \otimes \mathcal{V}E)\) := by formula...
  \(\forall\) (\((X , Y)\) : \(\mathcal{X}(M)\)) , \(\Omega(\pi^*X , \pi^*Y) = [X , Y]_E - [X_E , Y_E]\)
\itemprop
  let \(\Omega\) := curvature of \(\omega\) \\
  then \(\Omega(\pi^*(f X) , Y) = (\pi^* f) \Omega(X , Y)\)
\end{Definition}

\begin{Definition}
\itemnote
  主束の場合の接続と接続形式を定義する。
\itemdefi
  connection (\(\tau\)) := \(G\) invariant \(\dim (M)\) field of \(TP\) such that \(\pi_* : \tau_p \to T_{\pi(p)}X\) is isomorphism
\itemdefi
  for (\(\tau\) : connection) ,\\
  connection \(1\) form (\(\omega\)) : \(\Omega^1(P ; \mathfrak{g})\) :=
  \begin{indentblock}
    for (\(p\) : \(P\)) ,\\
    let \(f\) : \(T_p P \to \mathcal{V}_p P\) := projection induced by \(\tau_p\) \\
    let \(g\) : \(\mathcal{V}_p P \to \mathfrak{g}\) := induced by isomorphism of \(P \times \mathfrak{g} \to \mathcal{V}P\) \\
    return \(g \circ f\)
  \end{indentblock}
\itemprop
  for (\(\tau\) : connection) ,\\
  then \(T_p P = \mathcal{V}_p P \oplus \tau_p\)
\itemdefi
  for (\(\tau\) : connection) ,\\
  define \(\text{splitting of } 0 \to \mathcal{V}P \to TP \to \pi^*P \to 0\) := \\
  from projection : \(TP \to \mathcal{V}P\) := induced by \(\tau\)
\end{Definition}

\begin{Theorem}
\itemnote
  ファイバー束での定義と主束での定義が一致することを見る
\itemprop
  \WIP
\end{Theorem}

\begin{Definition}
\itemprop
  for (\(\tau\) : connection) ,\\
  for (\(X\) : \(\mathcal{X}(M)\)) ,\\
  define \(X_{P}\) : \(\Gamma(HP)\) := unique by \(\pi_* X_P = X\)
\end{Definition}

\begin{Theorem}
\itemnote
  接続・接続形式・分裂が対応することを示す。
\itemprop
  for (\(\omega\) := connection \(1\) form of \(\tau\)) ,\\
  \(\forall\) (\(V\) : \(\mathfrak{g}\)) , \(\omega(\tilde{V}) \equiv V\) and \\
  \(\forall\) (\(g\) : \(G\)) , \(R_g^* \omega = \text{Ad}_{g^{-1}}(\omega)\)
\itemprop
  let \(\text{connection } \to \{\omega : \Omega^1(P ; \mathfrak{g}) \mid \omega(\tilde{V}) \equiv V , R_g^* \omega = \text{Ad}_{g^{-1}}(\omega)\}\) := defined above \\
  this is isomorphism
\itemprop
  let \(\text{connection } \to \text{splitting of } 0 \to \mathcal{V}P \to TP \to \pi^*P \to 0\) := defined above \\
  this is isomorphism
\itemdefi
  We identifying all of this
\end{Theorem}

\begin{Definition}
\itemnote
  接続の曲率を定義する
\itemdefi
  for (\(\omega\) : connection \(1\) form) ,\\
  curvature : \(\Omega^2(P ; \mathfrak{g})\) := \(d \omega + [\omega , \omega]\)
\end{Definition}

\begin{Theorem}
\itemnote
  曲率の性質を示す
\itemdefi
  \(\Omega(X , Y) = [X , Y]_P - [X_P , Y_P]\)
\itemdefi
  \(\Omega(\tilde{V} , 0) = 0\)
\itemdefi
  \(g^* \Omega = \text{Ad}_{g^{-1}}(\Omega)\)
\end{Theorem}

\begin{Theorem}
\itemnote
  曲率の局所的な表示について
\itemwhen fix \(G \subset \text{GL}(k , \mathbb{R})\)
\itemdefi
  \(\omega_{ij}\) , \(\Omega_{ij}\) : \(\Omega^i(P)\) := \((i,j)\) component of \(\omega\) \(\Omega\)
\itemprop
  \(\Omega_{i,j} = d \omega_{ij} + \sum_k \omega_{ik} \wedge \omega_{kj}\)
\end{Theorem}

\begin{Theorem}
\itemdefi
  for (\(e_i , e_j\) : \(\mathbb{R}^n\)) ,\\
  \(e_i \wedge e_j\) : \(\mathbb{R}^n \to \mathbb{R}^n\) := \(v \mapsto \langle e_i , v \rangle e_j - \langle e_j , e \rangle e_i\)
\itemdefi
  \(\omega = - \sum_{i \lneqq j} \omega_{ij} e_i \wedge e_j\)
\end{Theorem}

\begin{Definition}
\itemnote
  共変微分を定義する
\itemdefi
  for (\((\pi : E \to M)\) : vector bundle) ,\\
  covariant derivative := \(\nabla\) : \(\mathbb{R}\text{-linear } \Gamma(E) \to \Gamma(T^*M \otimes E)\) such that \(\forall (f : C(M)) (e : \Gamma(E))\) , \(\nabla(f e) = d f \otimes e + f \nabla e\)
\itemdefi
  for (\(\nabla\) : connection) (\(X\) : vector field) ,\\
  \(\nabla_X\) : \(\Gamma(E) \to \Gamma(E)\) := \(\iota (X) \circ \nabla\)
\end{Definition}

\begin{Theorem}
\itemnote
  共変微分の性質を示す。
\itemprop
  forall (\(\nabla_1\) , \(\nabla_2\) : connection) ,\\
  \(\nabla_1 - \nabla_2\) is \(C(M)\) linear
\itemprop
  forall (\(\nabla\) : connection) (\(\alpha\) : \(\Omega^1(M)\)) ,\\
  \(\nabla + \alpha\) is connection
\end{Theorem}

\begin{Theorem}
\itemnote
  for (\(\omega\) : connection on \((P , M , \pi)\) : principal \(G\) bundle) (\(U\) : trivialization) (\(s\) : local section of \(U\)) (\(A\) : \(U \to G\)) ,\\
  let \(\phi_A\) : \(P \mid_{U} \to P \mid_{U}\) := \(x \mapsto x \cdot A(x)\) \\
  then \(s^* \phi_A^* (\omega \mid_{U}) = \text{Ad}_{A^{-1}} (s^* (\omega \mid_{U})) + A^{-1}dA\)
\end{Theorem}

\begin{Definition}
\itemnote
  共変微分をフレーム束の接続から構成する
\itemwhen \((\pi : E \to M)\) : vector bundle
\itemwhen \(\omega\) : connection form on \(P_{GL}(E)\)
\itemdefi
  for (\(U\) : trivialization) ,\\
  define \(\nabla_U\) : covariant derivative on \(U\) :=
  \begin{indentblock}
    take (\(\mathcal{E} = (e_1 , \ldots , e_n)\) : local frame on \(U\)) ,\\
    let \(\tilde{\omega}\) : \(A^1(U ; \mathfrak{g})\) := \(\mathcal{E}^*\omega\) ,\\
    let \(\tilde{\omega}_{ij}\) : \(A^1(U)\) := \((i,j)\) component of \(\tilde{\omega}\) ,\\
    define induced by \(e_j \mapsto \sum_j \tilde{\omega}_{ji} \otimes e_j\)
  \end{indentblock}
\itemdefi
  define \(\nabla\) : covariant detivative on \(E\) := unique by forall (\(U\) : trivialization) (\(s\) : \(\Gamma(E)\)) , \((\nabla s) \mid_{U} = \nabla_U (s \mid_U)\)
\end{Definition}

\begin{Theorem}
\itemprop
  \AIMAI{this construction can be in the same way when \(E\) is metric , oriented}
\end{Theorem}

\begin{Theorem}
\itemnote
  この構成が向き付けられた計量付きベクトル束の場合にはよい性質を持つことを示す。
\itemprop
  for (\((\pi : E \to M)\) : ori.met.vector bundle) (\(\omega\) : connection on \(P_{SO}(E)\)) ,\\
  let \(\nabla\) : covariant derivative on \(E\) := defined above ,\\
  then forall (\(V\) : \(\mathcal{X}(M)\)) (\(s_1\) , \(s_2\) : \(\Gamma(E)\)) , \(V \langle s_1 , s_2 , \rangle = \langle \nabla_V s_1 , s_2 \rangle + \langle s_1 , \nabla_V s_2 \rangle\)
\end{Theorem}

\begin{Definition}
\itemdefi
  for (\(\nabla\) : covariant derivative on \((E \to M)\)) ,\\
  define \(\tilde{\nabla}\) : \(\Omega^i(M , E) \to \Omega^{i + 1}(M , E)\) := induced by \(\alpha \otimes e \mapsto d \alpha \otimes e + (-1)^i \alpha \wedge \nabla e\) \\
  define \(R\) := \(\tilde{\nabla} \circ \nabla\) \\
  define \(R_{V , W}\) : \(\Gamma(E) \to \Gamma(E)\) := \(\)
\end{Definition}

\begin{Theorem}
\itemprop
  \(R e_i = \sum_j \tilde{\Omega}_{ij} \otimes e_j\)
\itemprop
  \(R\) is tensor
\itemprop
  forall (\(U\) : open subset of \(M\)) (\(V , W\) : \(\mathcal{X}(M) \mid_{U}\)) (\(e\) : \(\Gamma(E \mid_{U})\)) , \(R_{V , W}e = (\nabla_V \nabla_W - \nabla_W \nabla_V - \nabla_{[V , W]}) e\)
\itemprop
  forall (\(U\) : open subset of \(M\)) (\(V , W\) : \(\mathcal{X}(M) \mid_{U}\)) (\(s_1 , s_2\) : \(\Gamma(E \mid_{U})\)) , \(\langle R_{V, W} s_1 , s_2 + \langle s_1 , R_{V, W} s_2 \rangle = 0\)
\end{Theorem}